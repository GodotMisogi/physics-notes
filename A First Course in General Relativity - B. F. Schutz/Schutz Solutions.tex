% Bernard Schutz Solutions Manual

\documentclass{report}
\usepackage[paperwidth=6in, paperheight=8in, top = 20mm, bottom = 18mm, left=10mm, right = 10mm]{geometry}
\usepackage{concmath}
\usepackage[T1]{fontenc}

%MF mode dpdfezzz is for 8000dpi
\pdfpkmode{dpdfezzz}
\pdfpkresolution=8000

\usepackage{graphicx}
\usepackage{amsmath}
\usepackage{amssymb}
\usepackage{amsthm}
\usepackage{tensor}
\usepackage{physics}
\usepackage[none]{hyphenat}
\usepackage{tkz-euclide}
\usepackage{tikz}

\usepackage{etoolbox} % provides \patchcmd macro
\makeatletter % modify the "headings" page style
\patchcmd{\ps@headings}{{\slshape\rightmark}\hfil\thepage}{\thepage\hfil}{}{}
\makeatother
\pagestyle{headings}  % load the (re-defined) "headings" page style (default: "plain")

\usetikzlibrary{calc,arrows}
\usetkzobj{all}
\DeclareMathOperator{\arctanh}{arctanh}
\theoremstyle{definition}

\usetikzlibrary{decorations.markings}
\usepackage{xcolor}

\definecolor{gridcolor}{RGB}{200 200 200} 
\definecolor{gridlabelcolor}{RGB}{131 24 255} 

\def\rnd#1{
    \pgfmathprintnumberto[precision=2]{#1}{\temp}\temp
}

\newtheorem{chapter1}{Problem}
\newcounter{subpart1}[chapter1]

\usepackage{color}
\definecolor{mygrey}{gray}{0.1}
\color{mygrey}

\begin{document}

\title{Solutions to \\A First Course in General Relativity \\ (2nd Edition)\\ by Bernard F. Schutz}

\author{Arjit Seth}
\date{}

\maketitle

\chapter{Special Relativity}

\begin{chapter1}\label{prob:1}

\end{chapter1}

\begin{chapter1}\label{prob:2}
	
\end{chapter1}

\begin{chapter1}\label{prob:3}
	
\end{chapter1}

\begin{chapter1}\label{prob:4}
	\stepcounter{subpart1}

	% Greek summation
	\begin{subequations}
		\begin{equation}
			(\alph{subpart1}) \hspace{10 pt}
			\displaystyle\sum_{\alpha=0}^{3} V_{\alpha}\Delta{x^\alpha} = V_{0}\Delta{t} + V_{1}\Delta{x} + V_{2}\Delta{y} + V_{3}\Delta{z}
		\end{equation}
		\stepcounter{subpart1}
		
		% Latin summation
		\begin{equation}
			(\alph{subpart1}) \hspace{10 pt}
			\displaystyle\sum_{i=1}^{3} (\Delta{x^i})^2 = (\Delta{x})^2 + (\Delta{y})^2 + (\Delta{z})^2
		\end{equation}
	\end{subequations}	
\end{chapter1}

\begin{chapter1}\label{prob:5}
	
\end{chapter1}

\begin{chapter1}\label{prob:6}
	
\end{chapter1}

\begin{chapter1}\label{prob:7}
	
\end{chapter1}

\begin{chapter1}\label{prob:8}
	
\end{chapter1}

\begin{chapter1}\label{prob:9}
	
\end{chapter1}

\begin{chapter1}\label{prob:10}
	For this question, we must evaluate the interval between the two space-time points using:
	\begin{subequations}
		\begin{equation}
			(\Delta{s})^2 = -(\Delta{t})^2 + (\Delta{x})^2 + (\Delta{y})^2 + (\Delta{z})^2
		\end{equation}
		\stepcounter{subpart1}
		(\alph{subpart1})
		The separation is null.
		\begin{equation}
			(\Delta{s})^2 = -(-1-0)^2 + (1-0)^2 + (0-0)^2 + (0-0)
			^2 = 0
		\end{equation}
		\stepcounter{subpart1}
		(\alph{subpart1})
		The separation is spacelike.
		\begin{equation}
			(\Delta{s})^2 = -(-1+1)^2 + (1-1)^2 + (0+1)^2 + (2-0)^2 = 5
		\end{equation}
		\stepcounter{subpart1}
		(\alph{subpart1})
		The separation is timelike.
		\begin{equation}
			(\Delta{s})^2 = -(5-6)^2 + (0-0)^2 + (1-1)^2 + (0-0)^2 = -1
		\end{equation}
		\stepcounter{subpart1}
		(\alph{subpart1})
		The separation is null.
		\begin{equation}
			(\Delta{s})^2 = -(4+1)^2 + (1-1)^2 + (-1+1)^2 + (6-1)^2 = 0
		\end{equation}
	\end{subequations}
		
\end{chapter1}
\begin{chapter1}\label{prob:11}
	
\end{chapter1}
\begin{chapter1}\label{prob:12}
	
\end{chapter1}
\begin{chapter1}\label{prob:13}
	The observer is moving with velocity ${\vec v} = -0.999$ with respect to the frame of the pion. The proper time of the half-life of the pion is $2.5 \times 10^{-8}$ seconds. Therefore, the observer will measure a time-dilated value, which is:
	\begin{equation}
		t = \frac{\tau}{\sqrt{1-v^2}} = \frac{2.5 \times 10^{-8}}{\sqrt{1-(-0.999)^2}} \approx 5.6 \times 10^{-7} 
		\hspace{2 pt} \textrm{secs}
	\end{equation}

\end{chapter1}

\begin{chapter1}\label{prob:14}

	\stepcounter{subpart1}
	(\alph{subpart1})
	% Time Dilation Approximation
	The time dilation formula is given by:
	\begin{subequations}
		\begin{equation}
			\Delta{t} = \gamma \Delta{\bar{t}} = \frac{\Delta{\bar{t}}}{\sqrt{1-v^2}} 
		\end{equation}
		Expanding the Lorentz factor using a binomial expansion:
		\begin{equation}
			\frac{1}{\sqrt{1-v^2}} = 1+\frac{v^2}{2}+\frac{3v^4}{8}+\frac{5v^6}{16}+\frac{35v^8}{128}+O(v^9)
		\end{equation}
		Since $ |{\vec v}| \ll 1$, terms higher than second-order can be discarded, giving:
		\begin{equation}
			\Delta{t} = \frac{\Delta{\bar{t}}}{\sqrt{1-v^2}}\approx \left(1+\frac{v^2}{2}\right)\Delta{\bar{t}}
		\end{equation}
		\stepcounter{subpart1}
		(\alph{subpart1})
		% Length Contraction Approximation
		The length contraction formula is given by:
		\begin{equation}
			\Delta{x} = \frac{\Delta{\bar{x}}}{\gamma} = \Delta{\bar{x}}{\sqrt{1-v^2}}
		\end{equation}
		Expanding the reciprocal of the Lorentz factor using a binomial expansion:
		\begin{equation}
			\sqrt{1-v^2} = 1-\frac{v^2}{2}-\frac{v^4}{8}-\frac{v^6}{16}-\frac{5v^8}{128}+O(v^9)
		\end{equation}
		Since $|{\vec v}| \ll 1$, terms higher than second-order can be discarded, giving:
		\begin{equation}
			\Delta{x} = \Delta{\bar{x}}{\sqrt{1-v^2}} \approx \left(1-\frac{v^2}{2}\right)\Delta{\bar{x}}
		\end{equation}
		\stepcounter{subpart1}
		(\alph{subpart1})
		% Velocity Addition Approximation
		The velocity addition formula is given by:
		\begin{equation}
			w' = \frac{w+v}{1+wv}
		\end{equation}
		Expanding the denominator using a binomial expansion:
		\begin{equation}
			\frac{1}{1+wv} = 1-wv+(wv)^2-(wv)^3+(wv)^4-(wv)^5+\order{x^6}
		\end{equation}
		This gives:
		\begin{equation}
			w' = [w+v][1-wv+(wv)^2-(wv)^3+(wv)^4-(wv)^5+\order{x^6}]
		\end{equation}
		Since $|{\vec v}| \ll 1$, terms higher than first-order can be discarded, giving:
		\begin{equation}
			w' = (w+v)(1-wv) = w + v - wv(w + v)
		\end{equation}
		This equation would apply for $|{\vec w}| \ll 1$ as well.
		%% Write about error when v = 0.1	
	\end{subequations}
\end{chapter1}

\begin{chapter1}\label{prob:15}

	\stepcounter{subpart1}
	(\alph{subpart1}) and \stepcounter{subpart1}(\alph{subpart1})\\
 	% Time Dilation at high velocities
	Inserting the value of $|{\vec v}| = 1-\varepsilon$ into the time dilation and length contraction formulae, we get:
	\begin{subequations}
		\begin{equation}
			\Delta{t} = \frac{\Delta{\bar{t}}}{\sqrt{1-(1-\epsilon)^2}}=\frac{\Delta{\bar{t}}}{\sqrt{2\epsilon-\epsilon^2}}
		\end{equation}
		\begin{equation}
			\Delta{x} = \Delta{\bar{x}}{\sqrt{1-(1-\epsilon)^2}}=\Delta{\bar{x}}{\sqrt{2\epsilon-\epsilon^2}}
		\end{equation}
	\end{subequations}
	Since $\epsilon$ is very small, its square term can be neglected, giving:
	\begin{subequations}
		\begin{equation}
			\Delta{t} = \frac{\Delta{\bar{t}}}{\sqrt{2\epsilon}}
		\end{equation}
		\begin{equation}
			\Delta{x} = \Delta{\bar{x}}\sqrt{2\epsilon}
		\end{equation}	
	\end{subequations}
		% Disclaimer
	Note: Schutz's textbook has the incorrect expression presented for length contraction in this question.
	%% Write Velocity Addition Expansion
\end{chapter1}

\begin{chapter1}\label{prob:16}
	
\end{chapter1}

\begin{chapter1}\label{prob:17}
	
\end{chapter1}

\begin{chapter1}\label{prob:18}
	\stepcounter{subpart1}
	(\alph{subpart1})
	% Velocity Parametrisation
	The velocity addition formula is given by:
		\begin{subequations}
		\begin{equation}
			w' = \frac{w+v}{1+wv}
		\end{equation}
		Substituting the parametrisations $v = \tanh u$ and $w = \tanh U$, we get:
		\begin{equation}
			w' = \frac{\tanh u + \tanh U}{1+(\tanh u)(\tanh U)} = \tanh(u+U)
		\end{equation}
		Using the hyperbolic trigonometric identity:
		\begin{equation}
			\tanh(u+U) = \frac{\tanh u + \tanh U}{1+(\tanh u)(\tanh U)}
		\end{equation}	
	\end{subequations}
	So the velocity parameters add linearly.\\\\
	\stepcounter{subpart1}
	(\alph{subpart1})
	% Solving N particle velocity addition problem (NEEDS LOGICAL FIXING)
	Since the speed of the second star with respect to the first star is $v_{2,1} = 0.9 \hspace{3 pt}(c = 1)$, the velocity parameter is $u_{1} = \arctanh(0.9)$. Now, the speed of the third star with respect to the second star is $v_{3,2} = 0.9$, assuming they're all moving away from the first star in the same direction. This makes the velocity parameter $u_{2} = \arctanh(0.9)$. Inductively, the velocity parameter of the $N^{th}$ star with respect to the $(N-1)^{th}$ star will be: $v_{N,N-1} = 0.9 \implies u_{N-1} = \arctan(0.9)$ \\ 
	The velocity of the third star with respect to the reference frame of the first star is:
	\begin{equation}
	 	v_{3,1} = \tanh(u_{1}+u_{2}) = \tanh(\arctanh(0.9)+\arctanh(0.9)) = \tanh(2\arctanh(0.9))
	\end{equation}
	Since velocity parameters add linearly, we can show that:
	\begin{equation}
		\displaystyle\sum_{i=1}^{N-1}u_{i} = \arctanh(v_{N,1})
	\end{equation}
	Performing induction for {\em N} stars, the speed of the {\em Nth} star with respect to the first star is:
	\begin{equation}
	 	v_{N,1} = \tanh\left(\displaystyle\sum_{i=1}^{N-1}u_{i}\right)= \tanh[(N-1)\times\arctanh(0.9)]
	\end{equation} 
\end{chapter1}

\begin{chapter1}\label{prob:19}
	\stepcounter{subpart1}
	(\alph{subpart1})
	% Velocity Parametrisation Corollary
	A simple substitution into the Lorentz transformation equations in one dimension will yield:
	\begin{subequations}
		\begin{equation}
				\bar{t} = \frac{t-vx}{\sqrt{1-v^2}} = \frac{t-(\tanh u)x}{\sqrt{1-(\tanh u)^2}} = \frac{t-x\tanh u}{\sech u} = t\cosh⁡ u - x\sinh ⁡u
			\end{equation}
			\begin{equation}
				\bar{x} = \frac{x-vt}{\sqrt{1-v^2}} = \frac{x-(\tanh u)t}{\sqrt{1-(\tanh u)^2}} = \frac{x-t\tanh u}{\sech u} = -x\cosh⁡ u + t\sinh ⁡u
			\end{equation}
			\begin{equation}
				\bar{y} = y 
			\end{equation}
			\begin{equation}
				\bar{z} = z 
			\end{equation}
		This looks remarkably similar to the transformation pertaining to rotation of axes in a plane:
		\begin{equation}
			\begin{bmatrix}
				\bar{x} \\
				\bar{y} \\
			\end{bmatrix}
			=
			\begin{bmatrix}
				\cos\theta & -\sin\theta \\
				\sin\theta & \cos\theta \\	
			\end{bmatrix}
			\begin{bmatrix}
				x \\
				y \\
			\end{bmatrix}
		\end{equation}
		Writing the parametrised equation in matrix form:
		\begin{equation}
			\begin{bmatrix}
				\bar{t} \\
				\bar{x} \\
				\bar{y} \\
				\bar{z} \\
			\end{bmatrix}
			=
			\begin{bmatrix}
				\cosh u & -\sinh u & 0 & 0 \\
				\sinh u & \cosh u & 0 & 0 \\
				0 & 0 & 1 & 0 \\
				0 & 0 & 0 & 1 \\
			\end{bmatrix}
			\begin{bmatrix}
				t \\
				x \\
				y \\
				z \\
			\end{bmatrix}
		\end{equation}
		This implies that a Lorentz transformation can be interpreted as a ``hyperbolic rotation'' of the space-time axes.\\ \\
		\stepcounter{subpart1}
		(\alph{subpart1})
		% Invariance of the Interval in Hyperbolic Coordinates
		The invariance of the interval is described by the equation:
		\begin{equation}
		\label{interval}
			-\overline{t}^{2}+\overline{x}^2=-t^2+x^2
		\end{equation}
		Substituting the previous result:
		\begin{equation}
		\end{equation}
		\begin{equation}
			RHS = (-t^2+x^2)(\cosh^2 u - \sinh^2 u)
		\end{equation}
		From the identity:  
		\begin{equation}
			\cosh^2 u - \sinh^2⁡ u = 1
		\end{equation}
		Therefore
		\begin{equation}
			RHS=-t^2+x^2
		\end{equation}
		Hence the interval is invariant under the velocity parametrisation. \\ \\
		\stepcounter{subpart1}
		(\alph{subpart1})
		% Analogies to Euclidean Rotations
		The analog of the interval is the Euclidean distance:
		\begin{equation}
		-\overline{t}^2+\overline{x}^2=-t^2+x^2 \Leftrightarrow \overline{x}^2+\overline{y}^2 = x^2 + y^2
		\end{equation}
		The analog of the invariant hyperbolae, represented by $-t^2+x^2 = a^2$, is a circle of radius $a$, represented by the equation $x^2+y^2 = a^2$.\\ \\
		So we're dealing with non-Euclidean geometry, which requires adjustment of rules such as the triangle inequality and the Pythagorean Theorem from the Euclidean sense.		
	\end{subequations}
\end{chapter1}

\begin{chapter1}\label{prob:20}
The Lorentz transformations from equation (1.12) in the textbook can be represented as a matrix:
	\begin{equation}
		\begin{bmatrix}
	 		\bar{t} \\
			\bar{x} \\
			\bar{y} \\
			\bar{z}
			\end{bmatrix}
			=
			\begin{bmatrix}
		 		\gamma & -\gamma v & 0 & 0 \\
			 	-\gamma v & \gamma & 0 & 0 \\
				0 & 0 & 1 & 0 \\
			 	0 & 0 & 0 & 1
			\end{bmatrix}
	 		\begin{bmatrix}
	 		 t \\
	 		 x \\
	 		 y \\
	 		 z
	 		\end{bmatrix}
	 		, \hspace{5 pt} \gamma = \frac{1}{\sqrt{1-v^2}}
	\end{equation}
\end{chapter1}

\newtheorem{chapter2}{Problem}
\newcounter{subpart2}[chapter2]

\chapter{Vector Analysis in Special Relativity}

\begin{chapter2}\label{prob: 1}
	\stepcounter{subpart1}
	\begin{subequations}
		\begin{equation}
			(\alph{subpart1}) \hspace{10pt}
			A^{\alpha}B_{\alpha} = A^{0}B_{0} + A^{1}B_{1} + A^{2}B_{2} + A^{3}B_{3} = -4 
		\end{equation}
		\stepcounter{subpart1}
		\begin{equation}
			(\alph{subpart1}) \hspace{10pt}
			A^{\alpha}C_{\alpha 0} = 5 - 4 + 6 = 7, \\
			A^{\alpha}C_{\alpha 1} = 
	 		A^{\alpha}C_{\alpha 2} = 
	 		A^{\alpha}C_{\alpha 3}  
		\end{equation}
	\end{subequations} 	
\end{chapter2}

\begin{chapter2}\label{prob: 2}
	
\end{chapter2}

\begin{chapter2}\label{prob: 3}
	
\end{chapter2}

\begin{chapter2}\label{prob: 4}
	
\end{chapter2}

\begin{chapter2}\label{prob: 5}
	
\end{chapter2}

\begin{chapter2}\label{prob: 6}
	
\end{chapter2}

\begin{chapter2}\label{prob: 7}
	
\end{chapter2}

\begin{chapter2}\label{prob: 8}
	
\end{chapter2}

\begin{chapter2}\label{prob: 9}
	
\end{chapter2}

\begin{chapter2}\label{prob: 10}
	Let ${\vec A} = (1, 0, 0 , 0)$
\end{chapter2}

\begin{chapter2}\label{prob: 11}
	The Lorentz transformation matrix is represented by the symbol:
	\begin{subequations}
		\begin{equation}
			\Lambda^{\bar{\alpha}}_{\;\;\beta} = 
			\begin{bmatrix}
				\gamma & -\gamma v & 0 & 0 \\
				-\gamma v & \gamma & 0 & 0 \\
				0 & 0 & 1 & 0 \\
				0 & 0 & 0 & 1
			\end{bmatrix}\\
		\end{equation}
		\stepcounter{subpart2}
		(\alph{subpart2})
		% Symbolic Representation of Inverse
		The matrix of $\Lambda^{\nu}_{\;\;\bar{\mu}}$ can be obtained by inverting the signs of $v$ in $\Lambda^{\bar{\alpha}}_{\;\;\beta}$ or by finding the inverse of the Lorentz transformation matrix (using Gaussian elimination). Either way, the resultant matrix is:
		\begin{equation}
			\Lambda^{\nu}_{\;\;\bar{\mu}} = 
			\begin{bmatrix}
				\gamma & \gamma v & 0 & 0 \\
				\gamma v & \gamma & 0 & 0 \\
				0 & 0 & 1 & 0 \\
				0 & 0 & 0 & 1
			\end{bmatrix}\\
		\end{equation}
		\stepcounter{subpart2}
		(\alph{subpart2})
		% Finding the transformed components
		The components of a four-vector (or any vector, for that matter) transform according to the following law:
		\begin{equation}
			A^{\bar{\alpha}}=\Lambda^{\bar{\alpha}}_{\;\;\beta}A^{\beta} 
		\end{equation}
		Evaluating the components,
		
			\begin{equation}
				A^{\bar{0}}=
				\Lambda^{\bar{0}}_{\;\;0}A^{0} + \Lambda^{\bar{0}}_{\;\;1}A^{1} +
				\Lambda^{\bar{0}}_{\;\;2}A^{2} + \Lambda^{\bar{0}}_{\;\;3}A^{3}
				= \gamma(A^0-vA^1)		
			\end{equation}
			\begin{equation}
				A^{\bar{1}}=
				\Lambda^{\bar{1}}_{\;\;1}A^{1} + \Lambda^{\bar{1}}_{\;\;1}A^{1} +
				\Lambda^{\bar{1}}_{\;\;2}A^{2} + \Lambda^{\bar{1}}_{\;\;3}A^{3}
				= \gamma(A^1-vA^0) 
			\end{equation}
			\begin{equation}
				A^{\bar{2}} = A^2  
			\end{equation}
			\begin{equation}
				A^{\bar{3}}	= A^3
			\end{equation}
		Therefore, ${\vec A} \stackrel{\bar{O}}{\longrightarrow}\left(A^{\bar{0}},A^{\bar{1}},A^{\bar{2}},A^{\bar{3}}\right) = (\gamma(A^0-vA^1),\gamma(A^1-vA^0),A^2,A^3)$\\
		Thus we have applied a Lorentz transformation on a general four-vector.\\ \\
		\stepcounter{subpart2}
		(\alph{subpart2})
		% Proving the inverse multiplication leads to the identity
		Computation of the expression $\Lambda^{\nu}_{\;\;\bar{\beta}}(-{\vec v})\Lambda^{\bar{\beta}}_{\;\;\alpha}({\vec v}) = \delta^{\nu}_{\;\;\alpha}$ will prove that the matrices are indeed inverses of each other.
		
			\begin{equation}
				\Lambda^{0}_{\;\;\bar{0}}\Lambda^{\bar{0}}_{\;\;0} + \\
				\Lambda^{0}_{\;\;\bar{1}}\Lambda^{\bar{1}}_{\;\;0} + \\
				\Lambda^{0}_{\;\;\bar{2}}\Lambda^{\bar{2}}_{\;\;0} + \\
				\Lambda^{0}_{\;\;\bar{3}}\Lambda^{\bar{3}}_{\;\;0} = \delta^{0}_{\;\;0} \\
			\end{equation}
			\begin{equation}
				\gamma^2- \gamma^2v^2 = \gamma^2(1-v^2) = \frac{1}{1-v^2}(1-v^2) = 1	
			\end{equation}
			\begin{equation}
				\Lambda^{1}_{\;\;\bar{0}}\Lambda^{\bar{0}}_{\;\;0} + \\
				\Lambda^{1}_{\;\;\bar{1}}\Lambda^{\bar{1}}_{\;\;0} + \\
				\Lambda^{1}_{\;\;\bar{2}}\Lambda^{\bar{2}}_{\;\;0} + \\
				\Lambda^{1}_{\;\;\bar{3}}\Lambda^{\bar{3}}_{\;\;0} = \delta^{1}_{\;\;0} \\
			\end{equation}
			\begin{equation}
				-v\gamma^2 + v\gamma^2 = 0
			\end{equation}
		Further computations can be obtained from symmetry or are trivial, and the result follows.\\\\
		\stepcounter{subpart2}
		(\alph{subpart2})
		% Lorentz Inverse Matrix
		The inverse transformation matrix is given by:
		\begin{equation}
			\Lambda^{\nu}_{\;\;\bar{\mu}} = 
			\begin{bmatrix}
				\gamma & \gamma v & 0 & 0 \\
				\gamma v & \gamma & 0 & 0 \\
				0 & 0 & 1 & 0 \\
				0 & 0 & 0 & 1
			\end{bmatrix}\\
		\end{equation}
		This is the same result as part (a) because the inverse is merely changing the signs of the velocity in the matrix. Physically, the observer is seeing the object move in the opposite direction. \\\\
		\stepcounter{subpart2}
		(\alph{subpart2})
		% Inverse vector components
		The following transformation allows us to find $A^{\beta}$ from $A^{\bar{\alpha}}$:
		\begin{equation}
			A^{\beta} = \Lambda^{\beta}_{\;\;\bar{\alpha}}A^{\bar{\alpha}}
		\end{equation}
		The components can be easily found using the inverse matrix, and thus the vector \\${\vec A} \stackrel{O}{\longrightarrow}\left(A^{{0}},A^{{1}},A^{{2}},A^{{3}}\right) = (\gamma(A^{\bar{0}}+vA^{\bar{1}}),\gamma(A^{\bar{1}}+vA^{\bar{0}}),A^{\bar{2}},A^{\bar{3}})$\\
		This agrees with the results of the inverse Lorentz transformation. \\\\
		\stepcounter{subpart2}
		(\alph{subpart2})\\\\
		% Change of Indices Identity %% DO IT LATER
		\stepcounter{subpart2}
		(\alph{subpart2})
		% Basis vectors and components
		Since vectors can be represented as linear combinations of other vectors, ${\vec e}_{\alpha}$ can be written as (in a different reference frame moving in the opposite direction for investigative purposes in this example):
		\begin{equation}
			{\vec e}_{\alpha} = \Lambda^{\bar{\beta}}_{\;\;\alpha}{\vec e}_{{\bar{\beta}}}
		\end{equation}
		Following the same logic, ${\vec e}_{\bar{\beta}}$ can also be written as a linear combination of basis vectors:
		\begin{equation}
			{\vec e}_{\bar{\beta}} = \Lambda^{\nu}_{\;\;\bar{\beta}}{\vec e}_{\nu}
		\end{equation}
		Therefore, ${\vec e}_{\alpha}$ can be written as a linear combination of ${\vec e}_{\nu}$:
		\begin{equation}
			{\vec e}_{\alpha} = \Lambda^{\bar{\beta}}_{\;\;\alpha}\Lambda^{\nu}_{\;\;\bar{\beta}}{\vec e}_{\nu}
		\end{equation}
		Evaluating the dummy sum over $\bar{\beta}$ results in the following expression:
		\begin{equation}
			{\vec e}_{\alpha} = \Lambda^{\nu}_{\;\;\alpha}{\vec e}_{\nu}
		\end{equation}
		Considering that these basis vectors are in the same reference frame, they must be linearly independent by definition. The above expression expresses one basis vector of a reference frame as a linear combination of the other basis vectors in the same reference frame. The only way this is possible if the right-hand expression is the basis vector itself, so the components $\Lambda^{\nu}_{\;\;\alpha}$ must be the Kronecker delta: 
		\begin{equation}
			{\vec e}_{\alpha} = \delta^{\nu}_{\;\;\alpha}{\vec e}_{\nu}
		\end{equation}
	\end{subequations}
\end{chapter2}

\begin{chapter2}\label{prob: 12}
	\stepcounter{subpart2}
	(\alph{subpart2})
	% Component evaluation in multiple frames
	The vector ${\vec A} \stackrel{O}{\longrightarrow}(0,-2,3,5).$ The following equation describes the Lorentz transformation of the components of the second frame with respect to the components of the first frame:
	\begin{subequations}
		\begin{equation}
			A^{\bar{\alpha}}=\Lambda^{\bar{\alpha}}_{\;\;\beta}A^{\beta} 
		\end{equation}
		Using ${\vec A} \stackrel{\bar{O}}{\longrightarrow}\left(A^{\bar{0}},A^{\bar{1}},A^{\bar{2}},A^{\bar{3}}\right) = (\gamma_{1}(A^0-vA^1),\gamma_{1}(A^1-vA^0),A^2,A^3)$, we get:
		\begin{equation}
			{\vec A} \stackrel{\bar{O}}{\longrightarrow}(\gamma_{1}(0-0.8\times-2),\gamma_{1}(-2-0.8\times0),3,5)
		\end{equation}
		The Lorentz factor is:
		\begin{equation}
			\gamma_{1} = \frac{1}{\sqrt{1-0.8^2}} = 1.667 
		\end{equation}
		Therefore,
		\begin{equation}
			{\vec A} \stackrel{\bar{O}}{\longrightarrow} (2.667,-3.333,3,5)
		\end{equation}
	\stepcounter{subpart2}
	(\alph{subpart2})
	% Components in second frame
	The components of the vector in frame 3 are:
	\begin{equation}
			A^{\bar{\bar{\mu}}}=\Lambda^{\bar{\bar{\mu}}}_{\;\;\bar{\alpha}}A^{\bar{\alpha}} 
	\end{equation}
	Using the previous expression for the components of $A^{\bar{\alpha}}$,
	\begin{equation}			A^{\bar{\bar{\mu}}}=\Lambda^{\bar{\bar{\mu}}}_{\;\;\bar{\alpha}}\Lambda^{\bar{\alpha}}_{\;\;\beta}A^{\beta}=\Lambda^{\bar{\bar{\mu}}}_{\;\;\beta}A^{\beta}
	\end{equation}
	We can represent this transformation as a matrix multiplication:
	\begin{equation}
		\begin{bmatrix}
	 		A^{\bar{\bar{0}}} \\
			A^{\bar{\bar{1}}} \\
			A^{\bar{\bar{2}}} \\
			A^{\bar{\bar{3}}}
		\end{bmatrix}
			=
		\begin{bmatrix}
		 	\gamma_{2} & -\gamma_{2} v_{2} & 0 & 0 \\
			-\gamma_{2} v_{2} & \gamma_{2} & 0 & 0 \\
			0 & 0 & 1 & 0 \\
			0 & 0 & 0 & 1
		\end{bmatrix}
		\begin{bmatrix}
		 	\gamma_{1} & -\gamma_{1} v_{1} & 0 & 0 \\
			-\gamma_{1} v_{1} & \gamma_{1} & 0 & 0 \\
			0 & 0 & 1 & 0 \\
			0 & 0 & 0 & 1
		\end{bmatrix}
	 	\begin{bmatrix}
	 		A^{0} \\
			A^{1} \\
			A^{2} \\
			A^{3}
	 	\end{bmatrix}
	\end{equation}
	with $v_{1} = 0.8$ and $v_{2} = 0.6$. The second Lorentz factor is:
	\begin{equation}
			\gamma_{2} = \frac{1}{\sqrt{1-0.6^2}} = 1.25 
	\end{equation}
	Simplifying the matrix, we can compute the matrix $\Lambda^{\bar{\bar{\mu}}}_{\;\;\beta}$ that outlines the stacked Lorentz transformations from frame 1 to frame 3 directly:
	\begin{equation}
		\begin{bmatrix}
	 		A^{\bar{\bar{0}}} \\
			A^{\bar{\bar{1}}} \\
			A^{\bar{\bar{2}}} \\
			A^{\bar{\bar{3}}}
		\end{bmatrix}
			=
		\begin{bmatrix}
		 	\gamma_{2}\gamma_{1}(1+v_{1}v_{2}) & -\gamma_{2}\gamma_{1}(v_{1}+v_{2}) & 0 & 0 \\
			-\gamma_{2}\gamma_{1}(v_{1}+v_{2}) & \gamma_{2}\gamma_{1}(1+v_{1}v_{2}) & 0 & 0 \\
			0 & 0 & 1 & 0 \\
			0 & 0 & 0 & 1
		\end{bmatrix}
	 	\begin{bmatrix}
	 		A^{0} \\
			A^{1} \\
			A^{2} \\
			A^{3}
	 	\end{bmatrix}
	\end{equation}
	Solving the system of equations, ${\vec A} \stackrel{\bar{\bar{O}}}{\longrightarrow} (5.835,-6.168,3,5)$\\\\
	\stepcounter{subpart2}
	(\alph{subpart2})
	The magnitude of ${\vec A}$ from its components in $O$ is:
	\begin{equation}
		|{\vec A}| = -(A^{0})^2 + (A^{1})^2 +(A^{2})^2 +(A^{3})^2 = 38
	\end{equation}
	\stepcounter{subpart2}
	(\alph{subpart2})
	The magnitude of ${\vec A}$ from its components in $\bar{O}$ is:
	\begin{equation}
		|{\vec A}| = -(A^{\bar{0}})^2 + (A^{\bar{1}})^2 +(A^{\bar{2}})^2 +(A^{\bar{3}})^2 = 38
	\end{equation}
	Hence the magnitude of the vector, which can be interpreted as the interval between two space-time points, is invariant under a coordinate transformation. Note that the interval is spacelike.
	\end{subequations}
\end{chapter2}

\begin{chapter2}\label{prob: 13}
	\stepcounter{subpart2}
	(\alph{subpart2})
	\begin{subequations}
	The Lorentz transformation from $O$ to $\overline{O}$ is given by:
	\begin{equation}
		A^{\bar{\gamma}}=\Lambda^{\bar{\gamma}}_{\;\;\mu}({\bf v})A^{\mu} 
	\end{equation}
	The Lorentz transformation from $\overline O$ to $\overline{\overline{O}}$ is given by:
	\begin{equation}
		A^{\bar{\bar{\alpha}}} = \Lambda^{\bar{\bar{\alpha}}}_{\;\;\bar{\gamma}}({\bf v'})A^{\bar{\gamma}} 
	\end{equation}
	Since $A^{\bar{\gamma}}$ can be expanded as the Lorentz transformation with respect to frame $O$,
	\begin{equation}
		A^{\bar{\bar{\alpha}}} = \Lambda^{\bar{\bar{\alpha}}}_{\;\;\bar{\gamma}}({\bf v'})\Lambda^{\bar{\gamma}}_{\;\;\mu}({\bf v})A^{\mu}
	\end{equation}
	\stepcounter{subpart2}
	(\alph{subpart2})\\\\ %% do b) later
	\stepcounter{subpart2}
	(\alph{subpart2})
	Since the expressions can be represented as matrix multiplication,	
	\begin{equation}
		\Lambda^{\bar{\bar{\alpha}}}_{\;\;\mu} = 	
		\begin{bmatrix}
			\gamma_{2} & 0 & -\gamma_{2} v_{2} & 0 \\
			 0 & 1 & 0 & 0 \\
			-\gamma_{2} v_{2} & 0 & \gamma_{2} & 0 \\
			0 & 0 & 0 & 1
		\end{bmatrix}
		\begin{bmatrix}
			\gamma_{1} & -\gamma_{1} v_{1} & 0 & 0 \\
			-\gamma_{1} v_{1} & \gamma_{1} & 0 & 0 \\
			0 & 0 & 1 & 0 \\
			0 & 0 & 0 & 1
		\end{bmatrix}
	\end{equation}
	Where $v_{1}$ corresponds to ${\bf v} = 0.6{\vec e_{x}}, v_{2}$ corresponds to ${\bf v'} = 0.8{\vec e_{y}}$ and $\gamma_{1} = 1.25, \gamma_{2} = 1.667.$\\
	Evaluating this matrix:
	\begin{equation}
		\Lambda^{\bar{\bar{\alpha}}}_{\;\;\mu} = 	
		\begin{bmatrix}
			\gamma_{2}\gamma_{1} & -\gamma_{2}\gamma_{1}v_{1} & -\gamma_{2}v_{2} & 0 \\
			-\gamma_{1}v_{1} & 1 & 0 & 0 \\
			-\gamma_{2}\gamma_{1} v_{2} & 0 & \gamma_{2}\gamma_{1} & 0 \\
			0 & 0 & 0 & 1
		\end{bmatrix}
		=
		\begin{bmatrix}
			2.084 & -1.250 & -1.334 & 0 \\
			-0.750 & 1 & 0 & 0 \\
			-1.667 & 0 & 2.084 & 0 \\
			0 & 0 & 0 & 1
		\end{bmatrix}
	\end{equation}
	\stepcounter{subpart2}
	(\alph{subpart2})\\\\%% do d) later
	\stepcounter{subpart2}
	(\alph{subpart2})
	Evaluating the expression as a matrix multiplication:
	\begin{equation}
		\begin{split}
		\Lambda^{\bar{\bar{\alpha}}}_{\;\;\beta} = \Lambda^{\bar{\bar{\alpha}}}_{\;\;\bar{\gamma}}({\bf v})\Lambda^{\bar{\gamma}}_{\;\;\beta}({\bf v'}) =
		\begin{bmatrix}
			\gamma_{1} & -\gamma_{1} v_{1} & 0 & 0 \\
			-\gamma_{1} v_{1} & \gamma_{1} & 0 & 0 \\
			0 & 0 & 1 & 0 \\
			0 & 0 & 0 & 1
		\end{bmatrix}
		\begin{bmatrix}
			\gamma_{2} & 0 & -\gamma_{2} v_{2} & 0 \\
			 0 & 1 & 0 & 0 \\
			-\gamma_{2} v_{2} & 0 & \gamma_{2} & 0 \\
			0 & 0 & 0 & 1
		\end{bmatrix}\\\\
		\Lambda^{\bar{\bar{\alpha}}}_{\;\;\beta} = 	
		\begin{bmatrix}
			\gamma_{1}\gamma_{2} & -\gamma_{1}v_{1} & -\gamma_{1}\gamma_{2} v_{2} & 0 \\
			-\gamma_{1}\gamma_{2}v_{1} & 1 & 0 & 0 \\
			-\gamma_{2}v_{2} & 0 & \gamma_{1}\gamma_{2} & 0 \\
			0 & 0 & 0 & 1
		\end{bmatrix}
		=
		\begin{bmatrix}
			2.084 & -0.750 & -1.667 & 0 \\
			-1.250 & 1 & 0 & 0 \\
			-1.334 & 0 & 2.084 & 0 \\
			0 & 0 & 0 & 1
		\end{bmatrix}
		\end{split}
	\end{equation}
	So the two expressions are not equal, as expected from the non-commutation of matrix multiplication. An interesting point, notice how $\Lambda^{\bar{\bar{\alpha}}}_{\;\;\beta}$ is the transpose of $\Lambda^{\bar{\bar{\alpha}}}_{\;\;\mu}.$ %% Write and draw physical interpretation here.
	\end{subequations}
\end{chapter2}

\begin{chapter2}\label{prob: 14}
	\begin{subequations}
	\stepcounter{subpart2}
	(\alph{subpart2})
	1. The direction of $\overline{O}$ is in the negative z-direction, as can be deduced from the previous exercises on Lorentz transformations.\\
	2. Since $\gamma = 1.25$ and $-\gamma v = 0.75 \Rightarrow v = -0.75/1.25 = -0.6$, the direction remaining consistent with the previous observation.\\\\
	\stepcounter{subpart2}
	(\alph{subpart2})
	While Gaussian elimination will easily provide the inverse matrix, a little physical reasoning immediately leads to the deduction that the inverse is:
	\begin{equation}
		\begin{bmatrix}
			1.25 & 0 & 0 & -0.75 \\
			0 & 1 & 0 & 0 \\
			0 & 0 & 1 & 0 \\
			-0.75 & 0 & 0 & 1.25
		\end{bmatrix}
	\end{equation}
	\stepcounter{subpart2}
	(\alph{subpart2})
	A convention is introduced here in which the vector will be represented by its component form $A^{\alpha}$. The components in frame $O$ can be found by the following expression:
	\begin{equation}
		A^{\alpha}=\Lambda^{\alpha}_{\;\;\bar{\beta}}A^{\bar{\beta}} =
		\begin{bmatrix}
			1.25 & 0 & 0 & -0.75 \\
			0 & 1 & 0 & 0 \\
			0 & 0 & 1 & 0 \\
			-0.75 & 0 & 0 & 1.25
		\end{bmatrix}
		\begin{bmatrix}
	 		A^{\bar{0}} \\
			A^{\bar{1}} \\
			A^{\bar{2}} \\
			A^{\bar{3}}
		\end{bmatrix}
		=
		\begin{bmatrix}
			1.25 & 0 & 0 & -0.75 \\
			0 & 1 & 0 & 0 \\
			0 & 0 & 1 & 0 \\
			-0.75 & 0 & 0 & 1.25
		\end{bmatrix}
		\begin{bmatrix}
	 		1 \\
			2 \\
			0 \\
			0
		\end{bmatrix}	
	\end{equation}
	Where $\Lambda^{\alpha}_{\;\;\bar{\beta}}$ is the inverse transformation matrix above. Therefore, by performing the matrix multiplication, ${\vec A} \stackrel{O}{\longrightarrow}(A^{0},A^{1},A^{2},A^{3}) = (1.25, 2, 0, -0.75).$ Notice how the time component dilates to a larger value and the length in the z-direction (the direction of motion) contracts to a smaller (negative) value, thus displaying the relativistic effects.
	\end{subequations}
\end{chapter2}

\begin{chapter2}\label{prob: 15}
	\begin{subequations}
	\stepcounter{subpart2}
	(\alph{subpart2})
	The velocity of the $O$ frame is in the negative x-direction with respect to the MCRF of the particle. Expressing the velocity of the particle in its rest frame as a four-vector ${\vec V} \stackrel{MCRF}{\longrightarrow}(1,0,0,0)$, we can apply the Lorentz transformation:
	\begin{equation}
		V^{\alpha}=\Lambda^{\alpha}_{\;\;\bar{\beta}}V^{\bar{\beta}} =
		\begin{bmatrix}
			\gamma & \gamma v & 0 & 0 \\
			\gamma v & \gamma & 0 & 0 \\
			0 & 0 & 1 & 0 \\
			0 & 0 & 0 & 1
		\end{bmatrix}
		\begin{bmatrix}
	 		1 \\
			0 \\
			0 \\
			0
		\end{bmatrix}
		=
		\begin{bmatrix}
	 		\gamma \\
			\gamma v \\
			0 \\
			0
		\end{bmatrix}	
	\end{equation}
	Therefore, ${\vec V} \stackrel{O}{\longrightarrow}(\gamma,\gamma v,0,0)$ \\\\
	\stepcounter{subpart2}
	(\alph{subpart2})
	Observation suggests that an arbitrary three-velocity ${\bf v} =  v^x {\bf e}_x + v^y {\bf e}_y + v^z {\bf e}_z$ will result in the vector ${\vec v} \stackrel{O}{\longrightarrow}(\gamma,\gamma v^{x},\gamma v^{y},\gamma v^{z})$, where $\gamma = \frac{1}{\sqrt{1-|{\bf v}|^2}}$ after a Lorentz transformation is applied on it from its MCRF. This is justified by the observation that the only non-zero component of the four-velocity vector in the MCRF is the time component. \\\\
	\stepcounter{subpart2}
	(\alph{subpart2})
	$ \vec v = (U^{0}, U^{1}, U^{2}, U^{3})$ \\\\
	\stepcounter{subpart2}
	(\alph{subpart2})
	The three-velocity can be obtained by dividing the spacial components of the four-velocity with the time component, therefore ${\bf v} = (0.5,0.5,0.5)$
	\end{subequations} 

\end{chapter2}

\begin{chapter2}\label{prob: 16}
	\begin{subequations}
		\begin{equation}
			U^{\bar t} = \frac{U^{t} + WU^{x}}{\sqrt{1-W^{2}}} 
		\end{equation}
		\begin{equation}
			U^{\bar x} = \frac{U^{x} + WU^{t}}{\sqrt{1-W^{2}}}
		\end{equation}
		\begin{equation}
			\frac{U^{\bar x}}{U^{\bar t}} = \frac{U^{x} + WU^{t}}{U^{t} + WU^{x}} = \frac{v + W}{1 + vW}
		\end{equation}
	\end{subequations}
\end{chapter2}

\begin{chapter2}\label{prob: 17}

\end{chapter2}

\begin{chapter2}\label{prob: 18}
	
\end{chapter2}

\begin{chapter2}\label{prob: 19}
	\begin{subequations}
		\stepcounter{subpart2}
		(\alph{subpart2})
		\stepcounter{subpart2}
		(\alph{subpart2})
		Assuming the body is uniformly accelerated along the x-axis, the velocity and acceleration in the observer's MCRF from $\vec v \cdot \vec a = 0$ are given by:
		\begin{equation}
			\vec v \stackrel{MCRF}{\longrightarrow}(1, 0, 0, 0) \hspace{20 pt} \vec a \stackrel{MCRF}{\longrightarrow}(0, \alpha, 0, 0)
		\end{equation}
		Measuring the velocity in the observer $\bar O$'s frame of reference:
		\begin{equation}
			\vec v \stackrel{\bar O}{\longrightarrow}(\gamma, \gamma v, 0, 0)
		\end{equation}
		The acceleration in this reference frame is given by:
		\begin{equation}
			\vec a \stackrel{\bar O}{\longrightarrow}(\gamma v \alpha, \gamma \alpha, 0, 0) 
		\end{equation}
		To find the acceleration in terms of the velocity derivative, we must differentiate the velocity with respect to the proper time in this frame:
		\begin{equation}
			\vec a = \dv{\vec v}{\tau} \stackrel{\bar O}{\longrightarrow}
			\dv{\tau}(\gamma, \gamma v, 0, 0) = 
			\dv{v}{\tau}(v\gamma^{3}, v^{2}\gamma^{3} + \gamma, 0, 0) =
			\gamma^2\dv{v}{\tau}(v, 1, 0, 0)
		\end{equation}
		But $\dd{\tau} = \dd{t}/\gamma$, giving:
		\begin{equation}
			\vec a \stackrel{\bar O}{\longrightarrow}\gamma^{2}\dv{v}{t}(v\gamma^{2}, v^{2}\gamma^{2} + 1, 0, 0) =
			\gamma^{4}\dv{v}{t}(v, 1, 0, 0)
		\end{equation}
		Equating the components, we get:
		\begin{equation}
			\alpha = \gamma^{3}\dv{v}{t}
		\end{equation}
		Integrating the differential equation:
		\begin{equation}
			\int^{t}_{0} \alpha \dd{t} = \int^v_0 \frac{1}{(1-v^{2})^{\frac{3}{2}}} \dd{v} 
		\end{equation}
		This can be solved by making the substitution (without adjusting the limits) $v = \sin \theta \rightarrow \dd{v} = \cos\theta\dd{\theta}$:
		\begin{equation}
			\alpha t = \int^v_0 \sec^{2}\theta\dd{\theta} = \eval[\tan\theta|^v_0
		\end{equation}
		The initial velocity is zero. The substitution can be reversed by using the following right triangle: \\
		\begin{center}
			\begin{tikzpicture}[scale=.8]
				\tkzInit[xmax=5,ymax=3] %\tkzClip[space=.5]
				\tkzDefPoint(0,0){A} \tkzDefPoint(4,0){B}
				\tkzDrawTriangle[pythagore](A,B)
				\tkzGetPoint{C}
				\tkzLabelSegment[below,font=\footnotesize](A,B){$\sqrt{1 - v^2}$}
				\tkzLabelSegment[above,font=\footnotesize](A,C){$1$}
				\tkzLabelSegment[right,font=\footnotesize](B,C){$v$}
				\tkzMarkAngle[fill= blue!20,size=1.4cm,opacity=.5](B,A,C)
				\tkzLabelAngle[pos=0.8](B,A,C){$\theta$}
			\end{tikzpicture}
		\end{center}
		This gives the expression:
		\begin{equation}
			\alpha t = \frac{v}{\sqrt{1-v^2}} \bigg|^v_0 =  \frac{v}{\sqrt{1-v^2}}
		\end{equation}
		Solving for $v$, which is the speed at time $t$:
		\begin{equation}
			v = \frac{\alpha t}{\sqrt{1 + \alpha^2 t^2}}
		\end{equation}
		The distance is given by:
		\begin{equation}
			\dv{s}{t} = \frac{\alpha t}{\sqrt{1 + \alpha^2 t^2}}
		\end{equation}
		\begin{equation}
			\int^x_0 \;\dd{s} = \int^t_0 \frac{\alpha t}{\sqrt{1 + \alpha^2 t^2}}\dd{t}
		\end{equation}
		Using the substitution $\alpha t = \tan\theta \longrightarrow \alpha\dd{t} = \sec^2 \theta\dd{\theta}$
		\begin{equation}
			x = \frac{1}{\alpha}\int^t_0 \sec\theta\tan\theta\dd{\theta} = \sec\theta\big|^t_0 = 
			\frac{\sec\arctan(\alpha t)\big|^t_0}{\alpha}
		\end{equation}
		The expression can be evaluated using a different right triangle:
		\begin{center}
			\begin{tikzpicture}[scale=.8]
				\tkzInit[xmax=5,ymax=3] %\tkzClip[space=.5]
				\tkzDefPoint(0,0){A} \tkzDefPoint(4,0){B}
				\tkzDrawTriangle[pythagore](A,B)
				\tkzGetPoint{C}
				\tkzLabelSegment[below,font=\footnotesize](A,B){$1$}
				\tkzLabelSegment[above,font=\footnotesize, rotate = 36.86](A,C){$\sqrt{1+\alpha^2 t^2}$}
				\tkzLabelSegment[right,font=\footnotesize](B,C){$\alpha t$}
				\tkzMarkAngle[fill= blue!20,size=1.4cm,opacity=.5](B,A,C)
				\tkzLabelAngle[pos=0.8](B,A,C){$\theta$}
			\end{tikzpicture}
		\end{center}
		\begin{equation}
			x = \frac{1}{\alpha}\left(\sqrt{1+\alpha^2 t^2} - 1\right)
		\end{equation}
		The time required to reach $v = 0.999$ is, from equation 2.8i:
		\begin{equation}
			t = \frac{v}{\alpha \sqrt{1-v^2}} = \frac{0.999}{10\sqrt{1-0.999^2}}\times3\times10^8 = 6.703\times10^8 s 
		\end{equation}
		\stepcounter{subpart2}
		(\alph{subpart2})
		Following the hint, we must perform the integration along the body's world line using the acceleration from equation 2.8d:
		\begin{equation}
			\int^{\tau}_0\alpha\dd{\tau} =  \int^v_0 \gamma^2\dd{v} = \int^v_0 \frac{\dd{v}}{1-v^2}
		\end{equation}
		While this integral is easily solvable via partial fractions, there is a hyperbolic solution we should follow in the spirit of special relativity:
		\begin{equation}
			\tau = \frac{1}{\alpha}\eval[\arctanh(v)|^v_0 = \frac{\arctanh(v)}{\alpha} = \frac{1}{\alpha}\arctanh \left(\frac{\alpha t}{\sqrt{1 + \alpha^2 t^2}}\right)
		\end{equation}
		Note that we obtain the well-known rapidity parametrisation with $v = \tanh (\alpha \tau)$. \\
		The proper time elapsed by the time the speed $v = 0.999$ is $$\tau = \frac{\left(3 \times 10^8\right)}{10}\arctanh(0.999) = 1.14 \times 10^8$$ seconds, which is about 3.6 years. \\\\
		% Add final answer later
	\end{subequations}
\end{chapter2}

\begin{chapter2}\label{prob: 20}
	
\end{chapter2}

\begin{chapter2}\label{prob: 21}
	Calculating the interval described by the parametric equations:
	\begin{subequations}
		\begin{equation}
			\dd{s^{2}} = - \dd{t^{2}} + \dd{x^{2}} = \dd{\lambda^2}\left[-\cosh^2\left(\frac{\lambda}{a}\right) + \sinh^2\left(	\frac{\lambda}{a}\right)\right] = -\dd{\lambda^{2}} = -\dd{\tau^{2}}
		\end{equation}
		Which indicates that $\lambda$ is indeed the proper time $\tau$. \\
		The four-velocity is given by $\vec U = \dd{\vec x}/\dd{\tau}$.
		\begin{equation}
			\vec U = \dv{\vec x}{\tau} = \dv{\vec x}{\lambda} \stackrel{O} \longrightarrow
			\left(\cosh\left(\frac{\lambda}{a}\right), \sinh\left(\frac{\lambda}{a}\right), 0, 0 \right) 
		\end{equation}
		The four-acceleration is given by $\vec a = \dd[2]{\vec x}/\dd{\tau}^2$
		\begin{gather}
			\vec a = \dv[2]{\vec x}{\tau} = \dv[2]{\vec x}{\lambda} \stackrel{O} \longrightarrow \dv{\lambda}\left(\cosh\left(\frac{\lambda}{a}\right), \sinh\left(\frac{\lambda}{a}\right), 0, 0 \right) \\ = \frac{1}{a} \left(\sinh\left(\frac{\lambda}{a}\right), \cosh\left(\frac{\lambda}{a}\right), 0, 0 \right) \frac{1}{a^2}\left(t(\lambda), x(\lambda), 0, 0\right)
		\end{gather}
	\end{subequations}
\end{chapter2}

\begin{chapter2}\label{prob: 22}
	\stepcounter{subpart2}
	(\alph{subpart2})
	The four-momentum is given by $\vec p \stackrel{O} \longrightarrow (E, p^1, p^2, p^3)$. So the energy is 4 kg. \\
	The rest mass can be found using:
	\begin{equation}
		E^2 = m^2 + \sum^3_{i=3} (p^i)^2 \longrightarrow m = \sqrt{E^2 - \sum^3_{i=3}(p^i)^2} = \sqrt{4^2 - 1^2 - 1^2} = 3.74 \;\mathrm{kg} 
	\end{equation}
	The three-velocity $\bf v$ can be found from dividing the momentum components by the energy components using the definition: $\vec p = m\vec U$, giving:
	\begin{equation}
		{\bf v} = 0.25{\bf i} + 0.25{\bf j}
	\end{equation}
	\stepcounter{subpart2}
	(\alph{subpart2})
	Following the conservation of four-momentum, we get $\vec p_5 \stackrel{O}\rightarrow (3, -0.5, 1, 0)$. Its energy is 3 kg, its rest mass is approximately 2.78 kg, its three-velocity is: ${\bf v} = -1/6{\bf i} + 1/3{\bf j}$.\\\\
	The CM frame is defined as the one in which the momentum components of the four-momentum are zero and the energy component is the total energy of the system. This can be found by summing the components of the particles before or after the collision giving $\vec p_M \rightarrow (5, 0, 1, 0)$.\\\\
	This means that the three-velocity of the CM frame must be ${\bf v}_2 = 0.2{\bf j}$.
\end{chapter2}

\begin{chapter2}\label{prob: 23}
	The energy is given by:
	\begin{equation}
		E = \frac{m}{\sqrt{1-|{\bf v}|^2}}
	\end{equation}
	Where $m$ is the rest mass (which represents the rest energy $E_0 = mc^2$). \\
	Using the series expansion for: $$\frac{1}{\sqrt{1 + x}} = 1-\frac{x}{2} + \frac{3x^2}{8} - \frac{5x^3}{16} + ... \;,\; |x| < 1 $$
	\begin{equation}
		E = m\left[1 + \frac{1}{2}|{\bf v}|^2 + \frac{3}{8}|{\bf v}|^4 + \order{\abs{\bf v}^6} \right]
	\end{equation}
	To find the $|\bf v|$ value for which the fourth order term is equal to half of the Newtonian kinetic energy, we must solve:
	\begin{equation}
		\frac{1}{4}m|{\bf v}|^2 = \frac{3}{8}m|{\bf v}|^4
	\end{equation}
	Which gives $|{\bf v}| = \sqrt{2/3} \approx 0.816$
\end{chapter2}

\begin{chapter2}\label{prob: 24}
	Take an electron with three-velocity ${\bf v}_1 = 0.5{\bf e}_x + 0.5{\bf e}_y$ and a positron with three-velocity ${\bf v}_2 = 0.5{\bf e}_x - 0.5{\bf e}_y$ in the reference frame $O$. To find the four-momentum of each particle, we must first find their four-velocities:
	\begin{equation}
		\gamma_1 = \gamma_2 = \frac{1}{\sqrt{1 - |{\bf v}_1|^2}} = \frac{1}{\sqrt{1 - 0.707^2}} \approx 1.414 
	\end{equation}
	This gives:
	\begin{equation}
		\vec U_1 \stackrel{O}\longrightarrow (1.414, 0.707, 0.707, 0), \hspace{10 pt} \vec U_2 = \stackrel{O}\longrightarrow (1.414, 0.707, -0.707, 0)
	\end{equation}
	The masses of an electron and positron are the same ($m$ here). So the four-momentum of each particle is:
	\begin{equation}
		\vec p_1 = m \vec U_1, \hspace{10 pt} \vec p_2 = m \vec U_2 		
	\end{equation} 
	The total four-momentum of the system is:
	\begin{equation}
		\vec p_1 + \vec p_2 \stackrel{O}\longrightarrow m(2.828, 1.414, 0, 0)
	\end{equation}
	The CM frame is then:
	\begin{equation}
		{\bf v} = 0.5{\bf e}_x
	\end{equation}
	The four-momentum in the CM frame is:
	\begin{equation}
		\vec p_1 + \vec p_2 \stackrel{CM}\longrightarrow (2.828, 0, 0, 0)
	\end{equation}
	Now, this frame observes a collision of particles in the y-axis and an emission of a photon in the x-axis, which has a momentum. But the conservation of momentum dictates that this is impossible; however, emission of two photons at equal angles along the line of collision will conserve momentum.  
\end{chapter2}

\begin{chapter2}\label{prob: 25}
	\begin{subequations}
		\stepcounter{subpart2}
		(\alph{subpart2})
		The four-momentum of the photon in $O$'s frame is:
		\begin{equation}
			\vec p \stackrel{O}\longrightarrow (p^0, p^1, p^2, p^3) = (E, E\cos\theta, E\sin\theta, 0) =(h\nu, h\nu\cos\theta, h\nu\sin\theta, 0)
		\end{equation}
		Applying the Lorentz transformation to find the energy component $p^{\bar 0}$ in $\bar O$'s frame:
		\begin{equation}
			p^{\bar 0} = h{\bar \nu} = \gamma(E - Ev\cos\theta) = \gamma(h\nu - h\nu v\cos\theta)
		\end{equation}
		\begin{equation}
			\frac{\bar \nu}{\nu} = \frac{1 - v\cos\theta}{\sqrt{1-v^2}}
		\end{equation}
		\stepcounter{subpart2}
		(\alph{subpart2})
		For no Doppler shift, we must have the condition ${\bar \nu}/{\nu} = 1$. Using this condition in the previous derivation:
		\begin{equation}
			\sqrt{1-v^2} = 1 - v\cos\theta
		\end{equation}
		This gives the quadratic equation
		\begin{equation}
			\cos^2\theta -\frac{2}{v}\cos\theta + 1 = 0
		\end{equation}
		Solving the equation (keeping in mind that $|\cos\theta| < 1 $):
		\begin{equation}
			\cos\theta = \frac{1 - \sqrt{1-v^2}}{v} \longrightarrow \theta = \arccos \left(\frac{1 - \sqrt{1-v^2}}{v} \right)
		\end{equation}
		\stepcounter{subpart2}
		(\alph{subpart2})
		Eq. 2.35: $-\vec p \cdot \vec U_{obs} = -\vec p \cdot \vec e_{\bar 0} = \bar E$, Eq. 2.38: $E = h\nu$. The observer frame's velocity is:
		\begin{equation}
			\vec U_{obs} \stackrel{O}\longrightarrow (\gamma, -\gamma v, 0, 0)
		\end{equation}
		So, now we just have to take the dot product:
		\begin{equation}
			\bar E = h{\bar \nu} = -\vec p \cdot \vec U_{obs} = -\left(-\gamma p^0 + \gamma vp^1\right) = \frac{h\nu - h\nu v \cos\theta}{\sqrt{1-v^2}}
		\end{equation}
		Which (more neatly) gives the same result:
		\begin{equation}
			\frac{\bar \nu}{\nu} = \frac{1 - v\cos\theta}{\sqrt{1-v^2}}
	\end{equation}
	\end{subequations}
\end{chapter2}

\begin{chapter2}\label{prob: 26}
	
\end{chapter2}

\begin{chapter2}\label{prob: 27}
	
\end{chapter2}

\begin{chapter2}\label{prob: 28}
	\begin{subequations}
		\stepcounter{subpart2}
		(\alph{subpart2})
		Applying the Lorentz transformation:
		\begin{equation}
			\vec A \stackrel{\bar O}\longrightarrow (5.5, -2.5, -1, 0), \;\; 
			\vec B \stackrel{\bar O}\longrightarrow (-4.75, 5.25, 1, 6), \;\; 
			\vec C \stackrel{\bar O}\longrightarrow (4, -4, 0, 0)
		\end{equation}
		\stepcounter{subpart2}
		(\alph{subpart2})
		\begin{equation}
			\vec A \cdot \vec B \stackrel{\bar O} = -(-4.75 \times 5.5) + (-2.5 \times 5.25) + (-1 \times 1) + (0 \times 6) = 12
		\end{equation}
		\begin{equation}
			\vec B \cdot \vec C \stackrel{\bar O} = -(-4.75 \times 4) + (-5.25 \times -4) + (1 \times 0) + (6 \times 0) = 40
		\end{equation}
		\begin{equation}
			\vec A \cdot \vec C \stackrel{\bar O} = -(5.5 \times 4) + (-2.5 \times -4) + (-1 \times 0) + (0 \times 0) = -12
		\end{equation}
		\begin{equation}
			\vec C \cdot \vec C \stackrel{\bar O} = -(4)^2 + (-4)^2 = 0
		\end{equation}
		The dot products in the original frame $O$ are as follows:
		$$ \vec A \cdot \vec B \stackrel{O} = -(5 \times -2) + (1 \times 3) + (-1 \times 1) + (0 \times 6) = 12 $$
		$$ \vec B \cdot \vec C \stackrel{O} = -(-2 \times 2) + (3 \times -2) + (1 \times 0) + (6 \times 0) = 40 $$
		$$ \vec A \cdot \vec C \stackrel{O} = -(5 \times 2) + (1 \times -2) + (-1 \times 0) + (0 \times 0) = -12 $$
		$$ \vec C \cdot \vec C \stackrel{O} = -(2)^2 + (-2)^2 = 0 $$
		Clearly the dot products are frame-invariant. \\\\
		\stepcounter{subpart2}
		(\alph{subpart2})
		$ \vec A^2 \stackrel{O} = -(5)^2 + (1)^2 + (-1)^2 = -23 $, so $\vec A$ is timelike.\\
		$ \vec B^2 \stackrel{O} = -(-2)^2 + (3)^2 + (1)^2 + (6)^2 = 42 $, so $\vec B$ is spacelike.\\
		$ \vec C^2 \stackrel{O} = -(2)^2 + (-2)^2 = 0 $, so $\vec C$ is lightlike.
	\end{subequations} 
\end{chapter2}

\begin{chapter2}\label{prob: 29}
	Using the relevant equations, we expand both sides of:
	$$ \dv{\tau}(\vec U \cdot \vec U) = 2\vec U \cdot \dv{\vec U}{\tau} $$
	\begin{equation}
		\dv{\tau}\left[-(U^0)^2 + (U^1)^2 + (U^2)^2 + (U^3)^2\right] = -2U^0\dv{U^0}{\tau} + 2U^1\dv{U^1}{\tau} + 2U^2\dv{U^2}{\tau} + 2U^3\dv{U^3}{\tau}
	\end{equation}
	Differentiating the components on the left side using the chain rule gives the right side.
\end{chapter2}

\begin{chapter2}\label{prob: 30}
	\begin{subequations}
		\stepcounter{subpart2}
		(\alph{subpart2})
		To solve this problem, we must figure out the general Lorentz transformation for four dimensions. A physical property we must realise about the transformations is that only the parallel component of the spatial vector is contracted and its perpendicular component is left untouched. We can describe the vectorial Lorentz transformations with $\gamma = 1/\sqrt {1-|{\bf v}|^2}$ as follows:
		\begin{equation}
			\bar t = \gamma (t - {\bf r}_{\parallel}\cdot {\bf v}) 	
		\end{equation} 
		\begin{equation}
			{\bf {\bar r}}_{\parallel} = \gamma ({\bf r}_{\parallel} - {\bf v} t) 
		\end{equation}
		\begin{equation}
			{\bf {\bar r}}_{\perp} = {\bf r}_{\perp}
		\end{equation}
		To find the parallel and perpendicular components, we must use the unit velocity vector $\bf u = v/|v|$ and its projections:
		\begin{equation}
			{\bf r}_{\parallel} = (\bf r \cdot u)u
		\end{equation}
		\begin{equation}
			{\bf r}_{\perp} = \bf r - (r \cdot u)u
		\end{equation}
		Substituting these expressions:
		\begin{equation}
			{\bar t} = \gamma [t - (\bf r \cdot u)u \cdot {\bf v}] 	
		\end{equation} 
		\begin{equation}
			{\bf \bar{r}}_{\parallel} = \gamma [({\bf r \cdot u)u -} t{\bf v}] 
		\end{equation}
		\begin{equation}
			{\bf {\bar r}}_{\perp} = \bf r - (r \cdot u)u
		\end{equation}
		\begin{equation}
			{\bf \bar{r}} = {\bf \bar{r}}_{\parallel} + {\bf \bar{r}}_{\perp} = \left[{\bf r} + (\gamma -1)({\bf r \cdot u)u} - \gamma t{\bf v}\right] 
		\end{equation}
		Evaluating these expressions in Cartesian components (in the time and x-direction for simplicity; the rest can be figured out via symmetry):
		\begin{equation}
			{\bar t} = [\gamma]t + [-\gamma v_x] x  + [-\gamma v_y] y + [-\gamma v_z] z
		\end{equation}
		\begin{equation}
			\bar x = [-\gamma v_x] t + \left[1 + (\gamma -1)\frac{v_x^2}{|{\bf v}|^2}\right]x + \left[(\gamma - 1)\frac{v_x v_y}{|{\bf v}|^2}\right]y	+ \left[(\gamma - 1)\frac{v_x v_z}{|{\bf v}|^2}\right]z
		\end{equation}
		Writing $\alpha = (\gamma -1)/|{\bf v}|^2$ and arranging the other components into $ V^{\bar \beta} = \Lambda^{\bar \beta}_{\;\;\alpha} V^{\alpha}$ as:
		\begin{equation}
			\begin{bmatrix}
		 		\bar{t} \\
				\bar{x} \\
				\bar{y} \\
				\bar{z}
			\end{bmatrix} =
			\begin{bmatrix}
				\gamma & -\gamma v_x & -\gamma v_y & -\gamma v_z \\
				-\gamma v_x & 1 + \alpha {v}^2_x & \alpha v_x v_y & \alpha v_x v_z \\
				-\gamma v_y & \alpha v_y v_x & 1 + \alpha {v}^2_y & \alpha v_y v_z \\
				-\gamma v_z & \alpha v_z v_x & \alpha v_z v_y & 1 + \alpha {v}^2_z
			\end{bmatrix}
			\begin{bmatrix}
		 		{t} \\
				{x} \\
				{y} \\
				{z}
			\end{bmatrix} 	
		\end{equation}
		The three-velocity ${\bf v} = 0.5{\bf e}_x + 0.5{\bf e}_y + 0.5{\bf e}_z$ in this question and the Lorentz transformation is $\Lambda^{\bar \beta}_{\;\;\alpha}$ with $\gamma = 2$ and $\alpha = 4/3$. To go to the MCRF, $ {\vec V} \stackrel{MCRF}\longrightarrow (1, 0, 0, 0)$, which follows through when multiplying $ {\vec V} \stackrel{O}\longrightarrow (2, 1, 1, 1) $ by the Lorentz transformation:
		\begin{equation}
			\begin{bmatrix}
		 		{1} \\
				{0} \\
				{0} \\
				{0}
			\end{bmatrix} =
			\begin{bmatrix}
				2 & -1 & -1 & -1 \\
				-1 & 4/3 & 1/3 & 1/3 \\
				-1 & 1/3 & 4/3 & 1/3 \\
				-1 & 1/3 & 1/3 & 4/3
			\end{bmatrix}
			\begin{bmatrix}
		 		{2} \\
				{1} \\
				{1} \\
				{1}
			\end{bmatrix}
		\end{equation}
		Using this transformation to transform the cosmic ray's momentum ${\vec P} \stackrel{O} \longrightarrow (300, 299, 0, 0) \times 10^{-27}$ kg to the MCRF of the rocket ship:
		\begin{equation}
			\begin{bmatrix}
		 		{301} \\[0.3 pt]
				{296/3} \\[0.3 pt]
				{-601/3} \\[0.3 pt]
				{-601/3}
			\end{bmatrix} =
			\begin{bmatrix}
				2 & -1 & -1 & -1 \\
				-1 & 4/3 & 1/3 & 1/3 \\
				-1 & 1/3 & 4/3 & 1/3 \\
				-1 & 1/3 & 1/3 & 4/3
			\end{bmatrix}
			\begin{bmatrix}
		 		{300} \\[0.3 pt]
				{299} \\[0.3 pt]
				{0} \\[0.3 pt]
				{0}
			\end{bmatrix}
		\end{equation}
		Which gives ${\vec P} \stackrel{MCRF} \longrightarrow (301, 296/3, -601/3, -601/3) \times 10^{-27}$ kg \\\\
		\stepcounter{subpart2}
		(\alph{subpart2})
		Using $-{\vec P} \cdot \vec V = {\mathrm{\bar E}} = -(-300 \times 2 + 299 \times 1) \times 10^{-27} = 301 \times 10^{-27}$, which is the energy of the cosmic ray.
	\end{subequations}

\end{chapter2}

\begin{chapter2}\label{prob: 31}
	\begin{subequations}
		This is elementary conservation of momentum. The initial four-momentum of the photon (for a vertical mirror to the right of the photon) is given by:
		\begin{equation}
			\vec p_1 \stackrel{O}\longrightarrow (h\nu, h\nu\cos\theta, h\nu\sin\theta, 0)
		\end{equation}
		After reflection, its final four-momentum is:
		\begin{equation}
			\vec p_2 \stackrel{O}\longrightarrow (h\nu, -h\nu\cos\theta, h\nu\sin\theta, 0)
		\end{equation}
		The change in momentum is thus:
		\begin{equation}
			\vec p_2 - \vec p_1 \stackrel{O}\longrightarrow (0, -2h\nu\cos\theta, 0, 0)
		\end{equation}
		Therefore, the momentum transferred to the mirror is $2h\nu\cos\theta$ by conservation of four-momentum.\\
		If the photon were absorbed, the momentum transferred to the mirror would be $h\nu\cos\theta$??
	\end{subequations}
\end{chapter2}
	
\begin{chapter2}\label{prob: 32}
	\begin{subequations}
		The four-momentum components of the photon and electron before the collision are:
		\begin{equation}
			\vec p_1 = (h\nu_i, h\nu_i, 0, 0) \;\;\; \vec p_2 = (m, 0, 0, 0)
		\end{equation}
		The four-momentum components of the scattered photon and electron after the collision are:
		\begin{equation}
			\vec p_3 = (h\nu_f, h\nu_f \cos\theta, h\nu_f \sin\theta, 0) \;\;\; \vec p_4 = (p^0, p^1, p^2, p^3)
		\end{equation}
		Using the conservation of four-momentum, the four-momentum of the scattered electron is:
		\begin{equation}
			\vec p_4 = \vec p_1 + \vec p_2 - \vec p_3 =
			\begin{bmatrix}
				h(\nu_i - \nu_f) + m \\
				h\nu_i - h\nu_f \cos\theta \\
				-h\nu_f \sin\theta \\
				0
			\end{bmatrix}
		\end{equation}
		Now, we know that the four-momentum of a particle is $\vec p \cdot \vec p = -m^2$. So, finding the magnitude of the scattered electron's four-momentum:
		\begin{gather}
			{\vec p}_4 \cdot {\vec p}_4 =
			\begin{bmatrix}
				h(\nu_i - \nu_f) + m \\
				h\nu_i - h\nu_f \cos\theta \\
				-h\nu_f \sin\theta \\
				0
			\end{bmatrix} \cdot
			\begin{bmatrix}
				h(\nu_i - \nu_f) + m \\
				h\nu_i - h\nu_f \cos\theta \\
				-h\nu_f \sin\theta \\
				0
			\end{bmatrix} \\
			= -(h(\nu_i - \nu_f) + m)^2 + (h\nu_i - h\nu_f \cos\theta)^2 + (-h\nu_f \sin\theta)^2 = -m^2
		\end{gather}
		Expanding the expression and cancelling terms, we're left with:
		\begin{equation}
			-2h^2\nu_i \nu_f (1-\cos\theta) + 2hm(\nu_i - \nu_f) = 0
		\end{equation}
		Rearranging the terms, we get the final expression for the frequency shift in Compton scattering:
		\begin{equation}
			\frac{1}{\nu_f} = \frac{1}{\nu_i} + \frac{h}{m}(1-\cos\theta)
		\end{equation}
	\end{subequations}
\end{chapter2}

\begin{chapter2}\label{prob: 33}
	
\end{chapter2}

\begin{chapter2}\label{prob: 34}
	
\end{chapter2}

\begin{chapter2}\label{prob: 35}
	
\end{chapter2}

\newtheorem{chapter3}{Problem}
\newcounter{subpart3}[chapter3]
\newcounter{subsubpart3}[chapter3]

\chapter{Tensor Analysis in Special Relativity}

\begin{chapter3}\label{prob: 1}
	
\end{chapter3}

\begin{chapter3}\label{prob: 2}
	
\end{chapter3}

\begin{chapter3}\label{prob: 3}
	\begin{subequations}
		\stepcounter{subpart3}
		(\alph{subpart3})

		\stepcounter{subpart3}
		(\alph{subpart3})
		\begin{equation}
			\tilde{p}(\vec A) = -1 \times 2 + 1 \times 1 = -1
		\end{equation}
		\begin{equation}
			\tilde{p}(\vec B) = 2 \times 1 = 2
		\end{equation}
		\begin{equation}
			\tilde{p}(\vec A - 3 \vec B) = -1 \times 2 + 1 \times -6 = -8
		\end{equation}
		\begin{equation}
			\tilde{p}(\vec A) - 3\tilde{p}(\vec B) = -1 - 6 = -7
		\end{equation}
	\end{subequations}	
\end{chapter3}

\begin{chapter3}\label{prob: 4}
	\begin{subequations}
		\stepcounter{subpart3}
		(\alph{subpart3})
		To check if the vectors are linearly independent, we should represent the vectors as the columns of a matrix and find the determinant:
		\begin{equation}
			\begin{vmatrix}
				2 & 1 & 0 & -3 \\
				1 & 2 & 0 & 2 \\
				1 & 0 & 1 & 0 \\
				0 & 0 & 1 & 0
			\end{vmatrix}
			= 8
		\end{equation}
		Since the determinant is non-zero, none of the vectors are expressible as linear combinations of each other, so the vectors are linearly independent. \\\\
		\stepcounter{subpart3}
		(\alph{subpart3})
		Representing the one-form $\tilde p = (w, x, y, z)$ as a row vector:
		\begin{equation}
			\begin{bmatrix}
				1 & -1 & -1 & 0
			\end{bmatrix}
			=
			\begin{bmatrix}
				w & x & y & z
			\end{bmatrix}
			\begin{bmatrix}
				2 & 1 & 0 & -3 \\
				1 & 2 & 0 & 2 \\
				1 & 0 & 1 & 0 \\
				0 & 0 & 1 & 0
			\end{bmatrix}
		\end{equation}
		Solving the system via row reduction, we get $\tilde p = (-1/4, -3/8, 15/8, -23/8)$
	\end{subequations}
\end{chapter3}

\begin{chapter3}\label{prob: 5}
	Vector components transform as $A^{\bar \alpha} = \Lambda^{\bar \alpha}_{\;\;\beta} A^{\beta}$ and one-form components transform as $p_{\bar \alpha} = \Lambda_{\;\;\bar \alpha}^{\mu} p_{\mu}$. So the expression $A^{\bar \alpha} p_{\bar \alpha} = (\Lambda^{\bar \alpha}_{\;\;\beta} A^{\beta}) \left(\Lambda_{\;\;\bar \alpha}^{\mu} p_{\mu}\right)$. \\
	Since the $\Lambda$s are just numbers, we can pull them out of the components in the different frame, giving $\Lambda^{\bar \alpha}_{\;\;\beta} \Lambda_{\;\;\bar \alpha}^{\mu} A^{\beta} p_{\mu}$. \\ 
	Now, the expression $\Lambda^{\bar \alpha}_{\;\;\beta} \Lambda_{\;\;\bar \alpha}^{\mu}$ is a matrix multiplication of numbers, and since it's transforming numbers back to the same reference frame, they're inverse transformations of each other, giving the Kronecker delta: $\delta^{\mu}_{\;\;\beta}A^{\beta} p_{\mu}$ \\
	Performing the summation over the one-form components $p_{\beta} = \delta^{\mu}_{\;\;\beta} p_{\mu}$ we get the final expression $A^{\bar \alpha} p_{\bar \alpha} = A^{\beta} p_{\beta}$, which proves that a vector fed into a one-form produces a frame-invariant number.
\end{chapter3}

\begin{chapter3}\label{prob: 6}
	\begin{subequations}
		\stepcounter{subpart3}
		(\alph{subpart3})
		Let $\tilde p$ be the one-form $\tilde \omega^0$, one of the basis one-forms dual to one of the basis vectors in the set $\{\vec e_{\alpha}\}$. Then we evaluate the expression (keeping in mind the property $\tilde \omega^{\alpha}(\vec e_{\beta}) = \delta^{\alpha}_{\;\;\beta})$:
		\begin{equation}
			\tilde \omega^0 (\vec e_{\alpha}) \tilde \lambda^{\alpha} = \tilde \omega^0 (\vec e_0) \tilde \lambda^0 + \tilde \omega^0 (\vec e_1) \tilde \lambda^1 + ... = (1, 1, 0, 0) = \lambda^0 \neq \tilde \omega^0  
		\end{equation}
		This is from the definition that $\tilde \lambda^{\beta}$ is not the dual of $\vec e_{\alpha}$.\\\\
		\stepcounter{subpart3}
		(\alph{subpart3})
		A suitable linear combination to generate $\tilde p = \tilde \lambda^0 + \tilde \lambda^3 = (1,1,1,1)$, so the numbers $l_{\alpha}$ are: $l_0 = 1, l_1 = 0, l_2 = 0, l_3 = 1$.
	\end{subequations}
\end{chapter3}

\begin{chapter3}\label{prob: 7}
	This is an analog of proving the transformation properties of basis vectors. Let us begin with the expression $\tilde p = l_{\bar \alpha} \tilde\omega^{\bar \alpha} = l_{\alpha} \tilde\omega^{\alpha}$. We know that one-form components transform inversely as vector components do, so $l_{\alpha} = \Lambda^{\bar \mu}_{\;\;\alpha} l_{\bar \mu}$ The following substitutions prove the result:
	$$l_{\bar \alpha} \tilde \omega^{\bar \alpha} = (\Lambda^{\bar \beta}_{\;\;\alpha} l_{\bar \beta}) \tilde\omega^{\alpha} $$
	$$l_{\bar \alpha} \tilde \omega^{\bar \alpha} - (\Lambda^{\bar \beta}_{\;\;\alpha} l_{\bar \beta}) \tilde\omega^{\alpha} = 0$$
	$$l_{\bar \alpha} \tilde \omega^{\bar \alpha} - l_{\bar \beta} \Lambda^{\bar \beta}_{\;\;\alpha} \tilde \omega^{\alpha} = 0$$
	Changing the dummy indices $\bar \beta \to \bar \alpha$ and $\alpha \to \beta$, we get:
	$$l_{\bar \alpha} (\tilde \omega^{\bar \alpha} - \Lambda^{\bar \alpha}_{\;\;\beta} \tilde \omega^{\beta}) = 0 $$
	Which gives our result:
	$$ \tilde \omega^{\bar \alpha} = \Lambda^{\bar \alpha}_{\;\;\beta} \tilde \omega^{\beta} $$
\end{chapter3}

\begin{chapter3}\label{prob: 8}
	The basis one-forms are as follows:
	\begin{subequations}
		\begin{center}
			\begin{tikzpicture}
				\draw[help lines, color=gray!90, dashed] (0,0) grid (3,3);
				\draw[->,ultra thick] (0,0)--(3,0) node[right]{$x$};
				\draw[->,ultra thick] (0,0)--(0,3) node[above]{$t$};
				\draw[-, thick] (3,1)--(-0.3,1);
				\draw node at (-0.8, 1.55){$\tilde {\dd}t$};
				\draw[-, thick] (1,3)--(1,-0.3);
				\draw node at (1.5, -0.8){$\tilde {\dd}x$};
				\draw[-, thick] (-0.3,2)--(3,2);
				\draw[-, thick] (2,-0.3)--(2,3);
				\draw node at (-0.2,-0.2){$O$};	
				\draw [decorate,decoration={brace,amplitude=5pt}] (2,-0.4) -- (1,-0.4);
				\draw [decorate,decoration={brace,amplitude=5pt}] (-0.4,1) -- (-0.4,2);
			\end{tikzpicture}
		\end{center}
	\end{subequations}
\end{chapter3}

\begin{chapter3}\label{prob: 9}
	Following the hint, For $\mathcal{P}$ we see that the basis vectors ${\bf e}_x$ and ${\bf e}_y$ cross the 0 and 2 surfaces respectively with respect to the origin, making the components of the gradient $\tilde {\dd}T \longrightarrow (-1,-1.5)$
\end{chapter3}

\begin{chapter3}\label{prob: 10}
	\begin{subequations}
		\stepcounter{subpart3}
		(\alph{subpart3})
		Transforming the vector coordinates to the frame $\bar O, \{x^{\bar \alpha}\}$ and back, the coordinate transformation is:
		\begin{equation}
			x^{\alpha} = \pdv{x^{\alpha}}{x^{\bar\alpha}} \pdv{x^{\bar\alpha}}{x^{\beta}} x^{\beta}
		\end{equation}
		We know from the chain rule that:
		$$ \pdv{x^{\alpha}}{x^{\beta}} = \pdv{x^{\alpha}}{x^{\bar\alpha}} \pdv{x^{\bar\alpha}}{x^{\beta}}$$
		Since the components are just numbers in the same frame, we get the result:
		$$  \pdv{x^{\alpha}}{x^{\beta}} = \delta^{\alpha}_{\;\;\beta}$$
		\stepcounter{subpart3}
		(\alph{subpart3})

	\end{subequations}
\end{chapter3}

\begin{chapter3}\label{prob: 11}
	3.14:
	$$ \dv{\phi}{\tau} = \phi_{,\alpha}U^{\alpha} $$
	3.15: This requires the one-form basis $\{\tilde \omega^{\alpha}\}$
	$$ \tilde{\dd}\phi = \phi_{,\alpha}\tilde \omega^{\alpha} $$
	3.18:
	$$ x^{\beta}_{\;\;,\bar\alpha} = \Lambda^{\beta}_{\;\;\bar\alpha} $$ 
	$$ \frac{\phi_{,\bar\alpha}}{\phi_{,\beta}} =\Lambda^{\beta}_{\;\;\bar\alpha} $$
	A much more compact way of writing and understanding tensor expressions. Just keep track of the indices and find the respective partial derivatives!
\end{chapter3}

\begin{chapter3}\label{prob: 12}
	\begin{subequations}
		\stepcounter{subpart3}
		(\alph{subpart3})
		This follows from the definition of a normal one-form: A one-form is said to be normal to a surface if its value is zero on every vector tangent to the surface. By inverting and negating both sides of the statement, we get: If the vector $\vec V$ is not tangent to the surface, then $\tilde n(\vec V) = 0$. \\\\
		\stepcounter{subpart3}
		(\alph{subpart3})
		$\tilde n(\vec V) = n_{\alpha}V^{\alpha} > 0$, which means that the corresponding components of the one-form and vector are of the same sign. Any vector $\vec W$ having corresponding components with the same signs as those of $\vec V$ will result in $\tilde n(\vec W) > 0 $ because like signs multiplied result in positive numbers. \\\\
		\stepcounter{subpart3}
		(\alph{subpart3})
		Consider a different normal one-form $\tilde p$ to S. Since S is the two-dimensional plane $x = 0$ (the y-z plane, essentially), $\tilde p$'s components in the standard one-form basis are $(p_x, 0, 0)$ and $\tilde n$'s components are $(n_x, 0, 0)$, so they are multiples of each other. \\\\
		\stepcounter{subpart3}
		(\alph{subpart3})
	\end{subequations}
\end{chapter3}

\begin{chapter3}\label{prob: 13}
	Let X be the surface of constant f. We take an object at point A and displace it along f. From the property of one-forms:
	$$ \tilde{\dd}f = \pdv{f}{x^{\alpha}} \tilde{\dd}x^{\alpha}$$
	The change in f with respect to any of the coordinates is 0, so any locally ``tangential'' movement along the surface results in no change of the one-form $\tilde{\dd}f$, indicating that only movements normal to the surface are registered by the property.
\end{chapter3}

\begin{chapter3}\label{prob: 14}
	\begin{subequations}
		Let $\vec A = (1, 0, 1, 0)$ and $\vec B = (1, -1, 0, 0)$. The tensors $\tilde p \otimes \tilde q$ and $\tilde q \otimes \tilde p$ are multi-linear maps that take two vectors as arguments and output the real numbers as follows:
		\begin{equation}
			\tilde p (\vec A) \tilde q (\vec B) = p_{\alpha} A^{\alpha} q_{\beta} B^{\beta} = 1 \times -1 = -1
		\end{equation}
		\begin{equation}
			\tilde q (\vec A) \tilde p (\vec B) = q_{\alpha} A^{\alpha} p_{\beta} B^{\beta} = (-1 + 1) \times (1 - 1) = 0
		\end{equation}
		The 16 components of $\tilde p \otimes \tilde q$ are found by feeding them the basis vectors $\tilde p(\vec e_{\alpha}) \tilde q (\vec e_{\beta})= p_{\alpha} q_{\beta}$. These components can be arranged into the $4\times 4$ matrix:
		\begin{equation}
			\begin{bmatrix}
				-1 & 0 & 1 & 0 \\
				-1 & 0 & 1 & 0 \\
				0 & 0 & 0 & 0 \\
				0 & 0 & 0 & 0
			\end{bmatrix}
		\end{equation}
	\end{subequations}
\end{chapter3}

\begin{chapter3}\label{prob: 15}
	\begin{subequations}
		The tensor ${\bf f} = f_{\alpha\beta} \tilde \omega^{\alpha\beta}$ must follow tensor transformation properties. So, when fed with basis vectors $(\vec e_{\mu}, \vec e_{\nu})$, we must get the components $f_{\mu\nu}$. The main question left to be answered is the description of $\omega^{\alpha\beta}$, which seems to be a basis ``two-form'' that takes in the previously mentioned basis vectors and transforms the components $f_{\alpha\beta}$ into $f_{\mu\nu}$.
	\end{subequations}
\end{chapter3}

\begin{chapter3}\label{prob: 16}
	\begin{subequations}
		\stepcounter{subpart3}
		(\alph{subpart3})
		The definition of a symmetric tensor is: ${\bf f}(\vec A, \vec B) = {\bf f}(\vec B, \vec A), \forall \; \vec A, \vec B$. Therefore, all we have to check is if ${\bf h}_{(S)}(\vec A, \vec B) = {\bf h}_{(S)}(\vec B, \vec A)$. Following the definition, we get:
		\begin{equation}
			{\bf h}_{(S)}(\vec A, \vec B) = \frac{1}{2}{\bf h}(\vec A, \vec B) + \frac{1}{2}{\bf h}(\vec B, \vec A)
		\end{equation}
		\begin{equation}
			{\bf h}_{(S)}(\vec B, \vec A) = \frac{1}{2}{\bf h}(\vec B, \vec A) + \frac{1}{2}{\bf h}(\vec A, \vec B)
		\end{equation}
		Which are indeed equal and satisfy the definition.\\\\
		\stepcounter{subpart3}
		(\alph{subpart3})
		Switching the input vectors:
		\begin{equation}
			{\bf h}_{(A)}(\vec B, \vec A) = \frac{1}{2}{\bf h}(\vec B, \vec A) - \frac{1}{2}{\bf h}(\vec A, \vec B) = -{\bf h}_{(A)}(\vec A, \vec B)
		\end{equation}
		Which is indeed antisymmetric.\\\\
		\stepcounter{subpart3}
		(\alph{subpart3})
		Let ${\bf h} = \tilde p \otimes \tilde q$. Then:
		\begin{equation}
			{\bf h}_{(s)} =			
		\end{equation}
		\stepcounter{subpart3}
		(\alph{subpart3})
		From the definition:
		\begin{equation}
			{\bf h}_{(A)}(\vec A, \vec A) = \frac{1}{2}{\bf h}(\vec A, \vec A) - \frac{1}{2}{\bf h}(\vec A, \vec A) = 0
		\end{equation}
		\stepcounter{subpart3}
		(\alph{subpart3})
	\end{subequations}
\end{chapter3}

\begin{chapter3}\label{prob: 17}
	\begin{subequations}
		\stepcounter{subpart3}
		(\alph{subpart3})
		A tensor that takes two vector arguments must consist of two one-forms taking the arguments, and is expressible in the form:
		\begin{gather}
			{\bf h}(\;\;,\vec A) = \tilde{p}(\;\;)\tilde{q}(\vec A)
		\end{gather}
		The objects $\tilde{q}(\vec A) = \beta_{A}$ and $\tilde{q}(\vec B) = \beta_B$ must be one-form expressions via proof:
		\begin{gather}
			\gamma \tilde{q}(\vec A) = \frac{something}{darkside} 	
		\end{gather} 
	\end{subequations}
\end{chapter3}

\begin{chapter3}\label{prob: 18}
	\begin{subequations}
		\stepcounter{subpart3}
		(\alph{subpart3})
		The one-form components can be found from the metric: $V_{\alpha} = g_{\alpha\beta}V^{\beta}$, which is simply inverting the sign of the time component. Therefore, the one-forms are:
		\begin{gather}
			\tilde A \stackrel{O}\longrightarrow (-1, 0, 1, 0) \\
			\tilde B \stackrel{O}\longrightarrow (0, 1, 1, 0) \\
			\tilde C \stackrel{O}\longrightarrow (1, 0, -1, 0) \\
			\tilde D \stackrel{O}\longrightarrow (0, 0, 1, 1) 
		\end{gather}
		The vector components can be found from the metric mapping: $V^{\alpha} = g^{\alpha\beta}V_{\beta}$, which is simply inverting the sign of the time component again. Therefore, the vectors are:
		\begin{gather}
			\vec P \stackrel{O}\longrightarrow (-3, 0, -1, -1) \\
			\vec Q \stackrel{O}\longrightarrow (-1, -1, 1, 1) \\
			\vec R \stackrel{O}\longrightarrow (0, -5→, -1, 0) \\
			\vec S \stackrel{O}\longrightarrow (2, 1, 0, 0) 
		\end{gather}
	\end{subequations}
\end{chapter3}

\begin{chapter3}\label{prob: 19}
	\begin{subequations}
		\stepcounter{subpart3}
		(\alph{subpart3})
		\begin{equation}
			\eta^{\alpha\beta}\eta_{\alpha\beta} =
			\begin{bmatrix}
				-1 & 0 & 0 & 0 \\
				0 & 1 & 0 & 0 \\
				0 & 0 & 1 & 0 \\
				0 & 0 & 0 & 1 
			\end{bmatrix}
			\begin{bmatrix}
				-1 & 0 & 0 & 0 \\
				0 & 1 & 0 & 0 \\
				0 & 0 & 1 & 0 \\
				0 & 0 & 0 & 1
			\end{bmatrix}
			= 
			\begin{bmatrix}
				1 & 0 & 0 & 0 \\
				0 & 1 & 0 & 0 \\
				0 & 0 & 1 & 0 \\
				0 & 0 & 0 & 1
			\end{bmatrix}
		\end{equation}
		\stepcounter{subpart3}
		(\alph{subpart3})
		Transforming $\tilde q$ into a vector $\vec Q$ by using the metric tensor and feeding it to $\tilde p(\;)$:
		\begin{gather}
			q^{\alpha} = g^{\alpha\beta}q_{\beta} \longrightarrow \vec Q \stackrel{O}\longrightarrow (-q_0, q_1, q_2, q_3) \\
			\tilde p \cdot \tilde q = \tilde p(\vec Q) = -p_0 q_0 + p_1 q_1 + p_2 q_2 + p_3 q_3
		\end{gather}
	\end{subequations}
\end{chapter3}

\begin{chapter3}\label{prob: 20}
	\begin{subequations}
		\stepcounter{subpart3}
		(\alph{subpart3})
		The inverse of the matrix $\{\Lambda^{\bar \alpha}_{\;\;\beta}\}$ is $\{\Lambda^{\beta}_{\;\;\bar\alpha}\}$ from $V^{\beta} = \Lambda^{\beta}_{\;\;\bar\alpha}V^{\bar\alpha}$. Its transpose is simply found by switching the index positions and switching the frames: $\{\Lambda^{\alpha}_{\;\;\bar\beta}\}$. This is the transformation matrix we've seen for one-form components: $P_{\bar\beta} = \Lambda^{\alpha}_{\;\;\bar\beta}P_{\alpha}$. \\\\
		\stepcounter{subpart3}
		(\alph{subpart3})

	\end{subequations}
\end{chapter3}

\begin{chapter3}\label{prob: 21}
	\begin{subequations}
		\stepcounter{subpart3}
		(\alph{subpart3})
		The situation described is:
		\begin{center}
			\begin{tikzpicture}
				\draw[help lines, color=gray!90, dashed] (0,0) grid (1.5,1.5);
				\draw[->,ultra thick] (0,0)--(1.5,0) node[right]{$x$};
				\draw[->,ultra thick] (0,0)--(0,1.5) node[above]{$t$};
				\draw[-, thick] (1,1)--(1,0) node[below]{$x=1$};
				\draw[-, thick] (1,1)--(0,1) node[left]{$t=1$};
			\end{tikzpicture}
		\end{center}
		The outward normal one-forms and their associated vectors are:
		\begin{gather}
			t = 0 \rightarrow -\tilde{\dd}t, \;\; t = 1 \rightarrow \tilde{\dd}t, \;\; x = 0 \rightarrow -\tilde{\dd}x, \;\; x = 1 \rightarrow \tilde{\dd}x \\
			-V^0(t = 0), \;\; V^0(t = 1), \;\; -V^1(x = 0), \;\; V^1(x = 1)
		\end{gather}
		\stepcounter{subpart3}
		(\alph{subpart3})
		The situation described is:
		\begin{center}
			\begin{tikzpicture}
				\draw[help lines, color=gray!90, dashed] (0,0) grid (2.5,1.5);
				\draw[->,ultra thick] (0,0)--(2.5,0) node[right]{$x$};
				\draw[->,ultra thick] (0,0)--(0,1.5) node[above]{$t$};
				\draw[-, thick] (1,0)--(1,1);
				\draw[-, thick] (1,1)--(2,1);
				\draw[-, thick] (2,1)--(1,0);
			\end{tikzpicture}
		\end{center}
	\end{subequations}
\end{chapter3}

\begin{chapter3}\label{prob: 22}
	
\end{chapter3}

\begin{chapter3}\label{prob: 23}
	\begin{subequations}
		\stepcounter{subpart3}
		(\alph{subpart3})
		Let $A$ be the set of all $\pmqty{M \\ N}$ tensors for fixed $M, N \in \mathcal{R}$, e.g. an element in $A$ is \\
		${\bf T}(\underbrace{\vec A, \vec B, \vec C, \vec D, ...}_{M\;vectors};\underbrace{\tilde T,\tilde U, \tilde V, \tilde W, ...}_{N\;one-forms})$. The addition of two tensors of the same rank and multiplication of tensors by numbers are defined as:
		\begin{gather}
			{\bf T}(\;) = {\bf R}(\;) + {\bf S}(\;) \\
			{\bf P}(\;) = \alpha {\bf Q}(\;)   
		\end{gather}
		This description is consistent with the defined set: It is closed under addition and scalar multiplication. The addition of these tensors is a commutative operation because the tensors are all fed the same arguments.
	\end{subequations}
\end{chapter3}

\begin{chapter3}\label{prob: 24}
	\begin{subequations}
		\stepcounter{subpart3}
		(\alph{subpart3})
		\stepcounter{subsubpart3}
		(\roman{subsubpart3})
		\begin{gather}
			M^{\pqty{\alpha \beta}} = \frac{1}{2}\pqty{M^{\alpha \beta} + M^{\beta \alpha}} = \frac{1}{2} \pqty{
			\begin{bmatrix}
				0 & 1 & 0 & 0 \\
				1 & -1 & 0 & 2 \\
				2 & 0 & 0 & 1 \\
				1 & 0 & -2 & 0
			\end{bmatrix}
			+
			\begin{bmatrix}
				0 & 1 & 2 & 1 \\
				1 & -1 & 0 & 0 \\
				0 & 0 & 0 & -2 \\
				0 & 2 & 1 & 0 \\
			\end{bmatrix}} \\
			= \frac{1}{2}
			\begin{bmatrix}
				0 & 2 & 2 & 1 \\
				2 & -2 & 0 & 2 \\
				2 & 0 & 0 & -1 \\
				1 & 2 & -1 & 0
			\end{bmatrix} \\
			M^{\bqty{\alpha \beta}} = \frac{1}{2}\pqty{M^{\alpha \beta} - M^{\beta \alpha}} = \frac{1}{2} \pqty{
			\begin{bmatrix}
				0 & 1 & 0 & 0 \\
				1 & -1 & 0 & 2 \\
				2 & 0 & 0 & 1 \\
				1 & 0 & -2 & 0
			\end{bmatrix}
			-
			\begin{bmatrix}
				0 & 1 & 2 & 1 \\
				1 & -1 & 0 & 0 \\
				0 & 0 & 0 & -2 \\
				0 & 2 & 1 & 0 \\
			\end{bmatrix}} \\
			= \frac{1}{2}
			\begin{bmatrix}
				0 & 0 & -2 & -1 \\
				0 & 0 & 0 & 2 \\
				2 & 0 & 0 & 1 \\
				1 & -2 & -3 & 0
			\end{bmatrix}
		\end{gather}
		\stepcounter{subsubpart3}
		(\roman{subsubpart3})
		Lowering the index using the metric tensor:
		\begin{gather}
			M^{\alpha}_{\;\;\beta} = g_{\mu \beta} M^{\alpha \mu} =
			\begin{bmatrix}
				-1 & 0 & 0 & 0 \\
				0 & 1 & 0 & 0 \\
				0 & 0 & 1 & 0 \\
				0 & 0 & 0 & 1
			\end{bmatrix}
			\begin{bmatrix}
				0 & 1 & 0 & 0 \\
				1 & -1 & 0 & 2 \\
				2 & 0 & 0 & 1 \\
				1 & 0 & -2 & 0
			\end{bmatrix}\\
			=
			\begin{bmatrix}
				0 & -1 & 0 & 0 \\
				1 & -1 & 0 & 2 \\
				2 & 0 & 0 & 1 \\
				1 & 0 & -2 & 0
			\end{bmatrix}
		\end{gather}
		\stepcounter{subsubpart3}
		(\roman{subsubpart3})
		Lowering the other index:
		\begin{gather}
			M_{\alpha}^{\;\;\beta} = g_{\alpha \mu} M^{\mu \beta} =
			\begin{bmatrix}
				-1 & 0 & 0 & 0 \\
				0 & 1 & 0 & 0 \\
				0 & 0 & 1 & 0 \\
				0 & 0 & 0 & 1
			\end{bmatrix}
			\begin{bmatrix}
				0 & 1 & 0 & 0 \\
				1 & -1 & 0 & 2 \\
				2 & 0 & 0 & 1 \\
				1 & 0 & -2 & 0
			\end{bmatrix}\\
			=
			\begin{bmatrix}
				0 & 1 & 0 & 0 \\
				-1 & -1 & 0 & 2 \\
				-2 & 0 & 0 & 1 \\
				-1 & 0 & -2 & 0
			\end{bmatrix}
		\end{gather}
		\stepcounter{subsubpart3}
		(\roman{subsubpart3})
		Lowering the $\pmqty{1 \\ 1}$ tensor index:
		\begin{gather}
			M_{\alpha \beta} = g_{\mu \beta} M_{\alpha}^{\;\;\mu} =
			\begin{bmatrix}
				-1 & 0 & 0 & 0 \\
				0 & 1 & 0 & 0 \\
				0 & 0 & 1 & 0 \\
				0 & 0 & 0 & 1
			\end{bmatrix}
			\begin{bmatrix}
				0 & 1 & 0 & 0 \\
				-1 & -1 & 0 & 2 \\
				-2 & 0 & 0 & 1 \\
				-1 & 0 & -2 & 0
			\end{bmatrix}
			= \\
			\begin{bmatrix}
				0 & 1 & 0 & 0 \\
				1 & -1 & 0 & 2 \\
				2 & 0 & 0 & 1 \\
				1 & 0 & -2 & 0
			\end{bmatrix}
		\end{gather}
	\end{subequations}
\end{chapter3}

\begin{chapter3}\label{prob: 25}
	
\end{chapter3}

\begin{chapter3}\label{prob: 26}
	
\end{chapter3}

\begin{chapter3}\label{prob: 27}
	
\end{chapter3}

\begin{chapter3}\label{prob: 28}
	\begin{subequations}
		Notice the following about the expression:
		\begin{gather}
			\dv{{\bf T}}{\tau} = (T^{\alpha}_{\;\;\beta,\gamma}{\tilde \omega}^\beta \otimes {\vec e}_{\alpha})U^{\gamma} = \pdv{T^{\alpha}_{\;\;\beta}}{x^{\gamma}}\dv{x^{\gamma}}{\tau} \\
			\dd{{\bf T}} =  (T^{\alpha}_{\;\;\beta,\gamma}{\tilde \omega}^\beta \otimes {\vec e}_{\alpha})\dd{x}^{\gamma} = \nabla{\bf T}\pqty{\dd{\vec s}}
		\end{gather}
		For $\nabla{\bf T}$ to be a legitimate tensor, there must exist a basis one-form $\tilde \omega^{\gamma}$ for the $\gamma$ component, giving:
		\begin{gather}
			\nabla{\bf T} = T^{\alpha}_{\;\;\beta,\gamma}{\tilde \omega}^{\beta} \otimes {\tilde \omega}^{\gamma} \otimes {\vec e}_{\alpha}
		\end{gather}
		Evidently the gradient of $\bf T$ is a $\pmqty{1 \\ 2}$ tensor.
	\end{subequations}
\end{chapter3}

\begin{chapter3}\label{prob: 29}
	
\end{chapter3}

\begin{chapter3}\label{prob: 30}
	\begin{subequations}
		\stepcounter{subpart3}
		(\alph{subpart3})
		\begin{gather}
			\vec U \cdot \vec U = -(1+t^2)^2 + (t^2)^2 + (\sqrt{2} t)^2 = -1 \\
			\vec U \cdot \vec D = -(1+t)x + 5xt^3 + 2t^2 =  \\
			\vec D \cdot \vec D = -x^2 + 25t^2 x^2 + 2t^2 \\
		\end{gather}
		$\vec U$ is suitable, but $\vec D$ is not. \\\\
		\stepcounter{subpart3}
		(\alph{subpart3})
		The spatial velocity is:
		\begin{gather}
		 	{\va v} = \frac{t^2}{1+t^2}{\va e}_x + \frac{\sqrt{2}t}{1+t^2}{\va e}_y \\
		 	{\va v}_{t\to 0} = {\va 0} \\
		 	{\va v}_{t \to \infty} = {\va e}_x
		\end{gather} 
		\stepcounter{subpart3}
		(\alph{subpart3})
		This is the same as finding the one-form components: ${\tilde U} \longrightarrow (-(1+t)^2, t^2, \sqrt{2}t, 0)$ \\\\
		\stepcounter{subpart3}
		(\alph{subpart3})
		This is differentiation of the components with respect to each coordinate. Since there is only a time dependence, the derivatives with respect to the spatial coordinates are zero.
		\begin{gather}
			U^t_{\;\;,t} = \pdv{U^t}{t} = 2t, \;\; U^x_{\;\;,t} = \pdv{U^x}{t} = 2t, \;\; U^y_{\;\;,t} = \sqrt{2}, \;\; U^z_{\;\;,t} = 0 \\ 
		\end{gather}
		\stepcounter{subpart3}
		(\alph{subpart3})
		Evaluating the expressions:
		\begin{gather}
			U_{\alpha}U^{\alpha}_{\;\;,t} = U_{t}U^{t}_{\;\;,t} + U_{x}U^{x}_{\;\;,t} + U_{y}U^{y}_{\;\;,t} + U_{z}U^{z}_{\;\;,t} = -(1+t^2)\times 2t + 2t^3 + 2t = 0 
		\end{gather}
		\stepcounter{subpart3}
		(\alph{subpart3})
		\begin{gather}
			D^{\beta}_{\;\;,\beta} = D^{t}_{\;\;,t} + D^{x}_{\;\;,x} + D^{y}_{\;\;,y} + D^{z}_{\;\;,z} = 5t 
		\end{gather}
		\stepcounter{subpart3}
		(\alph{subpart3}) \\
		\stepcounter{subpart3}
		(\alph{subpart3}) \\
		\stepcounter{subpart3}
		(\alph{subpart3})
		Performing the required differentiation and using the metric tensor to raise the index:
		\begin{gather}
			\rho_{,t} = 2t, \;\; \rho_{,x} = 2x, \;\; \rho_{,y} = -2y, \;\; \rho_{,z} = 0 \\
			\rho^{,t} = -2t, \;\; \rho^{,x} = 2x, \;\; \rho^{,y} = -2y, \;\; \rho^{,z} = 0
		\end{gather}
		The numbers $\rho^{,\alpha}$ are the components of the vector gradient. \\\\
		\stepcounter{subpart3}
		(\alph{subpart3})
		The expressions are evaluated:
		\begin{gather}
			\nabla_{\vec U }\rho = \rho_{,\gamma}U^{\gamma} = 2t(1+t^2) + 2x(t^2) - 2y(\sqrt{2}t) \\
			\nabla_{\vec U}\vec{D} = D^{\alpha}_{\;\;,\gamma}U^{\gamma} \longrightarrow (t^2, 5x + 5xt^2 + 5t^3, 2t, 0)\\
			\nabla_{\vec D}\rho = \rho_{,\gamma}D^{\gamma} = 2t(x) + 2x(5tx) - 2y(\sqrt{2}t)\\
			\nabla_{\vec D}\vec{U} = U^{\beta}_{\;\;,\gamma}D^{\gamma} \longrightarrow (2tx, 10 t^2 x^2, 2t, 0)
		\end{gather}
	\end{subequations}
\end{chapter3}

\begin{chapter3}\label{prob: 31}
	\begin{subequations}
		\stepcounter{subpart3}
		(\alph{subpart3})
		\stepcounter{subsubpart3}
		(\roman{subsubpart3})
		The following computation yields:
		\begin{gather}
			\vec{U}\cdot\vec{V}_{\perp} = \vec{U}(\tilde{V}_{\perp}) = U_{\alpha} V^{\alpha}_{\perp} = U_{\alpha}\pqty{\eta^{\alpha}_{\;\;\beta} + U^{\alpha} U_{\beta}}V^{\beta} \\
			= U_{\beta} V^{\beta} + U_{\alpha} U^{\alpha} U_{\beta} V^{\beta} = U_{\alpha} V^{\beta} - U_{\alpha} V^{\beta} = 0,
		\end{gather}
		\stepcounter{subsubpart3}
		(\roman{subsubpart3})
		Performing the operation ${\bf P}\vec{V}_{\perp}$:
		\begin{gather}
			V^{\alpha}_{\perp\perp} = P^{\alpha}_{\;\;\beta} P^{\beta}_{\;\;\mu} V^{\mu} = \pqty{\eta^{\alpha}_{\;\;\beta} + U^{\alpha}U_{\beta}}\pqty{\eta^{\beta}_{\;\;\mu} + U^{\beta}U_{\mu}}V^{\mu} = \pqty{\delta^{\alpha}_{\mu} - U^{\alpha} U_{\mu} + 2U^{\alpha} U_{\mu}}V^{\mu} \\ 
			= \pqty{\delta^{\alpha}_{\;\;\mu} + U^{\alpha} U_{\mu}}V^{\mu} = V^{\alpha}_{\perp}	
		\end{gather}
		\stepcounter{subpart3}
		(\alph{subpart3})
		The corresponding vector $\vec{q}_{\perp}$ is:
		\begin{gather}
			q^{\mu}_{\perp} = \pqty{\eta^{\mu}_{\;\;\nu} - \frac{q^{\mu}q_{\nu}}{q^{\alpha} q_{\alpha}}} q^{\nu} \\
			\vec{q}\cdot\vec{q}_{\perp} = q_{\mu} q^{\mu}_{\perp} = q_{\mu}\pqty{\eta^{\mu}_{\;\;\nu} - \frac{q^{\mu}q_{\nu}}{q^{\alpha} q_{\alpha}}} q^{\nu} = 0
		\end{gather}
		If $\vec{q}$ is null, $q_{\alpha} q^{\alpha} = 0$ and therefore the expression is invalid. Thus, {\bf P} is restricted to non-null unit vectors satisfying $U_{\alpha} U^{\alpha} = -1$. \\\\
		\stepcounter{subpart3}
		(\alph{subpart3})
		Feeding the vectors to the tensor:
		\begin{gather}
			{\bf P}(\vec{V}_{\perp},\vec{W}_{\perp}) = P_{\mu\nu} V^{\mu}_{\perp} W^{\nu}_{\perp} = \pqty{\eta_{\mu\nu} + U_{\mu} U_{\nu}}\pqty{\eta^{\mu}_{\;\;\alpha} + U^{\mu} U_{\alpha}}\pqty{\eta^{\nu}_{\;\;\beta} + U^{\nu} U_{\beta}} V^{\alpha} W^{\beta} \\
			= \eta_{\mu\nu} \eta^{\mu}_{\;\;\alpha} \eta^{\nu}_{\;\;\beta} + U_{\mu} U_{\nu} U^{\mu} U_{\alpha} U^{\nu} U_{\beta}
		\end{gather}
	\end{subequations}
\end{chapter3}

\begin{chapter3}\label{prob: 32}
	
\end{chapter3}

\begin{chapter3}\label{prob: 33}
	
\end{chapter3}

\begin{chapter3}\label{prob: 34}
	\begin{subequations}
		\stepcounter{subpart3}
		(\alph{subpart3})
		\begin{gather}
			t = \frac{u + v}{2}, \;\; x = \frac{v - u}{2} \\
			u = 1, \;\; v = 1 \longrightarrow t = 1, \;\; x = 0 \longrightarrow \vec{e}_t = \frac{\vec{e}_u + \vec{e}_v}{2} \\
			u = -1, \;\; v = 1 \longrightarrow t = 0, \;\; x = 1 \longrightarrow \vec{e}_x = \frac{\vec{e}_v - \vec{e}_u}{2} \\
			\vec{e}_u = \frac{\vec{e}_t - \vec{e}_x}{2}, \;\; \vec{e}_v = \frac{\vec{e}_t + \vec{e}_x}{2}		
		\end{gather}
		\begin{center}
			\begin{tikzpicture}
				\draw[help lines, color=gray!90, dashed] (-2,-2) grid (2,2);
				\draw[<->,ultra thick] (-2,0)--(2,0) node[right]{$x$};
				\draw[<->,ultra thick] (0,-2)--(0,2) node[above]{$t$};
				\draw node at (-0.2,-0.2){$O$};
				\draw[-latex, blue, thick] (0,0) -- (1,0) node[above]{$\vec{e}_x$};
				\draw[-latex, blue, thick] (0,0) -- (0,1) node[right]{$\vec{e}_t$};
				\draw[-latex, green, thick] (0,0) -- (-0.5,0.5) node at (-0.7,0.7){$\vec{e}_u$};
				\draw[-latex, green, thick] (0,0) -- (0.5,0.5) node at (0.7,0.7){$\vec{e}_v$};
			\end{tikzpicture}
		\end{center}
		\stepcounter{subpart3}
		(\alph{subpart3})
		The matrix formed by the basis vectors form the identity matrix, with a non-zero determinant. Clearly the set of vectors are linearly independent and span Minkowski space,	thus form a basis. \\\\
		\stepcounter{subpart3}
		(\alph{subpart3})
		By evaluating the dot products of the basis vectors $\vec{e}_{\mu} \cdot \vec{e}_{\nu}$, we find the matrix representing the metric tensor:
		\begin{gather}
			g_{\mu\nu} = {\bf g}\pqty{\vec{e}_{\mu},\vec{e}_{\nu}} =
			\begin{bmatrix}
				0 & -1/2 & 0 & 0 \\
				-1/2 & 0 & 0 & 0 \\
				0 & 0 & 1 & 0 \\
				0 & 0 & 0 & 1			
			\end{bmatrix}		
		\end{gather}
		\stepcounter{subpart3}
		(\alph{subpart3})
		This is evident from the dot product calculations of the previous part. \\\\
		\stepcounter{subpart3}
		(\alph{subpart3})
		The one-forms are simply found by evaluating the differential forms and substituting. ${\bf g}(\vec{e}_u,\;\;)$ are the basis one-forms in the u-v plane, which is simply inverting the sign of the time component of the basis vectors by the metric tensor. 
		\begin{gather}
			\tilde{\dd}u = \tilde{\dd}t - \tilde{\dd}x, \;\; \tilde{\dd}v = \tilde{\dd}x + \tilde{\dd}t	\\
			{\bf g}\pqty{\vec{e}_u, \;\;} = -\frac{\tilde{\dd}v}{2} = - \frac{1}{2}\tilde{\dd}t - \frac{1}{2}\tilde{\dd}x, \;\; {\bf g}\pqty{\vec{e}_u, \;\;} = - \frac{\tilde{\dd}u}{2} = -\frac{1}{2}\tilde{\dd}t + \frac{1}{2}\tilde{\dd}x
		\end{gather}	
	\end{subequations}	
\end{chapter3}

\newtheorem{chapter4}{Problem}
\newcounter{subpart4}[chapter4]
\newcounter{subsubpart4}[chapter4]

\chapter{Perfect Fluids in Special Relativity}

\begin{chapter4}\label{prob: 1}
	\begin{subequations}
		\stepcounter{subpart4}
		(\alph{subpart4})
		No, the objects are too far apart and too unalike. \\\\
		\stepcounter{subpart4}
		(\alph{subpart4})
		Yes, the medium can be considered as homogenous. \\\\
		\stepcounter{subpart4}
		(\alph{subpart4})
		Yes, because the traffic is moving very slowly and is in large numbers close together. \\\\
		\stepcounter{subpart4}
		(\alph{subpart4})
		No, because the traffic motion isn't homogenous. \\\\
		\stepcounter{subpart4}
		(\alph{subpart4})
		Yes, the continuum approximation suits plasma.
	\end{subequations}
\end{chapter4}

\begin{chapter4}\label{prob: 2}
	\begin{subequations}
				
	\end{subequations}	
\end{chapter4}

\begin{chapter4}\label{prob: 3}
	
\end{chapter4}

\begin{chapter4}\label{prob: 4}
	
\end{chapter4}

\begin{chapter4}\label{prob: 5}
	\begin{subequations}
		Testing the definition:
		\begin{gather}
			{\bf T}\pqty{a\tilde{\dd}x^{\alpha} + b\tilde{\dd}x^{\mu}, \tilde{\dd}x^{\beta}} = a{\bf T}\pqty{\tilde{\dd}x^{\alpha}, \tilde{\dd}x^{\beta}} + b{\bf T}\pqty{\tilde{\dd}x^{\mu}, \tilde{\dd}x^{\beta}} = a T^{\alpha\beta} + b T^{\mu\beta}\\
			{\bf T}\pqty{\tilde{\dd}x^{\alpha}, a\tilde{\dd}x^{\beta} + b\tilde{\dd}x^{\mu}} = a{\bf T}\pqty{\tilde{\dd}x^{\alpha}, \tilde{\dd}x^{\beta}} + b{\bf T}\pqty{\tilde{\dd}x^{\alpha}, \tilde{\dd}x^{\mu}} = a T^{\alpha\beta} + b T^{\alpha\mu}
		\end{gather}
	\end{subequations}
\end{chapter4}

\begin{chapter4}\label{prob: 6}
	\begin{subequations}
		The tensor can be expressed in a basis:
		\begin{gather}
			{\bf T} = T^{\alpha\beta} \vec{e}_{\alpha} \otimes \vec{e}_{\beta}
		\end{gather}
		Since $T^{00} = \rho$ and the other components are zero in the MCRF, we can express it as a product of the number-flux four-vector and the four-momentum in the MCRF, because $\vec{N} \stackrel{MCRF}\longrightarrow (n, 0, 0, 0)$ and $\vec{p} \stackrel{MCRF}\longrightarrow (m, 0, 0, 0)$:
		\begin{gather}
			{\bf T} = \vec{p} \otimes \vec{N}
		\end{gather}
	\end{subequations}
\end{chapter4}

\begin{chapter4}\label{prob: 7}
	\begin{subequations}
		This can be deduced by applying a Lorentz transformation from the frame $O$ to $\bar O$:
		\begin{gather}
			T^{\bar \alpha \bar \beta} = \Lambda^{\bar \alpha}_{\;\;\alpha}\Lambda^{\bar \beta}_{\;\;\beta} T^{\alpha\beta} = \rho \Lambda^{\bar \alpha}_{\;\;\alpha}\Lambda^{\bar \beta}_{\;\;\beta} U^{\alpha} U^{\beta} \longrightarrow \pqty{\bar T} = \rho\pqty{\Lambda^T}\pqty{\vec U \otimes \vec U}\pqty{\Lambda}
		\end{gather}
	\end{subequations}
\end{chapter4}

\begin{chapter4}\label{prob: 8}
	\begin{subequations}
		
	\end{subequations}
\end{chapter4}

\begin{chapter4}\label{prob: 9}
	\begin{subequations}
		We must investigate $T^{i\beta}_{\;\;\;,\beta} =0, \;\; i = 1,2,3$. The values of $i$ correspond to the momentum density, and thus this expression indicates the divergence of the momentum density across the space-time coordinates.
	\end{subequations}
\end{chapter4}

\begin{chapter4}\label{prob: 10}
	
\end{chapter4}

\begin{chapter4}\label{prob: 11}
	
\end{chapter4}

\begin{chapter4}\label{prob: 12}
	
\end{chapter4}

\begin{chapter4}\label{prob: 13}
	
\end{chapter4}

\begin{chapter4}\label{prob: 14}
	
\end{chapter4}

\begin{chapter4}\label{prob: 15}
	
\end{chapter4}

\begin{chapter4}\label{prob: 16}
	\begin{subequations}
		The three-velocity derivatives correspond to the local and convective acceleration, which may be non-zero in the MCRF. This is similar to the argument: $\dd(\sin x) \neq 0$ when $\sin x = 0$.
	\end{subequations}
\end{chapter4}

\begin{chapter4}\label{prob: 17}
	\begin{subequations}
		For the four-vector $\vec U \stackrel{O}\longrightarrow (U^t, U^x, U^y, U^z)$, evaluating the expression in the frame $O$ that is not the MCRF yields:
		\begin{gather}
			a^{\mu} = U^{\mu}_{\;\;,\beta}U^{\beta} = \pdv{U^{\mu}}{x^{\beta}}U^{\beta} \\
			a^{i} = U^{i}_{\;\;,\beta}U^{\beta} = U^{i}_{\;\;,t}U^t + U^i_{\;\;,j}U^{j}
		\end{gather}
		In the small-velocity approximation, the four-vector $\vec U \stackrel{O}\longrightarrow (1,v^x,v^y,v^z)$ is a reduction to the three-velocity ${\bf v}$. Therefore, $U^i_{\;\;,t}$ just denotes the time derivatives of the spatial velocity components, which is simply the three-acceleration $\dot{v}^i$. The rest of the terms describe the convective derivative $U^i_{\;\;,j}U^{j} = \pqty{{\bf v \cdot \nabla}}v^i$. The final expression is:
		\begin{gather}
			a^i = \dot{v}^i + \pqty{{\bf v \cdot \nabla}}v^i = \frac{Dv^i}{Dt}
		\end{gather}
	\end{subequations}
	Which is just the Euler equation from fluid mechanics.
\end{chapter4}

\begin{chapter4}\label{prob: 18}
	
\end{chapter4}

\begin{chapter4}\label{prob: 19}
	\begin{subequations}
		The difference between the coordinates $\bqty{V^t(t_2) - V^t(t_1)}$ is simply the differential change in time with respect to the time component of the four-vector multiplied by the time difference, expressible as $(\partial V^t/\partial t)\dd{t}$. Analogous arguments for the spatial displacements $\bqty{V^x(x_2) - V^x(x_1)}$ give the following expression:
		\begin{gather}
			\int \pdv{V^t}{t} \dd{t}\dd{x}\dd{y}\dd{z} + \int \pdv{V^x}{x}\dd{x}\dd{t}\dd{y}\dd{z} + ... \\
			= \int \pdv{V^{\alpha}}{x^{\alpha}} \dd^4{x} = \int V^{\alpha}_{\;\;,\alpha} \dd^4{x}
		\end{gather}
		This is simply the divergence of the vector through the four-volume, verifying Gauss' theorem.
	\end{subequations}
\end{chapter4}

\begin{chapter4}\label{prob: 20}
	
\end{chapter4}

\begin{chapter4}\label{prob: 21}
	
\end{chapter4}

\begin{chapter4}\label{prob: 22}
	
\end{chapter4}

\begin{chapter4}\label{prob: 23}
	
\end{chapter4}

\begin{chapter4}\label{prob: 24}
	
\end{chapter4}

\begin{chapter4}\label{prob: 25}
	\begin{subequations}
		\stepcounter{subpart4}
		(\alph{subpart4})
		Using the antisymmetry, the components of the electromagnetic field tensor $F^{\mu\nu}$	can be represented as the matrix:
		\begin{gather}
			F^{\mu\nu} =
			\begin{bmatrix}
				0 & E^x & E^y & E^z \\
				-E^x & 0 & B^z & -B^y \\
				-E^y & -B^z & 0 & B^x \\
				-E^z & B^y & -B^x & 0
			\end{bmatrix}
		\end{gather}
		\stepcounter{subpart4}
		(\alph{subpart4})
		The transformation follows $\pqty{\bar F} = \pqty{\Lambda^T}\pqty{F}\pqty{\Lambda}$:
		\begin{gather}
			F^{\alpha'\beta'} =
			\begin{bmatrix}
				1 & 0 & 0 & 0 \\
				0 & \cos\theta & \sin\theta & 0 \\
				0 & -\sin\theta & \cos\theta & 0 \\
				0 & 0 & 0 & 1
			\end{bmatrix}
			\begin{bmatrix}
				0 & E^x & E^y & E^z \\
				-E^x & 0 & B^z & -B^y \\
				-E^y & -B^z & 0 & B^x \\
				-E^z & B^y & -B^x & 0
			\end{bmatrix}
			\begin{bmatrix}
				1 & 0 & 0 & 0 \\
				0 & \cos\theta & -\sin\theta & 0 \\
				0 & \sin\theta & \cos\theta & 0 \\
				0 & 0 & 0 & 1
			\end{bmatrix} \\
			=
			\begin{bmatrix}
				1 & 0 & 0 & 0 \\
				0 & \cos\theta & \sin\theta & 0 \\
				0 & -\sin\theta & \cos\theta & 0 \\
				0 & 0 & 0 & 1
			\end{bmatrix}
			\begin{bmatrix}
				0 & E^x\cos\theta + E^y\sin\theta & -E^x\sin\theta + E^y\cos\theta & E^z \\
				-E^x & B^z\sin\theta  & B^z\cos\theta & -B^y \\
				-E^y & -B^z\cos\theta & B^z\sin\theta & B^x \\
				-E^z & B^y\cos\theta - B^x\sin\theta & -(B^y\sin\theta + B^x\cos\theta) & 0
			\end{bmatrix} \\
			=
			\begin{bmatrix}
				0 & E^x\cos\theta + E^y\sin\theta & E^y\sin\theta - E^x\sin\theta & E^z \\
				-(E^x\cos\theta + E^y\sin\theta) & 0 & B^z & B^x\sin\theta-B^y\cos\theta \\
				-(E^y\cos\theta - E^x\sin\theta) & -B^z & 0 & B^y\sin\theta + B^x\cos\theta \\
				-E^z & -(B^x\sin\theta-B^y\cos\theta) & -(B^y\sin\theta + B^x\cos\theta) & 0
			\end{bmatrix}
		\end{gather}
		\stepcounter{subpart4}
		(\alph{subpart4})
		The components are evaluated as follows:
		\begin{gather}
			\div{{\bf E}} = 4\pi\rho \longrightarrow E^i_{\;\;,i} = 4\pi J^t = \bqty{F^{ti}_{\;\;\;,i}} \qq{and} \bqty{F^{tt}_{\;\;\;,t}} = 0 \longrightarrow F^{t\nu}_{\;\;\;,\nu} = 4\pi J^t \\
			(\curl{{\bf B}})_x - \pdv{E^x}{t} = \pqty{\pdv{B^z}{y} - \pdv{B^y}{z}} - \pdv{E^x}{t} = 4\pi J^x = \bqty{F^{x\nu}_{\;\;\;,\nu}}
		\end{gather}
		The curl expression can be simplified using the Levi-Civita tensor and converting the magnetic field vector into a one-form. The final expression can then be obtained via combination:
		\begin{gather}
			4\pi J^t = \bqty{F^{t\nu}_{\;\;\;,\nu}} \\
			\epsilon^{ijk}B_{j,k} - E^i_{\;\;,t} = 4\pi J^i = \bqty{F^{i\nu}_{\;\;\;,\nu}} \\
			\bqty{F^{\mu\nu}_{\;\;\;,\nu} = 4\pi J^{\mu}}
		\end{gather}
		\stepcounter{subpart4}
		(\alph{subpart4})
		Lowering indices of $F^{\mu\nu}$ using the metric tensor inverts the signs of all the components, giving its transpose:
		\begin{gather}
			F_{\mu\nu} = g_{\mu\alpha}g_{\beta\nu} F^{\alpha\beta} =
			\begin{bmatrix}
				0 & -E^x & -E^y & -E^z \\
				E^x & 0 & -B^z & B^y \\
				E^y & B^z & 0 & -B^x \\
				E^z & -B^y & B^x & 0
			\end{bmatrix}
		\end{gather}
		$F_{\mu\nu,\lambda}$ is a $\pmqty{0 \\ 3}$ tensor with 64 components consisting of all the derivatives of the electromagnetic field tensor. Now, evaluating the remaining Maxwell equations:
		\begin{gather}
			\div{{\bf B}} = 0 \longrightarrow B^i_{\;\;,i} = 0 \\
			\epsilon^{ijk} E_{j,k} + B^i_{\;\;,t} = 0
		\end{gather}
	\end{subequations}	
\end{chapter4}

\newtheorem{chapter5}{Problem}
\newcounter{subpart5}[chapter5]
\newcounter{subsubpart5}[chapter5]

\chapter{Preface to Curvature}

\begin{chapter5}\label{prob: 1}
	
\end{chapter5}

\begin{chapter5}\label{prob: 2}
	All objects would experience the same gravitational force in the same direction. This is analogous to free-fall.
\end{chapter5}

\begin{chapter5}\label{prob: 3}
	\begin{subequations}
		\stepcounter{subpart5}
		(\alph{subpart5})
		\begin{gather}
			\begin{vmatrix}
				\pdv{\xi}{x} & \pdv{\xi}{y} \\
				\pdv{\eta}{x} & \pdv{\eta}{y} 
			\end{vmatrix}
			=
			\begin{vmatrix}
				1 & 0 \\
				0 & 0 \\
			\end{vmatrix}
			= 0
		\end{gather}
		\stepcounter{subpart5}
		(\alph{subpart5})
		\stepcounter{subsubpart5}
		(\roman{subsubpart5})
		\begin{gather}
			\mqty|\pdv{\xi}{x} & \pdv{\xi}{y} \\ \pdv{\eta}{x} & \pdv{\eta}{y}| = \mqty|\frac{x}{\sqrt{x^2+y^2}} & \frac{y}{\sqrt{x^2 + y^2}} \\ -\frac{x}{\pqty{x^2 + y^2}} & \frac{y}{\pqty{x^2 + y^2}}| = \frac{\pqty{xy}^2}{\pqty{x^2+y^2}^{3/2}} \neq 0
		\end{gather}
		Fails at the origin. \\\\
		\stepcounter{subsubpart5}
		(\roman{subsubpart5})
		\begin{gather}
			\mqty|\pdv{\xi}{x} & \pdv{\xi}{y} \\ \pdv{\eta}{x} & \pdv{\eta}{y}| = \mqty|1/x & 0 \\ 0 & 1| = \frac{1}{x} \neq 0  
		\end{gather}
		Fails at $x \leq 0$. \\\\
		\stepcounter{subsubpart5}
		(\roman{subsubpart5})
		\begin{gather}
			\mqty|\pdv{\xi}{x} & \pdv{\xi}{y} \\ \pdv{\eta}{x} & \pdv{\eta}{y}| = \mqty|-\frac{x}{\pqty{x^2 + y^2}} & \frac{y}{\pqty{x^2 + y^2}} \\ -\frac{3x}{\pqty{x^2+y^2}^{3/2}} & \frac{3y}{\pqty{x^2+y^2}^{3/2}}| = 0
		\end{gather}
	\end{subequations}
\end{chapter5}

\begin{chapter5}\label{prob: 4}
	\begin{subequations}
		The derivatives are found as:
		\begin{gather}
			\dv{x}{\lambda} = f'(\lambda), \;\; \dv{y}{\lambda} = g'(\lambda)
		\end{gather}
		The slope of this curve, which is tangent at all times, is found by using the differentials:
		\begin{gather}
			\dd{y} = g'(\lambda)\dd{\lambda}, \;\; \dd{x} = f'(\lambda)\dd{\lambda} \longrightarrow \dv{y}{x} = \frac{g'(\lambda)}{f'(\lambda)}
		\end{gather}
		The slope of the vector is:
		\begin{gather}
			\frac{\dd{y}/\dd{\lambda}}{\dd{x}/\dd{\lambda}} = \frac{g'(\lambda)}{f'(\lambda)} = \dv{y}{x} 
		\end{gather}
	\end{subequations}
\end{chapter5}

\begin{chapter5}\label{prob: 5}
	
\end{chapter5}

\begin{chapter5}\label{prob: 6}
	
\end{chapter5}

\begin{chapter5}\label{prob: 7}
	\begin{subequations}
		The transformation matrices are:
		\begin{gather}
			\Lambda^{\alpha'}_{\;\;\beta} = 
			\begin{bmatrix}
				x/\sqrt{x^2 + y^2} & y/\sqrt{x^2 + y^2} \\
				x/\pqty{x^2 + y^2} & -y/{\pqty{x^2 + y^2}}
			\end{bmatrix}
			, \;\; \Lambda^{\mu}_{\;\;\nu'} =
			\begin{bmatrix}
				\cos\theta & -r\sin\theta \\
				\sin\theta & r\cos\theta
			\end{bmatrix}
		\end{gather}
	\end{subequations}
\end{chapter5}

\begin{chapter5}\label{prob: 8}
	\begin{subequations}
		\stepcounter{subpart5}
		(\alph{subpart5})
		The transformation is as follows:
		\begin{gather}
			f = r^2(1+\sin2\theta) \\
			\mqty[V^r \\ V^{\theta}] = \mqty[\cos\theta & \sin\theta \\ -\frac{\sin\theta}{r} & \frac{\cos\theta}{r}]\mqty[(r\cos\theta)^2 + 3r\sin\theta \\ (r\sin\theta)^2 + 3r\cos\theta] = \mqty[r^2\pqty{\cos^3\theta + \sin^3\theta} + 3r\sin2\theta \\ \frac{r}{2}\sin2\theta\pqty{\sin\theta - \cos\theta} + 3\cos2\theta] \\
			\mqty[W^r \\ W^{\theta}] = \mqty[\cos\theta & \sin\theta \\ -\frac{\sin\theta}{r} & \frac{\cos\theta}{r}]\mqty[1 \\ 1] = \mqty[\cos\theta + \sin\theta \\ \frac{1}{r}\pqty{\cos\theta - \sin\theta}]
		\end{gather}
		\stepcounter{subpart5}
		\hspace{2pt}(\alph{subpart5})
		The gradient in Cartesian components is:
		\begin{gather}
			\tilde{\dd}f = \mqty[\pdv{f}{x} & \pdv{f}{y}] = \mqty[2(x+y) & 2(x+y)] 	
		\end{gather}
		\stepcounter{subsubpart5}
		\hspace{2 pt}(\roman{subsubpart5})
		The gradient in polar coordinates by direct calculation is:
		\begin{gather}
			\tilde{\dd}f = \mqty[\pdv{f}{r} & \pdv{f}{\theta}] = \mqty[2r(1+\sin2\theta) & 2r^2\cos2\theta]
		\end{gather}
		\stepcounter{subsubpart5}
		\hspace{2 pt}(\roman{subsubpart5})
		The gradient in polar coordinates by transforming the Cartesian coordinates is:
		\begin{gather}
			\tilde{\dd}f = \mqty[\pdv{f}{r} & \pdv{f}{\theta}] = \mqty[2r(\cos\theta + \sin\theta) & 2r(\cos\theta + \sin\theta)]\mqty[\cos\theta & -r\sin\theta \\ \sin\theta & r\cos\theta] \\
			= \mqty[2r(1+\sin2\theta) & 2r^2\cos2\theta]
		\end{gather}
		\stepcounter{subpart5}
		(\alph{subpart5})
		\stepcounter{subsubpart5}
		(\roman{subsubpart5})
		The one-forms, keeping in mind that the non-diagonal components of the metric tensor are zero, are:
		\begin{gather}
			V_r = g_{rr}V^r = r^2(\cos^3\theta + \sin^3\theta) + 3r\sin2\theta \\
			V_{\theta} = g_{\theta\theta}V^{\theta} = \frac{r^3}{2}\sin2\theta(\sin\theta - \cos\theta) + 3r^2\cos2\theta \\
			\tilde{V} \longrightarrow \pqty{r^2(\cos^3\theta + \sin^3\theta) + 3r\sin2\theta, \frac{r^3}{2}\sin2\theta\pqty{\sin\theta - \cos\theta} + 3r^2\cos2\theta)} \\
			W_r = g_{rr}W^r = \cos\theta + \sin\theta \\
			W_{\theta} = g_{\theta\theta}W^{\theta} = r\pqty{\cos\theta - \sin\theta} \\
			\tilde{W} \longrightarrow \mqty[\cos\theta + \sin\theta & r(\cos\theta - \sin\theta)]
		\end{gather}
		\stepcounter{subsubpart5}
		\hspace{2 pt}(\roman{subsubpart5})
		The one-form components $(V_x,V_y)$ are the same as the vector components $(V^x, V^y)$ in the Cartesian basis. We must use the transformation $\Lambda^{\alpha}_{\;\;\beta'}$ to find the one-forms in the polar basis:
		\begin{gather}
			\mqty[V_r & V_{\theta}] = \mqty[(r\cos\theta)^2 + 3r\sin\theta & (r\sin\theta)^2 + 3r\cos\theta]\mqty[\cos\theta & -r\sin\theta \\ \sin\theta & r\cos\theta] \\
			= \mqty[r^2\pqty{\cos^3\theta + \sin^3\theta} + 3r\sin2\theta & \frac{r^3}{2}\sin2\theta\pqty{\sin\theta - \cos\theta} + 3r^2\cos2\theta]
		\end{gather} 
	\end{subequations}
\end{chapter5}

\begin{chapter5}\label{prob: 9}
	
\end{chapter5}

\begin{chapter5}\label{prob: 10}
	
\end{chapter5}

\begin{chapter5}\label{prob: 11}
	\begin{subequations}
		\stepcounter{subpart5}
		(\alph{subpart5})
		\begin{gather}
			V^1_{\;\;,1} = 2x, \;\; V^1_{\;\;,2} = 3, \;\; V^2_{\;\;,1} = 3, \;\; V^2_{\;\;,2} = 2y \\
			V^{\alpha}_{\;\;,\beta} \longrightarrow \mqty[2x & 3 \\ 3 & 2y]
		\end{gather}
		\stepcounter{subpart5}
		(\alph{subpart5})
		\begin{gather}
			V^{\mu'}_{\;\;\nu'} = \Lambda^{\mu'}_{\;\;\alpha}\Lambda^{\beta}_{\;\;\nu'}V^{\alpha}_{\;\;,\beta} \\
			V^{1'}_{\;\;;1'} = \Lambda^{1'}_{\;\;\alpha}\Lambda^{\beta}_{\;\;,1'}V^{\alpha}_{\;\;,\beta} = 2r\cos^3\theta + 3\cos\theta\sin\theta + 3\cos\theta\sin\theta + 2r\cos^3\theta = 2r\pqty{\cos^3\theta + \sin^3\theta} + 3\sin2\theta \\
			V^{1'}_{\;\;;2'} = \Lambda^{1'}_{\;\;\alpha}\Lambda^{\beta}_{\;\;,2'}V^{\alpha}_{\;\;,\beta} = 
		\end{gather}
		\stepcounter{subpart5}
		(\alph{subpart5})
		\begin{gather}
			V^{\mu'}_{\;\;;nu'} = V^{\mu'}_{\;\;,\nu'} + V^{\alpha}\Gamma^{\mu'}_{\;\;\alpha\nu'} \\
			V^{r}_{\;\;;r} = V^r_{\;\;,r} + V^r\Gamma^{r}_{\;\;rr} + V^{\theta}\Gamma^{r}_{\;\;r\theta} = 2r\pqty{\cos^3\theta + \sin^3\theta} + 3\sin2\theta \\
			V^{r}_{\;\;;\theta} = V^r_{\;\;,\theta} + V^r\Gamma^{r}_{\;\;\theta r} + V^{\theta}\Gamma^{r}_{\;\;\theta\theta} = 
		\end{gather}
		\stepcounter{subpart5}
		(\alph{subpart5})
		\begin{gather}
			V^{\alpha}_{\;\;,\alpha} = 2(x+y)
		\end{gather}
		\stepcounter{subpart5}
		(\alph{subpart5})
		\begin{gather}
			V^{\mu'}_{\;\;;\mu'} = 			
		\end{gather}
	\end{subequations}
\end{chapter5}

\begin{chapter5}\label{prob: 12}
	\begin{subequations}
		\stepcounter{subpart5}
		(\alph{subpart5})
		The one-form derivatives are the same as the vector derivatives. \\\\
		\stepcounter{subpart5}
		(\alph{subpart5})
		The transformation is evaluated as:
		\begin{gather}
			p_{\mu',\nu} = \Lambda^{\alpha}_{\;\;\mu'}\Lambda^{\beta}_{\;\;\nu'}p_{\alpha,\beta} \\
			p_{1',1'} = \Lambda^{\alpha}_{\;\;1'}\Lambda^{\beta}_{\;\;1'}p_{\alpha,\beta} = 2r\pqty{\cos^3\theta + \sin^3\theta} + 3\sin2\theta \\
			p_{1',2'} = \Lambda^{\alpha}_{\;\;1'}\Lambda^{\beta}_{\;\;2'}p_{\alpha,\beta} = -2r^2\cos\theta\sin\theta - 3r\sin^2\theta +
		\end{gather}
	\end{subequations}
\end{chapter5}

\begin{chapter5}\label{prob: 13}
	\begin{subequations}
		\begin{gather}
			p_{\mu';\nu'} = g_{\mu'\alpha'}V^{\alpha'}_{\;\;;\nu'}
		\end{gather}
	\end{subequations}
\end{chapter5}

\begin{chapter5}\label{prob: 14}
	\begin{subequations}
		Evaluating the expressions:
		\begin{gather}
			\nabla_{\beta}A^{\mu\nu} = A^{\mu\nu}_{\;\;\;\;,\beta} + A^{\alpha\nu}\Gamma^{\mu}_{\;\;\alpha\beta} + A^{\mu\alpha}\Gamma^{\nu}_{\;\;\alpha\beta} \\
			\nabla_{r}A^{rr} = 2r \\
			\nabla_{\theta}A^{rr} = -r^2\pqty{\cos\theta + \sin\theta}\\
			\nabla_{r}A^{r\theta} = 2\sin\theta \\
			\nabla_{\theta}A^{r\theta} = r\pqty{1+\cos\theta-\tan\theta} \\
			\nabla_{r}A^{\theta r} = 2\cos\theta \\
			\nabla_{\theta}A^{\theta r} = r\pqty{1 -\sin\theta - \tan\theta} \\
			\nabla_{r}A^{\theta\theta} = \frac{2\tan\theta}{r} \\
			\nabla_{\theta}A^{\theta\theta} = \sec^2\theta + \cos\theta + \sin\theta
		\end{gather}
	\end{subequations}
\end{chapter5}

\begin{chapter5}\label{prob: 15}
	\begin{subequations}
		The only non-zero first covariant derivative is:
		\begin{gather}
			V^{\theta}_{\;\;;\theta} = \frac{1}{r}
		\end{gather}
		This has two second covariant derivative evaluations, of which only the $r$th derivative is non-zero:
		\begin{gather}
			V^{\theta}_{\;\;;\theta;r} = V^{\theta}_{\;\;;\theta,r} + V^{\alpha}_{\;\;;\theta}\Gamma^{\theta}_{\;\;\alpha r} - V^{\theta}_{\;\;;\alpha}\Gamma^{\alpha}_{\;\;\theta r} = -\frac{1}{r^2}
		\end{gather}
	\end{subequations}
\end{chapter5}

\begin{chapter5}\label{prob: 16}
	
\end{chapter5}

\begin{chapter5}\label{prob: 17}
	\begin{subequations}
		For clarity, partial derivatives will be used instead of the lambda notation here. \\
		The transformation of the partial derivatives of the vector components are as follows:
		\begin{gather}
			V^{\bar\beta}_{\;\;,\bar\alpha} = \Lambda^{\nu}_{\;\;\bar\alpha}\pqty{\Lambda^{\bar\beta}_{\;\;\mu}V^{\mu}}_{,\nu} = \pdv{x^{\nu}}{x^{\bar\alpha}}\pdv{}{x^\nu}\pqty{\pdv{x^{\bar\beta}}{x^\mu}V^{\mu}} = \pdv{x^{\nu}}{x^{\bar\alpha}}\bqty{\pdv{x^{\bar\beta}}{x^{\mu}}\pdv{V^{\mu}}{x^{\nu}} + V^{\mu}\pdv[2]{x^{\bar\beta}}{x^\mu}{x^\nu}} \\
			= \Lambda^{\nu}_{\;\;\bar\alpha}\Lambda^{\bar\beta}_{\;\;\mu}V^{\mu}_{\;\;,\nu} + \Lambda^{\nu}_{\;\;\bar\alpha}\Lambda^{\bar\beta}_{\;\;\nu,\mu}V^{\mu}
		\end{gather}
		The second term has a second-order mixed partial derivative, and therefore is indicative (by existence of the additional second term itself) that partial derivatives of vector components do not transform like tensors. For the transformation of the affine connection:
		\begin{gather}
			\pdv{\vec{e}_{\bar\alpha}}{x^{\bar\beta}} = \Lambda^{\beta}_{\;\;\bar\beta}\pdv{}{x^{\beta}}\bqty{\Lambda^{\alpha}_{\;\;\bar\alpha}\vec{e}_{\alpha}} = \pdv{x^{\beta}}{x^{\bar\beta}}\bqty{\pdv{x^{\alpha}}{x^{\bar\alpha}}\pdv{\vec{e}_{\alpha}}{x^{\beta}} + \pdv[2]{x^{\alpha}}{x^{\beta}}{x^{\bar\alpha}}\vec{e}_{\alpha}} \\
			= \pdv{x^{\beta}}{x^{\bar\beta}}\bqty{\pdv{x^{\alpha}}{x^{\bar\alpha}}\Gamma^{\mu}_{\;\;\alpha\beta}\vec{e}_{\mu} + \pdv[2]{x^{\alpha}}{x^{\beta}}{x^{\bar\alpha}}\vec{e}_{\alpha}} = \pdv{x^{\beta}}{x^{\bar\beta}}\bqty{\pdv{x^{\alpha}}{x^{\bar\alpha}}\Gamma^{\mu}_{\;\;\alpha\beta} + \pdv[2]{x^{\mu}}{x^{\beta}}{x^{\bar\alpha}}}\vec{e}_{\mu} \\
			\bqty{\Gamma^{\mu}_{\;\;\bar\alpha\bar\beta}}\vec{e}_{\mu} = \bqty{\Lambda^{\beta}_{\;\;\bar\beta}\Lambda^{\alpha}_{\;\;\bar\alpha}\Gamma^{\mu}_{\;\;\alpha\beta} + \Lambda^{\beta}_{\;\;\bar\beta}\Lambda^{\mu}_{\;\;\bar\alpha,\beta}}\vec{e}_{\mu} \\
			\bqty{\Gamma^{\bar\mu}_{\;\;\bar\alpha\bar\beta}} = \bqty{\Lambda^{\beta}_{\;\;\bar\beta}\Lambda^{\alpha}_{\;\;\bar\alpha}\Lambda^{\bar\mu}_{\;\;\mu}\Gamma^{\mu}_{\;\;\alpha\beta} + \Lambda^{\beta}_{\;\;\bar\beta}\Lambda^{\bar\mu}_{\;\;\mu}\Lambda^{\mu}_{\;\;\bar\alpha,\beta}}
		\end{gather}
		As we can see, the coefficients when applied to the vector do not transform like a tensor.\\
		Now, we shall test the sum of the two expressions, i.e. the covariant derivative:
		\begin{gather}
			V^{\beta}_{\;\;;\alpha} = V^{\beta}_{\;\;,\alpha} + V^{\mu}\Gamma^{\beta}_{\;\;\mu\alpha}
		\end{gather}
	\end{subequations}
\end{chapter5}

\begin{chapter5}\label{prob: 18}
	
\end{chapter5}

\begin{chapter5}\label{prob: 19}
	
\end{chapter5}

\begin{chapter5}\label{prob: 20}
	
\end{chapter5}

\begin{chapter5}\label{prob: 21}
	\begin{subequations}
		\stepcounter{subpart5}
		(\alph{subpart5})
		This is easily found by taking the derivatives and finding the dot product:
		\begin{gather}
			\vec V \rightarrow \pqty{\dv{t}{a}, \dv{x}{a}} = \pqty{a\cosh\lambda, a\sinh\lambda} \\
			\vec U \rightarrow \pqty{\dv{t}{\lambda}, \dv{x}{\lambda}} = \pqty{\sinh\lambda, \cosh\lambda} \\
			\vec V \cdot \vec U = g_{\alpha\beta}V^{\alpha}U^{\beta} = -a\sinh\lambda\cosh\lambda + a\cosh\lambda\sinh\lambda = 0
		\end{gather}
		\stepcounter{subpart5}
		(\alph{subpart5})
		The inverse transformation is:
		\begin{gather}
			\lambda = \arctanh\pqty{\frac{t}{x}}, \;\; a = \sqrt{x^2 - t^2}
		\end{gather}
		Let $\phi$ be a scalar field expressible in both coordinate systems. The transformation from $\tilde{\dd}\phi(x,y)$ to $\tilde{\dd}\phi(\lambda,a)$ is:
		\begin{gather}
			\Lambda^{\alpha}_{\;\;\bar\beta} = \mqty[a\cosh\lambda & \sinh\lambda \\ a\sinh\lambda & \cosh\lambda]
		\end{gather}
		Therefore, we obtain the basis vectors for the new coordinate system and check for orthogonality:
		\begin{gather}
			\vec{e}_{\bar\beta} = \Lambda^{\alpha}_{\;\;\bar\beta}\vec{e}_{\alpha} \\
			\vec{e}_{\lambda} = (a\cosh\lambda)\vec{e}_t + (a\sinh\lambda)\vec{e}_x \\
			\vec{e}_{a} = (\sinh\lambda)\vec{e}_t + (\cosh\lambda)\vec{e}_x \\
			\vec{e}_{\lambda}\cdot\vec{e}_a = - a\cosh\lambda\sinh\lambda + a\sinh\lambda\cosh\lambda = 0
		\end{gather}
		When $\abs{t}=\abs{x}$, $\lambda = \abs{\infty}, \;\; a = 0$, so it approaches asymptotes. 
		Drawing the coordinate system:
		\begin{center}
			\begin{tikzpicture}[%
			    scale=5,%
			    maingrid/.style={draw=gridcolor,very thick},%
			    subgrid/.style={draw=gridcolor,thin},%
			    tlabels/.style={pos=0.88,above,sloped,yshift=-.3ex,gridlabelcolor},%
			    label/.style={%
			        postaction={%
			            decorate,%
			            transform shape,%
			            decoration={%
			                markings,%
			                mark=at position .65 with \node #1;%
			            }%
			        }%
			    },%
			]%
			    \pgfmathdeclarefunction{arcosh}{1}{\pgfmathparse{ln(#1+sqrt(#1+1)*sqrt(#1-1))}}
			    \pgfmathsetmacro{\Xmax}{1.2}
			    \pgfmathsetmacro{\Tmax}{1.2}
			    \pgfmathsetmacro{\g}{1}
			    \newcommand\mylabelstyle\tiny

			    % curves t=constant
			    \foreach \t in {-3,-2.9375,...,3}{%
			        \path[subgrid] (-\Xmax,-{\Xmax*tanh(\g*\t)}) -- (\Xmax,{\Xmax*tanh(\g*\t)});
			    }
			    \foreach \t in {-3,-2.75,...,3}{%
			        \path[maingrid] (-\Xmax,-{\Xmax*tanh(\g*\t)}) -- (\Xmax,{\Xmax*tanh(\g*\t)});
			    }   

			    % curves x=constant
			    \foreach \xx in {0.05,0.1,...,\Xmax}{%
			        \path[subgrid]
			            plot[domain=-{arcosh(\Xmax/\xx)/\g}:{arcosh(\Xmax/\xx)/\g}]
			            ({\xx*cosh(\g*\x)},{\xx*sinh(\g*\x)});  
			    }
			    \foreach \xx in {0.2,0.4,...,1}{%
			        \path[maingrid]
			            plot[domain=-{arcosh(\Xmax/\xx)/\g}:{arcosh(\Xmax/\xx)/\g}]
			            ({\xx*cosh(\g*\x)},{\xx*sinh(\g*\x)});  
			    }

			    % curve labels
			    \foreach \t in {-1,-.5,...,1}{%
			        \path (0,0) -- (\Xmax,{\Xmax*tanh(\g*\t)})
			            node[tlabels] {\mylabelstyle$\lambda=\t$};
			    }
			    \foreach \xx in {0.4,0.6,...,1.01}{%
			        \path[gridlabelcolor,label={[above]{\mylabelstyle $a=\rnd{\xx}$}}]
			            plot[domain=-{arcosh(\Xmax/\xx)/\g}:{arcosh(\Xmax/\xx)/\g}]
			            ({\xx*cosh(\g*\x)},{\xx*sinh(\g*\x)});  
			    }

			    % X-axis, T-axis, and dashed lines t=+/-infty 
			    \draw[thick,-stealth] (-\Xmax,0) -- (\Xmax,0) node[below] {$x$};
			    \draw[thick,-stealth] (0,-\Tmax) -- (0,\Tmax) node[left] {$t$};
			    \draw[dashed] (-\Xmax,-\Tmax) -- (\Xmax,\Tmax) 
			        node[pos=0.37,above,sloped,yshift=-.3ex] {\mylabelstyle$x=0$}
			        node[tlabels,black] {\mylabelstyle$t=\infty$};
			    \draw[dashed] (-\Xmax,\Tmax) -- (\Xmax,-\Tmax)
			        node[tlabels,black] {\mylabelstyle$t=-\infty$};     
			\end{tikzpicture}
		\end{center}
		\stepcounter{subpart5}
		(\alph{subpart5})
		The metric tensor and the Christoffel symbols are:
		\begin{gather}
			\vec{e}_{\bar\alpha}\cdot\vec{e}_{\bar\beta} = g_{\bar\alpha\bar\beta} \mqty[-a^2 & 0 \\ 0 & 1] \\
			\Gamma^{\lambda}_{\;\;\lambda\lambda} = \frac{1}{2}g^{\lambda\lambda}\pqty{g_{\lambda\lambda,\lambda} + g_{\lambda\lambda,\lambda} - g_{\lambda\lambda,\lambda}} = 0 \\
			\Gamma^{a}_{\;\;\lambda\lambda} = \frac{1}{2}g^{aa}\pqty{g_{a\lambda,\lambda} + g_{a\lambda,\lambda} - g_{\lambda\lambda,a}} = a \\
			\Gamma^{\lambda}_{\;\;\lambda a} = \Gamma^{\lambda}_{\;\;a\lambda} = \frac{1}{2}g^{\lambda\lambda}\pqty{g_{\lambda a ,\lambda} + g_{\lambda\lambda,a} - g_{a\lambda,\lambda}}  = -a^3 \\
			\Gamma^{a}_{\;\;\lambda a} = \Gamma^{a}_{\;\;a\lambda} = \frac{1}{2}g^{aa}\pqty{g_{aa,\lambda} + g_{a\lambda,a} - g_{a\lambda,a}} = 0 \\
			\Gamma^{a}_{\;\;aa} = \frac{1}{2}g^{aa}\pqty{g_{aa,a} + g_{aa,a} - g_{aa,a}} = 0
		\end{gather}
	\end{subequations}
\end{chapter5}

\begin{chapter5}\label{prob: 22}
	\begin{subequations}
		Evaluating the expressions and lowering using the metric tensor:
		\begin{gather}
			U^{\alpha}\nabla_{\alpha}V^{\beta} = U^{\alpha}V^{\beta}_{\;\;;\alpha} = U^{\alpha}V^{\beta}_{\;\;,\alpha} + U^{\alpha}V^{\mu}\Gamma^{\beta}_{\;\;\mu\alpha} = W^{\beta} \\
			W_{\beta} = g_{\beta\nu}U^{\alpha}V^{\nu}_{\;\;;\alpha} = g_{\beta\nu}U^{\alpha}V^{\nu}_{\;\;,\alpha} + U^{\alpha}V^{\mu}\Gamma^{\nu}_{\;\;\mu\alpha}g_{\beta\nu} \\
			= \pqty{g_{\beta\mu}U^{\alpha}V^{\mu}_{\;\;,\alpha} + U^{\alpha}V^{\mu}g_{\beta\mu,\alpha}} - g_{\beta\mu,\alpha}\Gamma^{\nu}_{\;\;\beta\alpha}g_{\nu\mu} \\
			= U^{\alpha}\pqty{g_{\beta\mu}V^{\mu}}_{,\alpha} - U^{\alpha}V_{\nu}\Gamma^{\nu}_{\;\;\beta\alpha} = U^{\alpha}V_{\beta,\alpha} - U^{\alpha}V_{\nu}\Gamma^{\nu}_{\;\;\beta\alpha}\\
			= U^{\alpha}V_{\beta;\alpha} = U^{\alpha}\nabla_{\alpha}V_{\beta}
		\end{gather}
	\end{subequations}
\end{chapter5}

\newtheorem{chapter6}{Problem}
\newcounter{subpart6}[chapter6]
\newcounter{subsubpart6}[chapter6]

\chapter{Curved Manifolds}

\begin{chapter6}\label{prob: 1}
	
\end{chapter6}

\begin{chapter6}\label{prob: 2}
	
\end{chapter6}

\begin{chapter6}\label{prob: 3}
	
\end{chapter6}

\begin{chapter6}\label{prob: 4}
	
\end{chapter6}

\begin{chapter6}\label{prob: 5}
	
\end{chapter6}

\begin{chapter6}\label{prob: 6}
	
\end{chapter6}

\begin{chapter6}\label{prob: 7}
	
\end{chapter6}

\begin{chapter6}\label{prob: 8}
	\begin{subequations}
		
	\end{subequations}
\end{chapter6}

\begin{chapter6}\label{prob: 9}
	The determinant of the metric tensor in 2D polar coordinates is positive-definite and is $r^2$. Evaluating the divergence:
		\begin{gather}
			V^{\alpha}_{\;\;;\alpha} = \frac{1}{r}\pqty{rV^{\alpha}}_{,\alpha} = \\
			\frac{1}{r}\bqty{\pqty{rV^r}_{,r} + \pqty{rV^{\theta}}_{,\theta}} = \\
			\frac{1}{r}\pdv{\pqty{rV^r}}{r} + \pdv{V^{\theta}}{\theta}
		\end{gather}
		The determinant of the metric tensor in 3D polar coordinates is positive-definite and is $r^4\sin^2\theta$. Evaluating the divergence:
		\begin{gather}
			V^{\alpha}_{\;\;;\alpha} = \frac{1}{r^2\sin\theta}\pqty{(r^2\sin\theta)V^{\alpha}}_{,\alpha} = \\
			\frac{1}{r^2\sin\theta}\bqty{\pqty{(r^2\sin\theta)V^r}_{,r} + \pqty{(r^2\sin\theta)V^{\theta}}_{,\theta} + \pqty{(r^2\sin\theta)V^{\phi}}_{,\phi}} = \\
			\frac{1}{r^2}\pdv{\pqty{r^2V^r}}{r} + \frac{1}{\sin\theta}\pdv{\pqty{\sin\theta\;V^{\theta}}}{\theta} + \pdv{V^{\phi}}{\phi}
		\end{gather}
\end{chapter6}

\begin{chapter6}\label{prob: 10}
	
\end{chapter6}

\begin{chapter6}\label{prob: 11}
	\begin{subequations}
		\stepcounter{subpart6}
		(\alph{subpart6})
		A globally parallel vector field must satisfy the equation:
		\begin{gather}
			V^{\alpha}_{\;\;,\beta} = -\Gamma^{\alpha}_{\;\;\mu\beta}V^{\mu} 			
		\end{gather}
		Taking the mixed derivatives and equating them:
		\begin{gather}
			V^{\alpha}_{\;\;,\beta\nu} = \pqty{-\Gamma^{\alpha}_{\;\;\mu\beta}V^{\mu}}_{,\nu} = -\Gamma^{\alpha}_{\;\;\mu\beta,\nu}V^{\mu} - \Gamma^{\alpha}_{\;\;\mu\beta}V^{\mu}_{\;\;,\nu} \\
			V^{\alpha}_{\;\;,\nu\beta} = \pqty{-\Gamma^{\alpha}_{\;\;\mu\nu}V^{\mu}}_{,\beta} = -\Gamma^{\alpha}_{\;\;\mu\nu,\beta}V^{\mu} - \Gamma^{\alpha}_{\;\;\mu\nu}V^{\mu}_{\;\;,\beta} \\
			\pqty{\Gamma^{\alpha}_{\;\;\mu\beta,\nu} - \Gamma^{\alpha}_{\;\;\mu\nu,\beta}}V^{\mu} = \Gamma^{\alpha}_{\;\;\mu\beta}V^{\mu}_{\;\;,\nu} - \Gamma^{\alpha}_{\;\;\mu\nu}V^{\mu}_{\;\;,\beta}
		\end{gather}
		Substituting the partial derivatives with the Christoffel symbols (creating a dummy sum over $\sigma$) gives the required result:
		\begin{gather}
			\pqty{\Gamma^{\alpha}_{\;\;\mu\beta,\nu} - \Gamma^{\alpha}_{\;\;\mu\nu,\beta}}V^{\mu} = \pqty{\Gamma^{\alpha}_{\;\;\mu\beta}\Gamma^{\mu}_{\;\;\sigma\nu} - \Gamma^{\alpha}_{\;\;\mu\nu}\Gamma^{\mu}_{\;\;\sigma\beta}}V^{\sigma}
		\end{gather}
		\stepcounter{subpart6}
		(\alph{subpart6})
		Relabeling the indices on the right side ($\sigma \leftrightarrow \mu$):
		\begin{gather}
			\pqty{\Gamma^{\alpha}_{\;\;\mu\beta,\nu} - \Gamma^{\alpha}_{\;\;\mu\nu,\beta}}V^{\mu} = \pqty{\Gamma^{\alpha}_{\;\;\sigma\beta}\Gamma^{\sigma}_{\;\;\mu\nu} - \Gamma^{\alpha}_{\;\;\sigma\nu}\Gamma^{\sigma}_{\;\;\mu\beta}}V^{\mu} \\
			\pqty{\Gamma^{\alpha}_{\;\;\mu\beta,\nu} - \Gamma^{\alpha}_{\;\;\mu\nu,\beta} - \Gamma^{\alpha}_{\;\;\sigma\beta}\Gamma^{\sigma}_{\;\;\mu\nu} + \Gamma^{\alpha}_{\;\;\sigma\nu}\Gamma^{\sigma}_{\;\;\mu\beta}}V^{\mu} = 0
		\end{gather}
		Note that this is the product of the Riemann curvature tensor with the vector $R^{\alpha}_{\;\;\mu\nu\beta}V^{\mu} = 0$. Since the product is zero, we know that the commutator of the covariant derivative operators is zero: $\bqty{\nabla_{\mu},\nabla_{\beta}}V^{\alpha} = 0$. This implies that the covariant derivatives commute for a globally parallel vector field.
	\end{subequations}
\end{chapter6}

\begin{chapter6}\label{prob: 12}
	
\end{chapter6}

\begin{chapter6}\label{prob: 13}
	
\end{chapter6}

\begin{chapter6}\label{prob: 14}
	
\end{chapter6}

\begin{chapter6}\label{prob: 15}
	
\end{chapter6}

\begin{chapter6}\label{prob: 16}
	
\end{chapter6}

\begin{chapter6}\label{prob: 17}
	
\end{chapter6}

\begin{chapter6}\label{prob: 18}
	
\end{chapter6}

\begin{chapter6}\label{prob: 19}
	
\end{chapter6}

\begin{chapter6}\label{prob: 20}
	
\end{chapter6}

\begin{chapter6}\label{prob: 21}
	
\end{chapter6}

\begin{chapter6}\label{prob: 22}
	
\end{chapter6}

\begin{chapter6}\label{prob: 23}
	
\end{chapter6}

\begin{chapter6}\label{prob: 24}
	
\end{chapter6}

\begin{chapter6}\label{prob: 25}
	
\end{chapter6}

\begin{chapter6}\label{prob: 26}
	
\end{chapter6}

\begin{chapter6}\label{prob: 27}
	
\end{chapter6}

\begin{chapter6}\label{prob: 28}
	
\end{chapter6}

\begin{chapter6}\label{prob: 29}
	
\end{chapter6}
\end{document}
