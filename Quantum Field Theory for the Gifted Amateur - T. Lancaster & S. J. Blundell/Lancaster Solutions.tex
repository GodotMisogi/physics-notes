% Lancaster et. al. Solutions Manual

\documentclass{report}
\usepackage[paperwidth=18cm, paperheight=23.6 cm, top = 20mm, bottom = 18mm, left=10mm, right = 10mm]{geometry}
\usepackage{fancyhdr}
\pagestyle{fancy}

\renewcommand{\chaptermark}[1]{%
\markboth{\chaptername
\ \thechapter.\ #1}{}}

\usepackage{graphicx}
\usepackage{amsmath, amsfonts, amssymb, amsthm}
\usepackage{tensor}
\usepackage{physics}
\usepackage{cancel}
\usepackage{hyphenat}
\usepackage{hyperref}
\hypersetup{colorlinks, linkcolor = [RGB]{66, 128, 128}, urlcolor = red, linktocpage = true}
\usepackage{enumitem}

% \usepackage{charter}

\DeclareMathOperator{\arctanh}{arctanh}

\newlist{subquests}{enumerate}{2}
\setlist[subquests, 1]{leftmargin=*, label = \textbf{\arabic*.}}
\setlist[subquests, 2]{leftmargin=*, label = (\alph*)}

\renewcommand{\familydefault}{\sfdefault}

\begin{document}

\title{Solutions to \\ Quantum Field Theory for the Gifted Amateur \\ by Tom Lancaster \& Stephen J. Blundell}

\author{Arjit Seth}

\maketitle


\chapter{Lagrangians}

\begin{subquests}
	\item \emph{Fermat's principle of least time.}
	\begin{gather*}
		t = \frac{\sqrt{x^2 + h_1^2}}{v_1} + \frac{\sqrt{(l-x)^2+h_2^2}}{v_2} = \frac{\sqrt{x^2 + h_1^2}}{c/n_1} + \frac{\sqrt{(l-x)^2+h_2^2}}{c/n_2} \\
		\dv{t}{x} = \frac{x}{c/n_1\sqrt{x^2 + h_1^2}} - \frac{(l-x)}{c/n_2\sqrt{(l-x)^2 + h_2^2}} = 0 \\
		n_1\sin\theta = n_2\sin\phi
	\end{gather*}
	
	\item \emph{Practice with functional derivatives.}
	\begin{subquests}
		\item
		\begin{gather*}
			\fdv{H[f]}{f(z)} = \lim_{\epsilon \to 0} \frac{1}{\epsilon}\bqty{\int G(x,y)[f(y) + \epsilon\delta(y-z)]\dd{y} - \int G(x,y)f(y)\dd{y}} = G(x,z) \\
	 	\end{gather*}

	 	\item
	 	\begin{align*}
	 		\fdv{I[f^{\alpha}]}{x_0}& =  \lim_{\epsilon \to 0} \frac{1}{\epsilon}\bqty{\int \bqty{f(x) + \epsilon\delta(x - x_0)}^{\alpha} - f(x)^{\alpha} \dd{x}} = \alpha\bqty{f(x_0)}^{\alpha-1}\\ 
	 		\therefore \fdv{I[f^3]}{f(x_0)} & =  \lim_{\epsilon \to 0} \frac{1}{\epsilon}\bqty{\int_{-1}^1 \bqty{f(x) + \epsilon\delta(x-x_0)}^3\dd{x} -\int_{-1}^1 [f(x)]^3 \dd{x}} = 3\bqty{f(x_0)}^2 \\
			% \frac{\delta^2 I[f^3]}{\delta f(x_0)\delta f(x_1)} = \lim_{\epsilon \to 0} \frac{3}{\epsilon}\bqty{\bqty{f(x_0) + \epsilon\delta(x_0 - x_1)}^2 - f^2(x_0)} =  6f(x_1) \\
	 	\end{align*}

	 	\item
	 	\begin{align*}
	 		\fdv{J[f]}{f(x)} & = \lim_{\epsilon \to 0} \frac{1}{\epsilon}\bqty{\int \pqty{\pdv{y}[f(y) + \epsilon\delta(y-x)]}^2 \dd{y} - \int \pqty{\pdv{f}{y}}^2 \dd{y}} \\
			& = \lim_{\epsilon \to 0} \frac{1}{\epsilon}\bqty{\int \bqty{\pdv{f}{y} + \epsilon\delta'(y-x)}^2 \dd{y} - \int \pqty{\pdv{f}{y}}^2 \dd{y}} =	2\pdv[2]{f}{x}
	 	\end{align*}
	\end{subquests}
	

	\item \emph{Euler-Lagrange equations using functional derivatives and more.}
	\begin{gather*}
		\fdv{G[f]}{f(x)} = \lim_{\epsilon \to 0} \frac{1}{\epsilon}\bqty{\int \bqty{g(y,f) + \pdv{g(y,f)}{f}\epsilon\delta(y-x)}\dd{y} - \int g(y,f)\dd{y}} = \pdv{g(x,f)}{f(x)}
 	\end{gather*}

 	\item \emph{Results on Dirac Delta functions.}
 	\begin{align*}
		\fdv{\phi(x)}{\phi(y)} & = \lim_{\epsilon \to 0} \frac{\phi(x) + \epsilon\delta(x-y) - \phi(x)}{\epsilon} = \delta(x-y) \\
		\fdv{\dot{\phi}(t)}{\phi(t_0)} & = \lim_{\epsilon\to 0} \frac{1}{\epsilon}\bqty{\dv{t}[\phi(t) + \epsilon\delta(t - t_0)] - \dot{\phi}(t)} = \dv{t}\delta(t-t_0) 
	\end{align*}

	\item \emph{Derivation of the wave equation.} 
	\begin{gather*}
		S = \int (T - V) \dd{t} = \frac{1}{2}\int \rho\pqty{\pdv{\psi}{t}}^2 - \mathcal T(\nabla\psi)^2 \dd{t} \\
		\fdv{S}{\psi} = \lim_{\epsilon \to 0}\frac{1}{2\epsilon}\left[\int\rho\pqty{\pdv{t}[\psi + \epsilon\delta(t-t_0)]}^2 - \mathcal T\pqty{\nabla[\psi + \epsilon\delta({\vb x} - {\vb y})]}^2 \;\dd{t} \right.	\left. - \int \rho\pqty{\pdv{\psi}{t}}^2 - \mathcal T(\nabla\psi)^2\;\dd{t} \right] \\
		= \int\bqty{\rho\pdv{t}\delta(t-t_0)\pdv{\psi}{t} - \mathcal T\nabla\delta({\vb x}-{\vb y})\nabla\psi}\;\dd{t} = 0 \\
		\implies \nabla^2 \psi = \frac{1}{v^2}\pdv[2]{\psi}{t},\quad v = \sqrt{\frac{\cal{T}}{\rho}}    
	\end{gather*}

	\item \emph{Functional derivative of a Wick expansion term in the generating functional.}
	\begin{gather*}
		Z_0[J] = \exp\pqty{-\frac{1}{2}\int\dd^4{x}\;\dd^4{y}\;J(x)\Delta(x-y)J(y)} \\
		\fdv{Z_0[J]}{J(z_1)} = \lim_{\epsilon \to 0} \frac{1}{\epsilon}\left[\exp\pqty{-\frac{1}{2}\int\dd^4{x}\;\dd^4{y}\;\bqty{J(x) + \epsilon \delta(x - z_1)}\Delta(x-y)\bqty{J(y) + \epsilon \delta(y - z_1)}} \right. \\
		\left. - \exp\pqty{-\frac{1}{2}\int\dd^4{x}\;\dd^4{y}\;J(x)\Delta(x-y)J(y)}\right]
	\end{gather*}
\end{subquests}
	

\chapter{Simple harmonic oscillators}

\begin{subquests}
	\item \emph{Commutators of ladder operators.}
	\begin{gather*}
		[\hat{a}, \hat{a}^{\dagger}] = \frac{m\omega}{2\hbar}\pqty{\hat{x} + \frac{i}{m\omega}\hat{p}}\pqty{\hat{x} - \frac{i}{m\omega}\hat{p}} - \frac{m\omega}{2\hbar}\pqty{\hat{x} - \frac{i}{m\omega}\hat{p}}\pqty{\hat{x} + \frac{i}{m\omega}\hat{p}} \\ 
		= \frac{1}{2i\hbar}\pqty{[\hat{x},\hat{p}] + [\hat{x},\hat{p}]} = 1
	\end{gather*}

	\item \emph{Perturbation theory and ladder operators.} The perturbative term $\hat H_p = \lambda \hat W = \lambda \hat{x}^4$. Its first-order correction is:
	\begin{gather*}
		E_n = E_n^{(0)} + \mel{\phi_n}{\hat H_p}{\phi_n} = \pqty{n + \frac{1}{2}}\hbar \omega + \mel**{n}{\lambda \hat{x}^4}{n} \\
		= \pqty{n + \frac{1}{2}}\hbar \omega + \lambda\pqty{\frac{\hbar}{2m\omega}}^2\mel**{n}{\pqty{\hat{a} + \hat{a}^{\dagger}}^4}{n} \\
		= \pqty{n + \frac{1}{2}}\hbar \omega + \lambda\pqty{\frac{\hbar}{2m\omega}}^2\mel**{n}{painful}{n}
	\end{gather*}

	\item \emph{Fourier transform of $\hat{x}_k$.} 
	\begin{gather*}
		\hat{x}_j = \frac{1}{\sqrt{N}}\sum_{k}\hat{\tilde{x}}_k e^{ikja}, \quad \hat{\tilde x}_k = \sqrt{\frac{\hbar}{2m\omega_k}}\pqty{\hat{a}_k + \hat{a}_{-k}^{\dagger}} \\
		\hat{x}_j = \frac{1}{\sqrt{N}}\sum_{k} \sqrt{\frac{\hbar}{2m\omega_k}}\pqty{\hat{a}_k + \hat{a}_{-k}^{\dagger}}e^{ikja} = \sqrt{\frac{h}{Nm}}\sum_{k}\frac{1}{\sqrt{2\omega_k}}\pqty{\hat{a}_k e^{ikja}+ \hat{a}_{k}^{\dagger}e^{-ikja}}
	\end{gather*}

	\item \emph{Ground state of the harmonic oscillator.}
	\begin{gather*}
		\sqrt{\frac{m\omega}{2\hbar}}\pqty{\hat{x} + \frac{i}{m\omega}\hat{p}} \ket{0}= 0 \\
		\mel{x}{\hat{x}}{0} + \frac{i}{m\omega}\mel{x}{\hat{p}}{0} = 0 \\
		\pqty{x + \frac{\hbar}{m\omega}\dv{x}}\ip{x}{0} = 0 \\
		\pqty{\dv{x} + \frac{m\omega}{\hbar}x}\ip{x}{0} = 0 
	\end{gather*}
	This is easily solved by separation of variables. Attempting a series solution for practice:
	\begin{gather*}
		\ip{x}{0} = \sum_{n=0}^{\infty} a_n x^n \\
		\sum_{n=0}^{\infty}na_{n} x^{n-1} + \sum_{n=0}^{\infty}\frac{m\omega}{\hbar}a_{n}x^{n+1} = 0 \\
		\sum_{n=0}^{\infty}(n+2)a_{n+2} x^{n+1} + \sum_{n=0}^{\infty}\frac{m\omega}{\hbar}a_n x^{n+1} = 0 \\
		a_{n+2} = -\frac{m\omega}{\hbar(n+2)}a_{n},\quad a_0 = A,\, a_1 = 0 \\
		\ip{x}{0} = A\bqty{1 + \pqty{-\frac{m\omega}{2\hbar}}x^2 + \frac{1}{2}\pqty{-\frac{m\omega}{2\hbar}}^2 x^4 + \frac{1}{6}\pqty{-\frac{m\omega}{2\hbar}}^3 x^6 + ...} \\
		\ip{x}{0} = A\exp(-\frac{m\omega x^2}{2\hbar}) \\
		A = 1/\left|\exp(-\frac{m\omega x^2}{2\hbar})\right| \\
		A = 1/\sqrt{\int_{-\infty}^{\infty}\exp(2\frac{m\omega}{2\hbar}x^2)} = \pqty{\frac{m\omega}{\pi \hbar}}^{1/4}\\
		\ip{x}{0} = \pqty{\frac{m\omega}{\pi \hbar}}^{1/4}\exp(-\frac{m\omega x^2}{2\hbar})
	\end{gather*}

\end{subquests}


\chapter{Occupation number representation}

\begin{subquests}
	\item \emph{Practice with exponentials and ladder operators.}
	\begin{gather*}
		\frac{1}{\cal{V}}\sum_{\vb{pq}} e^{i\pqty{\vb p \cdot\vb x - \vb q \cdot \vb y}}\bqty{\hat{a}_{\vb p}, \hat{a}_{\vb q}^{\dagger}} = \frac{1}{\cal{V}}\sum_{\vb{pq}} e^{i\pqty{\vb p \cdot \vb x - \vb q \cdot \vb y}}\delta_{\vb{pq}} = \frac{1}{\cal{V}}\sum_{\vb p}e^{i \vb p \cdot\pqty{\vb x-\vb y}} = \delta^{(3)}(\vb x - \vb y)
	\end{gather*}

	\item \emph{Ladder operator identities.}
	\begin{subquests}
		\item
		\begin{gather*}
			\bqty{\hat{a},\pqty{\hat{a}^{\dagger}}^n} = \bqty{\hat{a}\pqty{\hat{a}^{\dagger}}^n - \pqty{\hat{a}^{\dagger}}^n\hat{a}} \\
			= \bqty{(1+\hat{a}^{\dagger}\hat{a})\pqty{\hat{a}^{\dagger}}^{n-1} - \pqty{\hat{a}^{\dagger}}^{n-1}\pqty{\hat{a}^{\dagger}\hat{a}}} = \bqty{\pqty{\hat{a}^{\dagger}}^{n-1} - \bqty{\hat{a}^{\dagger}\hat{a},\pqty{\hat{a}^{\dagger}}^{n-1}}}		
		\end{gather*}

		\item
		\begin{gather*}
			\mel{0}{\hat{a}^n \pqty{\hat{a}^{\dagger}}^m}{0} = \sqrt{n!}\sqrt{m!}\braket{n}{m}
		\end{gather*}

		\item

		\item
	\end{subquests}
	
	\item \emph{Three-dimensional harmonic oscillator.}
	\begin{gather*}
		\hat{a}_i^{\dagger} = \sqrt{\frac{m\omega}{2\hbar}}\pqty{\hat{x}_i - \frac{i}{m\omega}\hat{p}_i} \\
		\bqty{\hat{a}_i,\hat{a}_j^{\dagger}} = \frac{m\omega}{2\hbar}\bqty{\pqty{\hat{x}_i + \frac{i}{m\omega}\hat{p}_i}\pqty{\hat{x}_j - \frac{i}{m\omega}\hat{p}_j} - \pqty{\hat{x}_j - \frac{i}{m\omega}\hat{p}_j}\pqty{\hat{x}_i + \frac{i}{m\omega}\hat{p}_i}} \\
		= \frac{m\omega}{2\hbar}\pqty{\bqty{\hat{x}_i,\hat{x}_j} + \frac{1}{m^2\omega^2}\bqty{\hat{p}_i,\hat{p}_j} - \frac{i}{m\omega}\pqty{\bqty{\hat{x}_j.\hat{p}_i} + \bqty{\hat{x}_i,\hat{p}_j}}} = \delta_{ij} \\
		\hat{a}_i^{\dagger}\hat{a}_i = \frac{\hat{p}_i^2}{2m\hbar\omega} + \frac{1}{2\hbar\omega}m\omega^2\hat{x}_i^2 + \frac{i}{2\hbar}\bqty{\hat{x}_i, \hat{p}_i } = \frac{1}{\hbar\omega}\bqty{\frac{\hat{p}_i^2}{2m} + \frac{1}{2}m\omega^2\hat{x}_i^2 - \frac{\hbar\omega}{2}} \\
		\hat{H} = \sum_{i=1}^3\frac{\hat{p}_i^2}{2m} + \frac{1}{2}m\omega^2\hat{x}_i^2 = \hbar\omega \sum_{i=1}^3 \pqty{\hat{a}_i^{\dagger} \hat{a}_i + \frac{1}{2}} \\
		\hat{L}^i \equiv -i\hbar\epsilon^{ijk}\hat{a}^{\dagger}_j\hat{a}_k		
	\end{gather*}

	\item \emph{Slater determinant for fermions.} Consider an $n$-particle state:
	\begin{gather*}
		\braket{\vb{p}'_1 \vb{p}'_2 \vb{p}'_3\dots \vb{p}'_n}{\vb{p}_n \vb{p}_{n-1} \vb{p}_{n-2} \dots \vb{p}_1} = \mel{0}{\hat{a}_{\vb p'_1}\hat{a}_{\vb p'_2} \hat{a}_{\vb p'_3} \dots \hat{a}_{\vb p_n} \hat{a}^{\dagger}_{\vb p_n} \hat{a}^{\dagger}_{\vb p_{n-1}} \hat{a}^{\dagger}_{\vb p_{n-2}}\dots \hat{a}^{\dagger}_{\vb p_1}}{0} 
	\end{gather*}
\end{subquests}


\chapter{Making second quantization work}

\begin{subquests}
	\item \emph{Commutation relations of density field operators.}
	\begin{gather*}
		\bqty{\hat{\psi}\pqty{\vb x}, \hat{\psi}^{\dagger}\pqty{\vb y}}_{\zeta} = \delta^{\pqty{3}}\pqty{\vb x - y},\quad \bqty{\hat{\psi}\pqty{\vb x}, \hat{\psi}\pqty{\vb y}}_{\zeta} = 0 \\
		\hat{\rho}\pqty{\vb x}\hat{\rho}\pqty{\vb y} = \hat{\psi}^{\dagger}\pqty{\vb x}\hat{\psi}\pqty{\vb x}\hat{\psi}^{\dagger}\pqty{\vb y}\hat{\psi}\pqty{\vb y} \\
		= -\zeta\hat{\psi}^{\dagger}\pqty{\vb x}\hat{\psi}^{\dagger}\pqty{\vb y}\hat{\psi}\pqty{\vb x}\hat{\psi}\pqty{\vb y} + \delta^{\pqty{3}}\pqty{\vb x - y}\hat{\psi}^{\dagger}\pqty{\vb x}\hat{\psi}\pqty{\vb y} \\
		= -\zeta^2\hat{\psi}^{\dagger}\pqty{\vb x}\hat{\psi}^{\dagger}\pqty{\vb y}\hat{\psi}\pqty{\vb y}\hat{\psi}\pqty{\vb x} + \delta^{\pqty{3}}\pqty{\vb x - y}\hat{\psi}^{\dagger}\pqty{\vb x}\hat{\psi}\pqty{\vb y}
	\end{gather*}
	So $\zeta = \pm 1$ yields the same result regardless of bosons or fermions.

	\item \emph{Single-particle density matrix in terms of ladder operators.}
	\begin{gather*}
		\hat{\rho}_1\pqty{\vb x - \vb y} = \expval{\hat{\psi}^{\dagger}\pqty{\vb x}\hat{\psi}\pqty{\vb y}} \\
		= \frac{1}{\cal{V}}\sum_{\vb p}\hat{a}_{\vb p}^{\dagger}e^{-i\vb p\cdot \vb x}\sum_{\vb q}\hat{a}_{\vb q}e^{i\vb q\cdot \vb y} = \frac{1}{\cal{V}}\sum_{\vb{pq}}e^{-i\pqty{\vb p\cdot \vb x - \vb q\cdot \vb y}}\expval{\hat{a}_{\vb p}^{\dagger}\hat{a}_{\vb q}}
	\end{gather*}

	\item \emph{Hubble Hamiltonian.} Solving its eigenvalue problem:
	\begin{gather*}
		|\hat{H} - \lambda\hat{I}| = \mqty|
			U - \lambda & -t & -t & 0 \\
			-t & -\lambda & 0 & -t \\
			-t & 0 & -\lambda & -t \\
			0 & -t & -t & U - \lambda
		| = 0 \\
		\lambda_1 = 0,\;\lambda_2 = U,\;\lambda_{3,4} = \frac{1}{2}\bqty{U \pm \sqrt{16t^2 + U^2}} \\
		\nu_1 = \mqty[0 \\ -1 \\ 1 \\ 0],\; \nu_2 = \mqty[-1 \\ 0 \\ 0 \\ 1],\; \nu_{3,4} = \mqty[1\\ \frac{-U \pm \sqrt{16t^2 + U^2}}{4t} \\ \frac{-U \pm \sqrt{16t^2 + U^2}}{4t} \\ 0]
	\end{gather*}
\end{subquests}

\chapter{Continuous systems}

\begin{subquests}
	\item \emph{Explicit time dependence of Lagrangian and Hamiltonian.}
	\begin{align*}
		\dv{L}{t} & = \pdv{L}{t} + \pdv{L}{q}\dot{q} + \pdv{L}{\dot{q}}\ddot{q} = \pdv{L}{t} + \dv{t}\pqty{\pdv{L}{\dot{q}}}\dot{q} + \pdv{L}{\dot{q}}\ddot{q} \\
		\dv{L}{t} & = \pdv{L}{t} + \dv{t}\pqty{\pdv{L}{\dot{q}}\dot{q}}
	\end{align*}
	The Hamiltonian is defined as the Legendre transformation with a canonical momentum $p = \partial{L}/\partial{\dot q}$. Therefore: 
	\begin{gather*}
		\pdv{L}{t} = \dv{\pqty{L - p\dot{q}}}{t} = -\dv{H}{t}  
	\end{gather*}
	\item \emph{Commutation relations of Poisson brackets.}
	\begin{gather*}
		\qty{A,B}_{PB} = \sum_i \pdv{A}{q_i}\pdv{B}{p_i} - \pdv{A}{p_i}\pdv{B}{q_i} \\
		\qty{B,A}_{PB} = \sum_i \pdv{B}{q_i}\pdv{A}{p_i} - \pdv{B}{p_i}\pdv{A}{q_i} = -\qty{A,B}_{PB}
	\end{gather*}
	\item \emph{Commutation relations of Hermitian operators.} Since A and B are Hermitian, $A = A^{\dagger}, B = B^{\dagger}$.
	\begin{gather*}
		\bqty{A,B}^{\dagger} = \pqty{AB - BA}^{\dagger} = B^{\dagger}A^{\dagger} - A^{\dagger}B^{\dagger} = -\bqty{B,A}
	\end{gather*}
	\item \emph{Investigating the non-relativistic limit of the relativistic free particle.} This is easily found by Taylor expansions of $\gamma$, then taking the low-velocity limit:
	\begin{align*}
		L & = -mc^2 \sqrt{1 - \frac{v^2}{c^2}} \approx -mc^2 + \frac{1}{2}mv^2 \\
		p & = \pdv{L}{v} = \gamma mv \approx mv \\
		H & = pv - L = \gamma mv^2 + \frac{mc^2}{\gamma} = \frac{1}{\gamma}\bqty{\pqty{1 - \frac{v^2}{c^2}}mv^2 + mc^2} = \gamma mc^2 \approx mc^2 + \frac{1}{2}mv^2	
	\end{align*}
	\item \emph{Extremisation of the spacetime interval.}
	\begin{gather*}
		\int_a^b \dd{s} = \int_a^b \sqrt{1-\frac{{\vb v}^2}{c^2}} \dd{t} = \int_a^b \frac{\dd{t}}{\gamma} = \int_a^b L \dd{t} \\
		\pdv{L^2}{\vb v} = \frac{2\vb v}{c^2} \\
		\dv{t}\pqty{\pdv{L^2}{\vb v}} - \pdv{L^2}{\vb x} = \frac{2\dot{\vb v}}{c^2} = 0
	\end{gather*}
	Since the acceleration is zero, the velocity is constant. Hence a straight world-line path does minimise the interval.

	\item \emph{Electromagnetic Lagrangian.}
	\begin{gather*}
		L = \frac{-mc^2}{\gamma} + q{\vb A}\cdot{\vb v} - qV \\
		\nabla{L} = q\bqty{\nabla\pqty{{\vb A}\cdot {\vb v}} - \nabla V} \\
		= q\bqty{\cancel{\vb(A \cdot \nabla)\vb v} + \cancel{(\vb v \cdot \nabla)\vb A} + {\vb v \times (\nabla \times \vb A)} + \cancel{\vb A \times (\nabla \times \vb v)} - q\nabla V} \\
		= q\bqty{\vb E + \vb v \times B},\;\;\because {\vb E} = -q\nabla V, \; \vb B = \nabla \times \vb A\\
		\pdv{L}{\vb v} = -\frac{mc^2}{2\sqrt{1- \frac{{\vb v}^2}{c^2}}}\pqty{-\frac{2\vb v}{c^2}} = \gamma m {\vb v} \\
		\dv{t}\pdv{L}{\vb v} = \nabla L \longrightarrow \dv{t}(\gamma m \vb v) = q\bqty{\vb E + \vb v \times \vb B}
	\end{gather*}

	\item \emph{Non-relativistic limit of the electromagnetic Lagrangian.}
	\begin{align*}
		L & = \frac{-mc^2}{\gamma} + q{\vb A}\cdot{\vb v} - qV \approx \frac{1}{2}m{\vb v}^2 + q{\vb A}\cdot{\vb v} - qV \\
		{\vb p} & = \pdv{L}{\vb v} = m{\vb v} + q{\vb A}
	\end{align*}
	Finding the Hamiltonian is equivalent to finding the energy in terms of momentum:
	\begin{align*}
		H & = {\vb p\cdot \vb v} - L = m{\vb v}^2 + q{\vb A \cdot \vb v} - L \\ 
		& = mc^2 + \frac{1}{2}m{\vb v}^2 + qV = mc^2 + \frac{1}{2m}({\vb p} - q{\vb A})^2 + qV, \quad {\vb v} = \frac{{\vb p} - q{\vb A}}{m}
	\end{align*}
	Adjusting the zero of the Hamiltonian by subtracting $mc^2$ gives the well-known result.

	\item \emph{Hunting for Lorentz invariants in electromagnetism.}
	\begin{gather*}
		\epsilon^{\alpha\beta\gamma\delta}F_{\alpha\beta}F_{\gamma\delta} = 
	\end{gather*}
	\item \emph{Deriving Maxwell's equations.} $(\epsilon_0 = \mu_0 = c = 1)$ The first equation is:
	\begin{gather*}
		\partial_{\mu}F^{\mu 0} = J^0 = \rho \\
		\implies \div \vb{E} = \rho
	\end{gather*}
	The second equation is:
	\begin{gather*}
		\partial_{\mu}F^{\mu i} = J^i = \vb J \\
		\implies -\pdv{\vb E}{t} + \curl \vb B = \vb J
	\end{gather*}
	The third equation is:
	\begin{gather*}
		\partial_{\lambda}F_{\mu 0} + \partial_0 F_{\lambda\mu} + \partial_{\mu}F_{0\lambda} = 0 \\
		\implies \curl \vb{E} + \pdv{\vb B}{t} = 0
	\end{gather*}
	The fourth equation is:
	\begin{gather*}
		\partial_{\lambda}F_{\mu i} + \partial_i F_{\lambda\mu} + \partial_{\mu}F_{i\lambda} = 0 \\
		\implies \div \vb B = 0
	\end{gather*}
	\item \emph{Deriving the continuity equation of electromagnetism.} Differentiating:
	\begin{gather*}
		\partial_{\beta}\partial_{\alpha}F^{\alpha\beta} = \partial_{\beta}J^{\beta} 
	\end{gather*}
	Since mixed partial derivatives are symmetric and $F^{\alpha\beta}$ is antisymmetric, the operation obviously gives zero:
	\begin{gather*}
		\partial_{\beta}\partial_{\alpha}F^{\alpha\beta} = \partial_{\beta}J^{\beta} = 0
	\end{gather*}
	The second equality can be interpreted as a continuity equation akin to fluid mechanics with the charge density $\rho$ and the current density $\vb J$:
	\begin{gather*}
		\pdv{\rho}{t} + \div \vb{J} = 0
	\end{gather*}
\end{subquests}


\chapter{A first stab at relativistic quantum mechanics}

\begin{subquests}
	\item \emph{Massive scalar field Lagrangian.}
	\begin{gather*}
		\mathcal L = \frac{1}{2}\pqty{\partial_{\mu}\phi}^2 - \frac{1}{2}m^2\phi^2  = \frac{1}{2}\partial_{\mu}\phi\partial^{\mu}\phi - \frac{1}{2}m^2\phi^2 \\
		\pdv{\mathcal L}{\phi} = -m^2\phi, \;\;\; \pdv{\mathcal L}{\pqty{\partial_{\mu}\phi}} = \partial^{\mu}\phi \\
		\pdv{\mathcal L}{\phi}-\partial_{\mu}\pqty{\pdv{\mathcal L}{\pqty{\partial_{\mu}\phi}}} = 0 \\
		\pqty{\partial^2 + m^2}\phi = 0 \\
		\pi = \pdv{\mathcal L}{\dot{\phi}} = \partial^0 \phi = \dot{\phi} \\
		\mathcal H = \pi\dot{\phi} - \mathcal L = \frac{1}{2}\pi^2 + \frac{1}{2}\pqty{\nabla{\phi}}^2 + \frac{1}{2}m^2\phi^2 \\
	\end{gather*}
\end{subquests}

\chapter{Examples of Lagrangians, or how to write down a theory}

\begin{subquests}
	\item \emph{Massive scalar field with a twist.}
	\begin{gather*}
		\mathcal L = \frac{1}{2}\pqty{\partial_{\mu}\phi}^2 - \frac{1}{2}m^2\phi^2 - \sum_{n=1}^{\infty}\lambda_n \phi^{2n+2}\\
		\pdv{\mathcal L}{\phi} = -m^2\phi - \sum_{n=1}^{\infty}\lambda_n \pqty{2n+2}\phi^{2n+1} \\
		\pdv{\mathcal L}{\pqty{\partial_{\mu}\phi}} = \partial^{\mu}\phi \\
		\pdv{\mathcal L}{\phi}-\partial_{\mu}\pqty{\pdv{\mathcal L}{\pqty{\partial_{\mu}\phi}}} = 0 \\
		\partial_{\mu}\partial^{\mu}\phi + m^2\phi + \sum_{n=1}^{\infty}\lambda_n \pqty{2n+2}\phi^{2n+1} = 0 \\
		\pqty{\partial^2 + m^2}\phi + \sum_{n=1}^{\infty}\lambda_n \pqty{2n+2}\phi^{2n+1} = 0
	\end{gather*}

	\item \emph{Massive scalar field with a source.}
	\begin{gather*}
		\mathcal L = \frac{1}{2}\bqty{\partial_{\mu}\phi(x)}^2 - \frac{1}{2}m^2\bqty{\phi(x)}^2 + J(x)\phi(x)\\
		\pdv{\mathcal L}{\phi(x)} = -m^2\phi(x) +J(x) \\
		\pdv{\mathcal L}{\pqty{\partial_{\mu}\phi(x)}} = \partial^{\mu}\phi(x) \\
		\pdv{\mathcal L}{\phi(x)}-\partial_{\mu}\pqty{\pdv{\mathcal L}{\pqty{\partial_{\mu}\phi(x)}}} = 0 \\
		\partial_{\mu}\partial^{\mu}\phi(x) + m^2\phi(x) - J(x) = 0 \\
		\pqty{\partial^2 + m^2}\phi(x) = J(x)
	\end{gather*}

	\item \emph{Two coupled massive scalar fields.}
	\begin{gather*}
		\mathcal L = \frac{1}{2}\pqty{\partial_{\mu}\phi_1}^2 - \frac{1}{2}m^2\phi_1^2 + \frac{1}{2}\pqty{\partial_{\mu}\phi_2}^2 - \frac{1}{2}m^2\phi_2^2 - g\pqty{\phi_1^2 + \phi_2^2}^2 \\
		\pdv{\mathcal L}{\phi_1} = -m^2\phi_1 - 4g\phi_1\pqty{\phi_1^2 + \phi_2^2} = 0, \;\;\; \pdv{\mathcal L}{\phi_2} = -m^2\phi_2 - 4g\phi_2\pqty{\phi_1^2 + \phi_2^2} = 0 \\
		\pdv{\mathcal L}{\pqty{\partial_{\mu}\phi_1}} = \partial^{\mu}\phi_1, \;\; \pdv{\mathcal L}{\pqty{\partial_{\mu}\phi_2}} = \partial^{\mu}\phi_2 \\	
		\partial_{\mu}\partial^{\mu}\phi_1 + m^2\phi_1 + 4g\phi_1\pqty{\phi_1^2 + \phi_2^2} = 0 \\
		\partial_{\mu}\partial^{\mu}\phi_1 + m^2\phi_1 + 4g\phi_2\pqty{\phi_1^2 + \phi_2^2} = 0
	\end{gather*}
	\item \emph{Introducing the conjugate momentum.} Referring to Chapter 6's solution:
	\begin{gather*}
		\Pi^{\mu} = \pdv{\mathcal L}{\pqty{\partial_{\mu}\phi}} = \partial^{\mu}\phi
	\end{gather*}
\end{subquests}


\chapter{The passage of time}

\begin{subquests}
	\item \emph{Properties of a specific form of the time-evolution operator.} Let $\hat{U}(t_1,t_2) = \exp\bqty{i\hat{H}(t_2 - t_1)}$:
	\begin{gather*}
		\hat{U}(t_1,t_1) = \exp\bqty{i\hat{H}(t_1 - t_1)} = 1 \\
		\hat{U}(t_3,t_2)\hat{U}(t_2, t_1) = \exp\bqty{i\hat{H}(t_3 - t_1)} = \hat{U}(t_3,t_1) \\
		i\dv{t_2}\exp\bqty{i\hat{H}(t_2 - t_1)} = i^2\exp\pqty{i\hat{H}t_2}\hat{H}\exp\pqty{-i\hat{H}t_1} = \hat{H}\hat{U}(t_2, t_1),\;\because \bqty{\hat U, \hat H} = 0 \\
	\end{gather*}
	The time evolution operator is unitary, so $\hat{U}^{-1} = \hat{U}^{\dagger}$. Therefore:
	\begin{align*}
		\hat{U}^{\dagger}(t_2, t_1) & = \exp\bqty{i\hat{H}(t_1 - t_2)} = \hat{U}(t_1,t_2) \\
		\hat{U}^{\dagger}(t_2, t_1)\hat{U}(t_2,t_1) & = \exp\bqty{i\hat{H}(t_1 - t_2)}\exp\bqty{i\hat{H}(t_2 - t_1)} = 1
	\end{align*}

	\item \emph{Time-dependence of ladder operators.}
	\begin{align*}
		\hat{H} & = \sum_k E_k \hat{a}_k^\dagger \hat{a}_k \\
		\hat{a}^{\dagger}_{k}(t) & = e^{i\hat{H}t/\hbar}\hat{a}^{\dagger}_{k}(0)e^{-i\hat{H}t/\hbar} \\
		\dv{\hat{a}^{\dagger}_{k}(t)}{t} & = \frac{i}{\hbar}\pqty{e^{i\hat{H}t/\hbar}\bqty{\hat{H},\hat{a}^{\dagger}_{k}(0)}e^{-i\hat{H}t/\hbar}} \\
		& = \frac{iE_k}{\hbar}\pqty{e^{i\hat{H}t/\hbar}\bqty{\hat{n}_k,\hat{a}^{\dagger}_{k}(0)}e^{-i\hat{H}t/\hbar}} = \frac{iE_k}{\hbar}\hat{a}^{\dagger}_{k}(t) \\
		\int \frac{\dd{\hat{a}^{\dagger}_{k}(t)}}{\hat{a}^{\dagger}_{k}(t)} & 	= \int \frac{iE_k}{\hbar}\dd{t} \implies \hat{a}^{\dagger}_{k}(t) = \hat{a}^{\dagger}_{k}(0)e^{iE_kt/\hbar}
	\end{align*}

	\item \emph{Time-dependence of an operator of the form $\hat X = X_{lm} \hat{a}^{\dagger}_l \hat{a}_m$.}
	\begin{align*}
		\hat{X}(t) & = e^{i\hat{H}t/\hbar}X_{lm}\hat{a}^{\dagger}_l \hat{a}_m e^{-i\hat{H}t/\hbar} \\
		\dv{\hat{X}}{t} & = 
	\end{align*}

	\item \emph{Hamiltonian of a spin-1/2 particle in a magnetic field.}
	\begin{align*}
		\dv{\hat{S}^z_H}{t} & = \frac{1}{i\hbar}\bqty{\hat{S}^z_H,\omega\hat{S}^y_H} = \frac{\omega}{i\hbar}\bqty{\hat{S}^z_H,\hat{S}^y_H} = \frac{\omega}{i\hbar}\pqty{-i\hbar\hat{S}^x_H} = -\omega\hat{S}^x_H \\
		\dv{\hat{S}^x_H}{t} & = \frac{1}{i\hbar}\bqty{\hat{S}^x_H,\omega\hat{S}^y_H} = \frac{\omega}{i\hbar}\bqty{\hat{S}^z_H,\hat{S}^y_H} = \frac{\omega}{i\hbar}\pqty{i\hbar\hat{S}^z_H} = \omega\hat{S}^z_H 
	\end{align*}
	Spin behaves like angular momentum.
\end{subquests}

\chapter{Quantum mechanical transformations}

\begin{subquests}
	\item \emph	{Generators of the translation operator.}
	\begin{align*}
		\hat{U}\pqty{\vb a} & = \exp[-i{\vb{\hat p} \cdot \vb a}] \\
		\pdv{\hat{U}\pqty{\vb a}}{\vb a}\bigg|_{{\vb a} = 0} & = -i{\vb{\hat p}}\exp[-i{\vb{\hat{p}}\cdot \vb 0}]\\
		\implies \vb{\hat{p}} & = -\frac{1}{i}\pdv{\hat{U}\pqty{\vb a}}{\vb a}\bigg|_{{\vb a} = 0}
	\end{align*}

	\item \emph	{Generators of the Lorentz group for four-vectors.}
	\begin{gather*}
		K = \frac{1}{i}\pdv{{\vb \Lambda}\pqty{\phi^1}}{\phi^1}\bigg|_{{\phi^1} = 0} = \frac{1}{i}\mqty|
			\sinh \phi^1 & \cosh \phi^1 & 0 & 0 \\
			\cosh \phi^1 & \sinh \phi^1 & 0 & 0 \\
			0 & 0 & 0 & 0 \\
			0 & 0 & 0 & 0
		|_{\phi^1 = 0}
		= -i\mqty[
			0 & 1 & 0 & 0 \\
			1 & 0 & 0 & 0 \\
			0 & 0 & 0 & 0 \\
			0 & 0 & 0 & 0
		]
	\end{gather*}
	and similarly for $\phi^i$.

	\item \emph	{Infinitesimal Lorentz transformations.}
	Going to the MCRF and composing boosts:
	\begin{gather*}
		\Lambda^{\mu}_{\;\nu} = \lim_{{\vb v}\to 0}\mqty[
			\gamma & \gamma v^1 & \gamma v^2 & \gamma v^3 \\
			\gamma v^1 & \gamma & 0 & 0 \\
			\gamma v^2 & 0 & \gamma & 0 \\
			\gamma v^3 & 0 & 0 & \gamma
		] = \mqty[
			1 &  v^1 &  v^2 &  v^3 \\
			v^1 & 1 & 0 & 0 \\
			v^2 & 0 & 1 & 0 \\
			v^3 & 0 & 0 & 1
		]
	\end{gather*}
	For an infinitesimal counter-clockwise rotations, compose the matrices:
	\begin{align*}
		\Lambda^{\mu}_{\;\nu} & = \mqty[
			1 & 0 & 0 & 0 \\
			0 & 1 & \theta^3 & 0 \\
			0 & -\theta^3 & 1 & 0\\
			0 & 0 & 0 & 1
		]
		\mqty[
			1 & 0 & 0 & 0 \\
			0 & 1 & 0 & -\theta^2 \\
			0 & 0 & 1 & 0 \\
			0 & \theta^2 & 0 & 1
		]
		\mqty[
			1 & 0 & 0 & 0 \\
			0 & 1 & 0 & 0 \\
			0 & 0 & 1 & \theta^1 \\
			0 & 0 & -\theta^1 & 1
		] = \mqty[
			1 & 0 & 0 & 0 \\
			0 & 1 & \theta^3 & -\theta^2 \\
			0 & -\theta^3 & 1 & \theta^1 \\
			0 & \theta^2 & -\theta^1 & 1
		]
	\end{align*}
	Compose the boosts and rotation matrices:
	\begin{align*}
		\Lambda^{\mu}_{\nu} & = \Lambda^{\mu}_{\;\bar{\nu}}\Lambda^{\bar{\nu}}_{\;\nu} = 
		L_z R_z L_y R_y L_x R_x \\
		\Lambda^{\mu}_{\nu} & = 
		\mqty[
			1 & v^1 & v^2 & v^3 \\
			v^1 & 1 & \theta^3 & -\theta^2 \\
			v^2 & -\theta^3 & 1 & \theta^1 \\
			v^3 & \theta^2 & -\theta^1 & 1
		]
	\end{align*}
	Extracting the identity matrix, the general infinitesimal Lorentz transformation can be written as:
	\begin{gather*}
		\vb \Lambda = \vb 1 + \mathbf{\omega} = \mqty[
			1 & 0 & 0 & 0 \\
			0 & 1 & 0 & 0 \\
			0 & 0 & 1 & 0 \\
			0 & 0 & 0 & 1
		] +	\mqty[
			0 & v^1 & v^2 & v^3 \\
			v^1 & 0 & \theta^3 & -\theta^2 \\
			v^2 & -\theta^3 & 0 & \theta^1 \\
			v^3 & \theta^2 & -\theta^1 & 0
		]
	\end{gather*}
	The following tensors are indeed antisymmetric:
	\begin{align*}
		\omega^{\mu\nu} & = \omega^{\mu}_{\;\lambda}g^{\lambda\nu} =
		\mqty[
			0 & v^1 & v^2 & v^3 \\
			v^1 & 0 & \theta^3 & -\theta^2 \\
			v^2 & -\theta^3 & 0 & \theta^1 \\
			v^3 & \theta^2 & -\theta^1 & 0
		]
		\mqty[
			1 & 0 & 0 & 0 \\
			0 & -1 & 0 & 0 \\
			0 & 0 & -1 & 0 \\
			0 & 0 & 0 & -1 
		] = \mqty[
			0 & -v^1 & -v^2 & -v^3 \\
			v^1 & 0 & -\theta^3 & \theta^2 \\
			v^2 & \theta^3 & 0 & -\theta^1 \\
			v^3 & -\theta^2 & \theta^1 & 0
		] \\
		\omega_{\mu\nu} & = g_{\mu\lambda}\omega^{\lambda}_{\;\nu} =
		\mqty[
			1 & 0 & 0 & 0 \\
			0 & -1 & 0 & 0 \\
			0 & 0 & -1 & 0 \\
			0 & 0 & 0 & -1 
		]\mqty[
			0 & v^1 & v^2 & v^3 \\
			v^1 & 0 & \theta^3 & -\theta^2 \\
			v^2 & -\theta^3 & 0 & \theta^1 \\
			v^3 & \theta^2 & -\theta^1 & 0
		] = \mqty[
			0 & v^1 & v^2 & v^3 \\
			-v^1 & 0 & -\theta^3 & \theta^2 \\
			-v^2 & \theta^3 & 0 & -\theta^1 \\
			-v^3 & -\theta^2 & \theta^1 & 0
		]
	\end{align*}

	\item \emph{Generators of the Poincar\'e group.}
\end{subquests}


\chapter{Symmetry}

\begin{subquests}
	\item \emph{Commutation relations between scalar field and its conjugate momentum.}
	\begin{gather*}
		\bqty{\phi(x),P^{\alpha}} = \phi(x)P^{\alpha} - P^{\alpha}\phi(x) =  \int \bqty{\phi(x) T^{0\alpha} - T^{0\alpha}\phi(x)}\,\dd^3{y}
	\end{gather*}

	\item \emph{Noether current of $N$-field system}.

	\item \emph{Energy-momentum tensor and momentum of the massive scalar field.}
	\begin{align*}
		T^{\mu\nu} & = \Pi^{\mu}\partial^{\nu}\phi - g^{\mu\nu}\cal{L} \\
		T^{00} & = \Pi^{0}\partial^{0}\phi - g^{00}\bqty{\frac{1}{2}\pqty{\partial_{\mu}\phi}^2 - \frac{1}{2}m^2\phi^2} = \pi\dot{\phi} - \mathcal L \\ 
		& = \frac{1}{2}\pi^2 + \frac{1}{2}\pqty{\nabla{\phi}}^2 + \frac{1}{2}m^2\phi^2 \\
		\partial_{\mu}T^{\mu\nu} & = \partial_{\mu}\bqty{\partial^{\mu}\partial^{\nu}\phi - g^{\mu\nu}\cal{L}} \\
		& = \partial^2\phi\partial^{\nu}\phi - \partial^{\mu}\phi\partial_{\mu}\partial^{\nu}\phi - \frac{1}{2}\bqty{\partial^{\rho}\phi \partial^{\nu}\partial_{\rho}\phi+ \partial_{\rho}\phi\partial^{\nu}\partial^{\rho}\phi - 2m^2\phi\partial^{\nu}\phi} \\
		& = \pqty{\partial^2 + m^2}\phi\bqty{\partial^{\nu}\phi} = 0 \\
		P^i & = \int T^{0i}\dd^3{x} = \int \pqty{\Pi^0\partial^i \phi - g^{0i}\cal{L}}\dd^3{x} \\
		& = \int \partial^0\phi\partial^i\phi\,\dd^3{x}
	\end{align*}
	The Klein-Gordon equation, which is the equation of motion for scalar field theory, satisfies the divergence of the energy-momentum tensor.

	\item \emph{Energy-momentum tensor and momentum of the electromagnetic field.}
	\begin{align*}
		{\cal L} & = -\frac{1}{4}F_{\mu\nu}F^{\mu\nu} = -\frac{1}{2}\bqty{\partial_{\mu}A_{\nu}\partial^{\mu}A^{\nu} - \partial_{\mu}A_{\nu}\partial^{\nu}A^{\mu}} \\
		\pdv{\pqty{\partial_{\mu}A_{\nu}\partial^{\mu}A^{\nu}}}{\pqty{\partial_{\sigma}A_{\rho}}} & = \delta^{\sigma}_{\mu}\delta^{\rho}_{\nu}\partial^{\mu}A^{\nu} + \partial_{\mu}A_{\nu}g^{\alpha\sigma}g^{\rho\beta}\delta^{\mu}_{\alpha}\delta^{\nu}_{\beta}= 2\partial^{\sigma}A^{\rho} \\
		\pdv{\pqty{\partial_{\mu}A_{\nu}\partial^{\nu}A^{\mu}}}{\pqty{\partial_{\sigma}A_{\rho}}} & = \delta^{\sigma}_{\mu}\delta^{\rho}_{\nu}\partial^{\nu}A^{\mu} + \partial_{\mu}A_{\nu}g^{\alpha\rho}g^{\sigma\beta}\delta^{\mu}_{\alpha}\delta^{\nu}_{\beta}= 2\partial^{\rho}A^{\sigma} \\
		\pdv{\cal L}{\pqty{\partial_{\sigma}A_{\rho}}} & = -\pqty{\partial^{\sigma}A^{\rho} - \partial^{\rho}A^{\sigma}} = - F^{\sigma\rho} = \Pi^{\sigma\rho} \\
		T^{\mu}_{\nu} & = \Pi^{\mu\sigma}\partial_{\nu}A_{\sigma} - \delta^{\mu}_{\nu}{\cal L} \\
		T^{\mu\nu} & = g^{\alpha\nu}T^{\mu}_{\alpha} = -F^{\mu\sigma}\partial^{\nu}A_{\sigma} + \frac{1}{4}g^{\mu\nu}F_{\alpha\beta}F^{\alpha\beta} \\
		X^{\lambda\mu\nu} & = F^{\mu\lambda}A^{\nu} = -F^{\lambda\mu}A^{\nu} = X^{\mu\lambda\nu} \\
		\tilde{T}^{\mu\nu} & = T^{\mu\nu} + \partial_{\nu}X^{\lambda\mu\nu} = T^{\mu\nu} + \partial_{\nu}\pqty{F^{\mu\lambda}A^{\nu}} \\
		& = -F^{\mu\sigma}\partial^{\nu}A_{\sigma} + \frac{1}{4}g^{\mu\nu}F_{\alpha\beta}F^{\alpha\beta} + \cancel{\partial_\lambda F^{\mu\lambda}A^{\nu}} + F^{\mu\lambda}\partial_{\lambda}A^{\nu} \\
		\bqty{\lambda\to\sigma} & = F^{\mu\sigma}\pqty{\partial_{\sigma}A^{\nu} - \partial^{\nu}A_{\sigma}} + \frac{1}{4}g^{\mu\nu}F_{\alpha\beta}F^{\alpha\beta} = F^{\mu\sigma}F^{\nu}_{\sigma} + \frac{1}{4}g^{\mu\nu}F_{\alpha\beta}F^{\alpha\beta} \\ 
		\tilde{T}^{00} & = F^{0\sigma}F^0_{\sigma} + \frac{1}{4}g^{00}F_{\alpha\beta}F^{\alpha\beta} = {E}^2 + \frac{1}{2}\pqty{{B}^2 - {E}^2} = \frac{1}{2}\pqty{{E}^2 + {B}^2} \\
		\tilde{T}^{i0} & = F^{i\sigma}F^0_\sigma + \cancel{\frac{1}{4}g^{i0}F_{\alpha\beta}F^{\alpha\beta}} = \epsilon^{ijk}E_j B_k = \pqty{\vb E \times \vb B}^i
	\end{align*}
\end{subquests}


\chapter{Canonical quantization of fields}

\begin{subquests}
	\item \emph{Commutation relations of quantum field position operators.} Let $\displaystyle \int \frac{\dd^3{\mathbf p}}{(2\pi)^{3/2} (2E_{\mathbf p})^{1/2}} \equiv \displaystyle \int_{\mathbf p}$:
	\begin{align*}
		\bqty{\hat{\phi}(x),\hat{\phi}(y)} & = \int_{\mathbf p} \pqty{\hat{a}_{\vb p}e^{-i p \cdot x} + \hat{a}^{\dagger}_{\vb p}e^{i p \cdot x}} \int _{\mathbf q}\pqty{\hat{a}_{\vb q}e^{-i q \cdot y} + \hat{a}^{\dagger}_{\vb q}e^{i q \cdot y}} \\
		& - \int_{\mathbf q}\pqty{\hat{a}_{\vb q}e^{-i q \cdot y} + \hat{a}^{\dagger}_{\vb q}e^{i q \cdot y}} \int_{\mathbf p} \pqty{\hat{a}_{\vb p}e^{-i p \cdot x} + \hat{a}^{\dagger}_{\vb p}e^{i p \cdot x}} \\
		& = \int \dd^3{\vb p}\int \frac{\dd^3{\vb q}}{\pqty{2\pi}^3}\frac{1}{\pqty{4E_{\vb p}E_{\vb q}}^{\frac{1}{2}}}\pqty{\bqty{\hat{a}_{\vb p},\hat{a}^{\dagger}_{\vb q}}e^{-i p \cdot x}e^{i q \cdot y} + \bqty{\hat{a}^{\dagger}_{\vb p},\hat{a}_{\vb q}}e^{i p \cdot x}e^{-i q \cdot y}} \\
		& = \int \dd^3{\vb p}\int \frac{\dd^3{\vb q}}{\pqty{2\pi}^3}\frac{1}{\pqty{4E_{\vb p}E_{\vb q}}^{\frac{1}{2}}}\pqty{\delta^{(3)}\pqty{\vb p - \vb q}e^{-i p \cdot x}e^{i q \cdot y} - \delta^{(3)}\pqty{\vb q - \vb p}e^{i p \cdot x}e^{-i q \cdot y}} \\
		& = \int\frac{\dd^3{\vb p}}{\pqty{2\pi}^3}\frac{1}{2E_{\vb p}}\pqty{e^{-i p\cdot (x - y)} - e^{i p \cdot(x - y)}} = 0, \quad \vb p \mapsto -\vb p
	\end{align*}

	\item \emph{Commutation relations of quantum field position operator and its conjugate momentum.} 
	\begin{align*}
		\bqty{\hat{\phi}(x),\hat{\Pi}^0(y)} & = \int_{\mathbf p} \pqty{\hat{a}_{\vb p}e^{-i p \cdot x} + \hat{a}^{\dagger}_{\vb p}e^{i p \cdot x}}\int _{\mathbf q}\pqty{-iE_{\vb q}}\pqty{\hat{a}_{\vb q}e^{-i q \cdot y} - \hat{a}^{\dagger}_{\vb q}e^{i q \cdot y}} \\
		& -\int _{\mathbf q}\pqty{-iE_{\vb q}}\pqty{\hat{a}_{\vb q}e^{-i q \cdot y} - \hat{a}^{\dagger}_{\vb q}e^{i q \cdot y}}\int_{\mathbf p} \pqty{\hat{a}_{\vb p}e^{-i p \cdot x} + \hat{a}^{\dagger}_{\vb p}e^{i p \cdot x}} \\
		& = i\int \dd^3{\vb p}\int \frac{\dd^3{\vb q}}{\pqty{2\pi}^3}\frac{E_{\vb q}}{\pqty{4E_{\vb p}E_{\vb q}}^{\frac{1}{2}}}\pqty{\bqty{\hat{a}_{\vb p},\hat{a}^{\dagger}_{\vb q}}e^{-i p \cdot x}e^{i q \cdot y} + \bqty{\hat{a}_{\vb q},\hat{a}^{\dagger}_{\vb p}}e^{i p \cdot x}e^{-i q \cdot y}} \\
		& = i\frac{i}{2}\int \dd^3{\vb p}\int \frac{\dd^3{\vb q}}{\pqty{2\pi}^3}\frac{E_{\vb q}}{\pqty{4E_{\vb p}E_{\vb q}}^{\frac{1}{2}}}\pqty{\delta^{(3)}\pqty{\vb p - \vb q}e^{-i p \cdot x}e^{i q \cdot y} + \delta^{(3)}\pqty{\vb q - \vb p}e^{i p \cdot x}e^{-i q \cdot y}} \\
		& = \frac{i}{2}\int\frac{\dd^3{\vb p}}{\pqty{2\pi}^3}\pqty{e^{-i p \cdot (x-y)} + e^{i p \cdot (x-y)}} = i\,\delta^{(3)}(x - y), \quad \vb p \mapsto -\vb p
	\end{align*}
\end{subquests}

\chapter{Examples of canonical quantization}

\begin{subquests}
	\item \emph{Complex scalar field theory.}
	\begin{align*}
		\hat{\mathcal H} & = \partial^0\hat{\psi}^{\dagger}\hat{\psi} + \partial^0\hat{\psi}\hat{\psi}^{\dagger} + \nabla\hat{\psi}^{\dagger}\cdot\nabla\hat{\psi} + m^2 \hat{\psi}^{\dagger}\hat{\psi} \\
		& = \int_{\mathbf q}\pqty{iE_{\vb q}}\pqty{\hat{a}^{\dagger}_{\vb q}e^{i q \cdot x} - \hat{b}_{\vb q}e^{-i q \cdot x}} \int_{\mathbf p} \pqty{\hat{a}_{\vb p}e^{-i p \cdot x} + \hat{b}^{\dagger}_{\vb p}e^{i p \cdot x}} \\
		& + \int_{\mathbf p} \pqty{-iE_{\vb p}}\pqty{\hat{a}_{\vb p}e^{-i p \cdot x} - \hat{b}^{\dagger}_{\vb p}e^{i p \cdot x}}\int_{\mathbf q}\pqty{\hat{a}^{\dagger}_{\vb q}e^{i q \cdot x} + \hat{b}_{\vb q}e^{-i q \cdot x}}  \\
		& + \int_{\mathbf q}\pqty{-i\vb q}\pqty{\hat{a}^{\dagger}_{\vb q}e^{-i q \cdot x} - \hat{b}_{\vb q}e^{i q \cdot x}} \int_{\mathbf p} \pqty{i\vb p}\pqty{\hat{a}_{\vb p}e^{-i p \cdot x} - \hat{b}^{\dagger}_{\vb p}e^{i p \cdot x}}\\
		& + m^2 \int_{\mathbf q}\pqty{\hat{a}^{\dagger}_{\vb q}e^{-i q \cdot x} + \hat{b}_{\vb q}e^{i q \cdot x}}\int_{\mathbf p} \pqty{\hat{a}_{\vb p}e^{-i p \cdot x} + \hat{b}^{\dagger}_{\vb p}e^{i p \cdot x}} \\
		& = \int\dd^3{\vb p}\int\frac{\dd^3{\vb q}}{\pqty{2\pi}^3}\frac{-iE_{\vb q}}{\pqty{4E_{\vb p}E_{\vb q}}^{\frac{1}{2}}}\pqty{\hat{a}^{\dagger}_{\vb q}\hat{a}_{\vb p}e^{i(q-p)\cdot x} - \hat{b}_{\vb q}\hat{b}^{\dagger}_{\vb p}e^{- i(\vb q -\vb p)\cdot \vb x}} \\
		& + \int\dd^3{\vb p}\int\frac{\dd^3{\vb q}}{\pqty{2\pi}^3}\frac{-iE_{\vb p}}{\pqty{4E_{\vb p}E_{\vb q}}^{\frac{1}{2}}}\pqty{\hat{a}_{\vb p}\hat{a}^{\dagger}_{\vb q}e^{i(q-p)\cdot x} - \hat{b}^{\dagger}_{\vb p}\hat{b}_{\vb q}e^{-i(q-p)\cdot x}} \\
		& + \int\dd^3{\vb p}\int\frac{\dd^3{\vb q}}{\pqty{2\pi}^3}\frac{\vb p \cdot \vb q}{\pqty{4E_{\vb p}E_{\vb q}}^{\frac{1}{2}}}\pqty{\hat{a}^{\dagger}_{\vb q}\hat{a}_{\vb p}e^{i(q-p)\cdot x} + \hat{b}_{\vb q}\hat{b}^{\dagger}_{\vb p}e^{-i(q-p)\cdot x}} \\
		& + m^2 \int\dd^3{\vb p}\int\frac{\dd^3{\vb q}}{\pqty{2\pi}^3}\frac{1}{\pqty{4E_{\vb p}E_{\vb q}}^{\frac{1}{2}}}\pqty{\hat{a}^{\dagger}_{\vb q}\hat{a}_{\vb p}e^{i(q-p)\cdot x} + \hat{b}_{\vb q}\hat{b}^{\dagger}_{\vb p}e^{-i(q-p)\cdot x}}
	\end{align*}

	\item \emph{Commutation relations of complex scalar fields.}
	\begin{subquests}
		\item
		\begin{align*}
			\bqty{\hat{\psi}(x),\hat{\psi}^{\dagger}(y)} & = \int_{\mathbf p} \pqty{\hat{a}_{\vb p}e^{-i p \cdot x} + \hat{b}^{\dagger}_{\vb p}e^{i p \cdot x}}\int_{\mathbf q}\pqty{\hat{a}^{\dagger}_{\vb q}e^{-i q \cdot x} + \hat{b}_{\vb q}e^{i q \cdot x}} \\
			& - \int_{\mathbf q}\pqty{\hat{a}^{\dagger}_{\vb q}e^{-i q \cdot x} + \hat{b}_{\vb q}e^{i q \cdot x}}		\int_{\mathbf p} \pqty{\hat{a}_{\vb p}e^{-i p \cdot x} + \hat{b}^{\dagger}_{\vb p}e^{i p \cdot x}}
		\end{align*}

		\item

	\end{subquests}
	
	\item \emph{Commutation relations of Noether charges for two scalar fields.}
	\begin{subquests}
		\item
		\begin{align*}
			\mqty[
				\phi_1' \\
				\phi_2' 
			]
			& = \mqty[
				\cos\alpha & -\sin\alpha \\
				\sin\alpha & \cos\alpha \\
			]\mqty[
				\phi_1 \\
				\phi_2
			] \\
			\bqty{\hat{Q}_N, \hat{\phi_1}} & = -iD\hat{\phi_1} = i\hat{\phi_2}
		\end{align*}

		\item
		\begin{gather*}
			\bqty{\hat{Q}_N, \hat{\phi_2}} = -iD\hat{\phi_2} = -i\hat{\phi_1}
		\end{gather*}
		
		\item
		\begin{align*}
			\bqty{\hat{Q}_N, \hat{\psi}} & = \frac{1}{\sqrt{2}}\bqty{\hat{Q}_N, \hat{\phi_1}} + \frac{i}{\sqrt{2}}\bqty{\hat{Q}_N, \hat{\phi_2}} = \frac{i}{\sqrt{2}}\hat{\phi_2} + \frac{1}{\sqrt{2}}\hat{\phi_1} = \hat{\psi} \\
			\mqty[
				\phi_1' \\
				\phi_2' 
			]
			& = \mqty[
				\cos\alpha & -\sin\alpha \\
				\sin\alpha & \cos\alpha \\
			]\mqty[
				\phi_1 \\
				\phi_2
			] \\
			\bqty{\hat{Q}_N, \hat{\phi_1}} & = -iD\hat{\phi_1} = i\hat{\phi_2},\;\;\bqty{\hat{Q}_N, \hat{\phi_2}} = -iD\hat{\phi_2} = -i\hat{\phi_1} \\
			\bqty{\hat{Q}_N, \hat{\psi}} & = \frac{1}{\sqrt{2}}\bqty{\hat{Q}_N, \hat{\phi_1}} + \frac{i}{\sqrt{2}}\bqty{\hat{Q}_N, \hat{\phi_2}} = \frac{i}{\sqrt{2}}\hat{\phi_2} + \frac{1}{\sqrt{2}}\hat{\phi_1} = \hat{\psi}
		\end{align*}
	\end{subquests}
	

	\item \emph{Using Noether's theorem to derive the number-phase uncertainty relation.}
	Note: $D\hat{\theta} = \pm 1$. Substituting:
	\begin{gather*}
		\bqty{\hat{Q}_N, \hat{\theta}} = -iD\hat{\theta} = i \\
		\bqty{\int\rho({\vb x},t)\,\dd^3{\mathbf x}, \theta({\vb x},t)} = \int\dd^3{\vb x}\;\bqty{\rho,\theta} = i
	\end{gather*}

	\item \emph{Equations of motion of non-relativistic complex scalar field theory.}
	\begin{gather*}
		\pdv{\mathcal L}{\psi} - \partial_{\mu}\pqty{\pdv{\mathcal L}{\pqty{\partial_{\mu}\psi}}} = \pdv{\mathcal L}{\psi} - \partial_{\mu}\Pi^{\mu}_{\psi} = 0 \\
		\pdv{\mathcal L}{\psi} = -V(x)\psi^{\dagger}(x), \quad \Pi^0_{\psi} = i\psi^{\dagger} \\
		\partial_0\Pi^0_{\psi} = i\partial_0 \psi^{\dagger}, \quad \partial_i \Pi^i_{\psi} = -\frac{1}{2m}\nabla^2\psi^{\dagger}\\
		\therefore i\partial_0 \psi^{\dagger} -\frac{1}{2m}\partial_i\partial^i\psi^{\dagger} - V(x)\psi^{\dagger}(x) = 0 \\
		\implies i\partial_0\psi^{\dagger} = \hat{H}\psi^{\dagger}, \quad \hat{H} = -\frac{1}{2m}\nabla^2 + \hat{V} \\
		V = 0 \implies i\pdv{\psi}{t} = -\frac{1}{2m}\nabla^2\psi \\
		iT'(t)X(x) = -\frac{1}{2m}X''(x)T(t) \\
		\frac{T'}{T} = -iE \implies T(t) = Ae^{-iEt} \\
		X'' + 2mEX = 0 \implies X(x) = Be^{ipx} + Ce^{-ipx},\; p = \sqrt{2mE} \\
		T(t)X(x) = Ae^{i(px-Et)} + Be^{-i(px - Et)}
	\end{gather*}

	\item \emph{Noether current for non-relativistic complex scalar field theory.}
	\begin{gather*}
		J^0_N = i\Psi^{\dagger}(i\Psi) + i\Psi(-i\Psi^{\dagger}) \\
		Q_{N_c} = \int \hat{\Psi}\hat{\Psi}^{\dagger} - \hat{\Psi}^{\dagger}\hat{\Psi}\;dd \\
		= \int dd\bqty{\int \frac{\dd^3{\vb p}}{\pqty{2\pi}^{\frac{3}{2}}}\hat{a}_{\vb p}e^{-i\vb p\cdot \vb x}\int \frac{\dd^3{\vb q}}{\pqty{2\pi}^{\frac{3}{2}}}\hat{a}^{\dagger}_{\vb q}e^{i\vb q\cdot \vb x} - \int \frac{\dd^3{\vb q}}{\pqty{2\pi}^{\frac{3}{2}}}\hat{a}^{\dagger}_{\vb q}e^{iq\cdot x}\int \frac{\dd^3{\vb p}}{\pqty{2\pi}^{\frac{3}{2}}}\hat{a}_{\vb p}e^{-i\vb p\cdot \vb x}} \\
		\frac{1}{\pqty{2\pi}^3}\int dd\bqty{\int \dd^3{\vb p}\int \dd^3{\vb q}\;\hat{a}_{\vb p}\hat{a}^{\dagger}_{\vb q}e^{i(\vb p - \vb q)\cdot \vb x} - \hat{a}^{\dagger}_{\vb q}\hat{a}_{\vb p}e^{-i(\vb p - \vb q)\cdot \vb x}} \\
		= \int \dd^3{\vb p}\int \dd^3{\vb q}\;\bqty{\hat{a}_{\vb p}\hat{a}^{\dagger}_{\vb q}\delta^3(\vb p - \vb q) - \hat{a}^{\dagger}_{\vb q}\hat{a}_{\vb p}\delta^3(\vb q - \vb p)} \\
		= \int \dd^3{\vb p}\bqty{\hat{a}_{\vb p},\hat{a}^{\dagger}_{\vb p}} = \vb p
	\end{gather*}
	So momentum is conserved, naturally.

	\item \emph{Transformation of the complex scalar field.}
\end{subquests}


\chapter{Fields with many components and massive electromagnetism}

\begin{subquests}
	\item Angular momentum form of internal symmetries.
	\begin{subquests}
		\item
		$\vec{J}$ represents the Levi-Civita tensor as a vector of matrices.
		\begin{gather*}
			\hat{\vec{{\vb Q}}}_{N_c} = \int \dd^3{\vb p}\;\hat{\vb A}^{\dagger}\vec{J}\hat{\vb A} 
		\end{gather*}

		\item
		The inverse transformations and resultant computations are as follows:
		\begin{gather*}
			\hat{a}_1 = \frac{1}{\sqrt 2}\pqty{\hat{b}_{-1} - \hat{b}_1},\; \hat{a}_2 = -\frac{i}{\sqrt 2}\pqty{\hat{b}_{-1} + \hat{b}_{1}},\; \hat{a}_3 = \hat{b}_0 \\
			\hat{Q}^{2}_{N_c} = \\
			\hat{Q}^{3}_{N_c} = -i\int \dd^3{\vb p}\;\pqty{\hat{a}_{1\vb p}^{\dagger}\hat{a}_{2\vb p} - \hat{a}^{\dagger}_{2 \vb p}\hat{a}_{1\vb p}} = \int \dd^3{\vb p}\;\pqty{\hat{b}^{\dagger}_{1\vb p}\hat{b}_{1\vb p} - \hat{b}^{\dagger}_{-1\vb p}\hat{b}_{-1\vb p}} \\
			J_{\hat{b}}^1 = \frac{1}{\sqrt 2}
			\mqty[
				0 & -1 & 0 \\
				1 & 0 & -1 \\
				0 & 1 & 0
			],\;
			J_{\hat{b}}^2 = -\frac{i}{\sqrt 2}
			\mqty[
				0 & 1 & 0 \\
				1 & 0 & 1 \\
				0 & 1 & 0	
			],\;
			J_{\hat{b}}^3 = 
			\mqty[
				1 & 0 & 0 \\
				0 & 0 & 0 \\
				0 & 0 & -1	
			] 
		\end{gather*}
	\end{subquests}
	
	\item Lorentz boosting and circular polarization.
	\begin{subquests}
		\item

		\item

		\item
	\end{subquests}

	\item Projection tensors.

	\item Playing with projection tensors.
\end{subquests}


\chapter{Gauge fields and gauge theory}

\begin{subquests}
	\item Quantizing the electromagnetic field tensor.

	\item The spin of the photon.
	\begin{subquests}
		\item

		\item
	\end{subquests}
\end{subquests}

\chapter{Discrete transformations}

\begin{subquests}
	\item \emph{Gamma decay of a pion.}

	\item \emph{Classification of physical quantities.}
	\begin{subquests}
		\item Magnetic flux: Vector.

		\item Angular momentum: Pseudovector.

		\item Charge: Scalar.

		\item Scalar product of vector and pseudovector: Pseudoscalar.

		\item Scalar product of two vectors: Scalar.

		\item Scalar product of two pseudovectors: Scalar.
	\end{subquests}

	\item \emph{Representations of spinors.}
	\begin{subquests}
		\item $\mathbf R(\vb{\hat x}, \theta)$

		\item $\mathbf R(\vb{\hat y}, \theta)$

		\item $\mathbf R(\vb{\hat z}, \theta)$
	\end{subquests}
\end{subquests}


\chapter{Propagators and Green's functions}
\begin{subquests}
	\item \emph{Green's function for a particle in an infinite potential well.}
	\begin{subquests}
		\item
		\begin{gather*}
			\mel{x}{\hat{H}}{\psi} = E\braket{x}{\psi} \\
			\frac{\hbar^2}{2m}\dv[2]{\psi}{x} + E{\psi} = 0, \quad V = 0 \\
			\psi_n(x) = Ae^{ikx} + Be^{-ikx},\quad k = \sqrt{2mE}/\hbar \\
			\psi(0) = \psi(a) = 0 \implies B = -A \\
			\psi_n(x) = \sqrt{\frac{2}{a}}\sin\pqty{\frac{n\pi x}{a}}
		\end{gather*}

		\item
		\begin{gather*}
			E_n = \frac{\hbar^2k^2}{2m} = \frac{\hbar^2n^2\pi^2}{2ma^2} \\
			G^+(n,t_2,t_1) = \theta(t_2 - t_1)e^{-iE_n\pqty{t_2 - t_1}}
		\end{gather*}

		\item
		\begin{gather*}
			G^+(n,\hbar\omega) = \frac{i}{\hbar\omega - E_n + i\epsilon}
		\end{gather*}
	\end{subquests}
	\newpage
	\item \emph{Green's function in the energy expression.}
	\begin{subquests}
		\item 
		\begin{align*}
			G^+_0(x,t,y,0) & = \theta(t)\ip{x(t)}{y(t)} \\
			& = \theta(t)\mel{x}{e^{-i\hat Ht}}{y} \\
			& = \theta(t)\sum_n e^{iE_n t}\ip{x}{n}\ip{n}{y} = \theta(t)\sum_n\phi_{n}(x)\phi_n^*(y)e^{-iE_n t} \\
			G^+_0(x,y,E) & = \int G^+_0(x,t,y,0) \dd{t} \\
			& = \int_{-\infty}^{\infty} \theta(t)\sum_n\phi_{n}(x)\phi_n^*(y)e^{-iE_n t}e^{iEt} \dd{t} \\
			& = \int_{0}^{\infty} \sum_n\phi_{n}(x)\phi_n^*(y)e^{-i(E-E_n)t} \dd{t}
		\end{align*}
		Using a damping factor $e^{-\epsilon t}$ to ensure convergence, then switching the order of summation and integration:
		\begin{align*}
			G^+_0(x,y,E) & = \sum_n\int_{0}^{\infty} \theta(t)\phi_{n}(x)\phi_n^*(y)e^{i(E - E_n + i\epsilon)t} \dd{t}\\
			& = \sum_n \frac{i\phi_{n}(x)\phi_n^*(y)}{E - E_n + i\epsilon}
		\end{align*}
		
		\item The integral definition of the Heaviside step function is:
		\begin{gather*}
			\theta(t) := i\int_{-\infty}^{\infty}\frac{\dd z}{2\pi}\frac{e^{-izt}}{z+i\epsilon}
		\end{gather*}
		Substituting this into the original expression and changing the order of integration:
		\begin{align*}
			G^+_0(p,t,0) & = \theta(t)e^{-iE_p t} \\
			G^+_0(p,E) & = \int_{-\infty}^{\infty}\int_{0}^{\infty}\frac{i}{2\pi\pqty{z+i\epsilon}}e^{i\pqty{E - E_p - z}t} \dd{t}\dd{z} \\
			& = \int_{-\infty}^{\infty}\frac{i}{\pqty{z+i\epsilon}}\delta{\pqty{E - E_p - z}} \dd{z} = \frac{i}{E - E_p + i\epsilon}
		\end{align*}
	\end{subquests}
	\newpage
	\item \emph{Green's function for the harmonic oscillator.}
	\begin{subquests}
		\item The one-dimensional harmonic oscillator with the corresponding forcing function $f(t)$ has the following differential equation:
		\begin{gather*}
			m\pdv[2]{t}A(t-u) + m\omega_0^2A(t-u) = \tilde F(\omega)e^{-i\omega(t-u)}
		\end{gather*}
		Using operator methods to solve the differential equation:
		\begin{align*}
			A_P(t-u) & = \frac{1}{\pqty{D^2 + \omega_0^2}}\frac{\tilde F(\omega)}{m}e^{-i\omega(t-u)} = \pqty{1 + \frac{D^2}{\omega_0^2}}^{-1}\frac{\tilde F(\omega)}{m\omega_0^2}e^{-i\omega(t-u)}, \quad D = \dv{t} \\
			& = \frac{\tilde F(\omega)}{m\omega_0^2}e^{i\omega u}\bqty{\sum_{k=0}^{\infty} \pqty{\frac{iD}{\omega_0}}^{2k}e^{-i\omega t}} = \frac{\tilde F(\omega)}{m\omega_0^2}e^{-i\omega(t-u)}\sum_{k=0}^{\infty}\pqty{\frac{\omega}{\omega_0}}^{2k}\\
			& = \frac{\tilde F(\omega)}{m\omega_0^2}e^{-i\omega(t-u)}\bqty{\frac{1}{1-\omega^2/\omega_0^2}} = -\frac{\tilde F(\omega)}{m\pqty{\omega^2 - \omega_0^2}}e^{-i\omega(t-u)} 
		\end{align*}
		Therefore the solution is:
		\begin{gather*}
			A(t-u) = c_1 \cos\omega_0 t + c_2 \sin\omega_0 t - \frac{\tilde F(\omega)}{m\pqty{\omega^2 - \omega_0^2}}e^{-i\omega(t-u)} 
		\end{gather*}

		\item The differential equation that satisfies the Green's function is:
		\begin{gather*}
			\bqty{m\pdv[2]{t} + m\omega_0^2}G(t,t') = \delta(t-t')
		\end{gather*}
		Taking the Fourier transform, rearranging and then taking its inverse:
		\begin{align*}
			-m(\omega^2 - \omega_0^2)G(\omega,t') & = \int_{-\infty}^{\infty} \delta(t - t') e^{i\omega t} \dd{t} = e^{i\omega t'} \\
			G(t,t') & = -\frac{1}{m}\int_{-\infty}^{\infty}\frac{\dd{\omega}}{2\pi}\frac{e^{-i\omega(t-t')}}{\omega^2 - \omega_0^2} 
		\end{align*}
		Using the previous result to verify the solution:
		\begin{align*}
			A(t) & = \int G(t,t') f(t') \dd{k} \\
			& = -\frac{1}{2\pi m}\int_0^{\infty}\int_{-\infty}^{\infty}\frac{\tilde F(\omega)}{\omega^2 - \omega_0^2}e^{i\omega t'}\dd{\omega}\dd{k}
			% & =  \frac{1}{2\pi i m}\int_{-\infty}^{\infty}\frac{\tilde F(\omega)}{\omega(\omega_0^2 - \omega^2)}e^{-i\omega u}\dd{\omega} \\
			% & = \frac{1}{m\omega_0^2}\int_{-\infty}^{\infty}\frac{1}{2\pi i}\bqty{\frac{1}{\omega} + \frac{1}{\omega_0^2 - \omega^2}}\tilde F(\omega)e^{-i\omega(t-u)} \dd{\omega}
		\end{align*}

		\item Taking the Laplace transform of the differential equation form of the Green's function:
		\begin{gather*}
			G(s,u) = \frac{e^{us}}{m(s^2 + \omega_0^2)}
		\end{gather*}
		Using convolution to find the inverse:
		\begin{gather*}
			G^+(t,u) = \frac{1}{m\omega_0}\int_0^{t} \delta(k-u) \sin \omega_0(t - k) \dd{k}  = \frac{1}{m\omega_0}\sin\omega_0(t-u)
		\end{gather*}

		\item The trajectory is:
	\end{subquests}
	
	\item \emph{Green's function of the Klein-Gordon equation.}
	\begin{subquests}
		\item Taking the three-dimensional Fourier transform:
		\begin{gather*}
			\int_{-\infty}^{\infty} \pqty{\nabla^2 + \mathbf k^2}G_{\mathbf k}(\mathbf x)e^{-i\vb q\cdot \vb x}\,\dd^3{\vb x} = 1 \\
			\tilde G_{\mathbf k}(\mathbf q) = \frac{1}{\mathbf k^2 - \mathbf q^2}
		\end{gather*}
		
		\item The Fourier transform of $G^+_{\mathbf k}(\mathbf x)$ with a damping factor is:
		\begin{align*}
			\tilde G^+_{\mathbf k}(\mathbf q) & = \int_{-\infty}^{\infty} -\frac{e^{i\pqty{\abs{\mathbf k} + i\epsilon}\abs{\mathbf x}}}{4\pi\abs{\mathbf x}}e^{-i\vb q\cdot \vb x}\,\dd^3{\vb x} \\ 
			& = -\frac{1}{2}\int_{-1}^{1}\int_0^{\infty}{\abs{\mathbf x}e^{-i\pqty{\abs{\mathbf q}\cos\theta -\abs{\mathbf k} - i\epsilon}\abs{\mathbf x}}} \dd{\abs{\mathbf x}}\dd\pqty{\cos\theta} \\
			& = \frac{i}{2\abs{\mathbf q}}\int_{0}^{\infty} \bqty{e^{i\abs{\mathbf q}\abs{\mathbf x}}- e^{-i\abs{\mathbf q}\abs{\mathbf x}}}e^{i\pqty{\abs{\mathbf k} + i\epsilon}\abs{\mathbf x}} \dd{\abs{\mathbf x}} \\
			& = \frac{1}{2\abs{\vb q}}\bqty{\frac{1}{\pqty{\abs{\vb k} + \abs{\vb q} + i\epsilon}}- \frac{1}{ \pqty{\abs{\vb k} - \abs{\vb q} + i\epsilon}}}
		\end{align*}

		\item
	\end{subquests}
\end{subquests}


\chapter{Propagators and fields}

\begin{subquests}
	\item \emph{Retarded field propagator for a free particle.}

\end{subquests}

\chapter{The S-matrix}

\chapter{Expanding the S-matrix: Feynman diagrams}

\chapter{Scattering theory}

\chapter{Statistical physics: a crash course}

\chapter{The generating functional for fields}

\chapter{Path integrals: I said to him, `You're crazy'}

\begin{subquests}
	\item \emph{Physicist's treatment of operators.}

	\item \emph{Path integral derivation of Wick's theorem.}
	\begin{subquests}
		\item Let
		\begin{gather*}
			I(a) = -2\int_{-\infty}^{\infty}\exp\pqty{-\frac{ax^2}{2}} \dd{x} = -2\sqrt{\frac{2\pi}{a}}
		\end{gather*}
		Differentiating under the integral sign:
		\begin{gather*}
			I'(a) = \int_{-\infty}^{\infty}x^2\exp\pqty{-\frac{ax^2}{2}} \dd{x} = \sqrt{\frac{2\pi}{a^3}}
		\end{gather*}

		\item 
		\begin{gather*}
			J_n(a) = (-2)^\frac{n}{2} \int_{-\infty}^{\infty}\exp\pqty{-\frac{ax^2}{2}} \dd{x} = (-2)^{\frac{n}{2}}\sqrt{\frac{2\pi}{a}} \\
			\dv[k]{J_n(a)}{a} = (-2)^{\frac{n+1}{2}}\sqrt{\pi}\frac{(-1/2)!}{(-1/2-k)!}a^{-\frac{1}{2}-k} \\
			\dv[n/2]{J_n(a)}{a} = (-2)^{\frac{n+1}{2}}\sqrt{\pi}\frac{\Gamma(1/2)}{\Gamma\pqty{\frac{1-n}{2}}} a^{-\pqty{\frac{n+1}{2}}} = \frac{i^n \pi}{\Gamma\pqty{\frac{1-n}{2}}}\pqty{\frac{2}{a}}^{\frac{n+1}{2}} \\
			\expval{x^n} = \frac{\int_{-\infty}^{\infty}x^n\exp\pqty{-\frac{ax^2}{2}} \dd{x}}{\int_{-\infty}^{\infty}\exp\pqty{-\frac{ax^2}{2}} \dd{x}} = \frac{i^n \sqrt{\pi}}{\Gamma\pqty{\frac{1-n}{2}}}\pqty{\frac{2}{a}}^{n/2} \\
			= \begin{cases} 0 & \forall\;n \in 2\mathbb{Z^+} + 1 \\ a^{-n/2}\prod_{k = 1}^{n/2}(2k-1) & \forall\;n \in 2\mathbb{Z^+} \end{cases} \\
			\therefore \dv[n]{J_{2n}(a)}{a} = \frac{1}{a^n}\prod_{k=1}^{n}(2k-1)
		\end{gather*}
	\end{subquests}
\end{subquests}


\end{document}